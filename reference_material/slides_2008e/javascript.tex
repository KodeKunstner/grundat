\documentclass[svgnames,handout]{beamer}

\usepackage{preamble}
\usepackage{coursespecific}
%\renewcommand*{\IncludeAnswers}{1}

\renewcommand*{\Lecture}{12}
\subtitle{JavaScript}
\author{Ken Friis Larsen\\\scriptsize (Some slides prepared together
  with Henning Niss)}


\begin{document}

\begin{frame}
  \titlepage{}
\end{frame}


% \begin{frame} 
% \frametitle{Today's Script}
% \begin{itemize}
% \item Manipulating HTML pages through DOM
% \item Ajax architecture
% \item Intelligent clients
% \end{itemize}
% \end{frame}



\begin{frame} 
\frametitle{JavaScript---Not Just For Web Applications}

\begin{itemize}
\item Windows batch scripting (Active Scripting)
\item Dashboard Widgets (Mac OS X Tiger)
\item Vista Widgets
\item Greasemonkey scripts (Firefox)
\item Unity3D game engine scripting
\item ActionScript is an other dialect of ECMAScript
\item PDF documents can contain JavaScript
\item Scripting of Adobe Photoshop
\item Can be (and is) embedded as scripting language for applications
  that needs it
\end{itemize}
  
\end{frame}


\begin{frame}[fragile] 
\frametitle{Manipulating HTML In JavaScript}

\begin{itemize}
\item We modify an HTML page in JavaScript by modifying the
  \intro{Document Object Model} (DOM)
\item The DOM is a hierarchical tree representation of an HTML document
\item Each node corresponds to either an \intro{element node}
  (corresponding to an HTML element) or a \intro{text node}
  (corresponding to the text in an HTML element).
\item We manipulate the DOM tree in two ways:
  \begin{itemize}
  \item By adding, removing, and modifying nodes
  \item By manipulating the CSS style of nodes
  \end{itemize}
\end{itemize}
\end{frame}


\begin{frame}\frametitle{DOM Navigation}

  \begin{itemize}
  \item Navigation based on the id attribute or the tag name:
    \begin{itemize}
    \item \texttt{\textit{elem}.getElementById("homer")}: get the child
      element with id \texttt{homer}
    \item \texttt{\textit{elem}.getElementsByTagName("tag")}: get
      array of child elements with tag name \texttt{tag}
    \item (often we just use \texttt{document} as element)
    \end{itemize}

  \item Navigation based on a node (\texttt{n} is a node in the
    DOM tree):
    \begin{itemize}
    \item \texttt{n.parentNode}: parent of the node;
    \item \texttt{n.childNodes}: array of all children;
    \item \texttt{n.firstChild}: the first child;
    \item \texttt{n.lastChild}: the last child;
    \item \texttt{n.childNodes[i]}: the \texttt{i}'th child.
    \end{itemize}
  \end{itemize}
\end{frame}


\begin{frame}\frametitle{DOM Manipulation}

Adding children to a node (\texttt{n} is a node in the DOM tree):
\begin{itemize}
\item \texttt{n.appendChild(new)}: add a node after all children;
\item \texttt{n.insertBefore(new,old)}: add a node before a child;
\item \texttt{n.replaceChild(new,old)}: replace a child;
\item \texttt{n.removeChild(old)}: remove a child.
\end{itemize}

\medskip{}
Frames and Tables have special DOM methods for convenience. 
\end{frame}



\begin{frame}[fragile]  
\frametitle{Using DOM}
\begin{footnotesize}
\begin{semiverbatim}\color<2->{gray}
<html><head><title>Dated Titulation</title>
<script type="text/javascript" >
function titulate(male, name) \{ ... \}
{\color<2->{black}function addParaTo(text, div) \{
   var out = \intro<2->{document.getElementById}(div);
   var par = \intro<2->{document.createElement}("p");
   par.appendChild(\intro<2->{document.createTextNode}( text ));
   out.appendChild(par);
\}}
function titulateWithDate(name, out) \{
   var today = Date();
   var text = "Hi "+titulate(true, name)+ " at " + today;
   addParaTo(text, out)
\}
</script></head><body>
  <h1>Dated Titulation</h1>
<div id="output"></div>
<script type="text/javascript" >
titulateWithDate("Ken Friis Larsen", "output");
</script></body></html>
\end{semiverbatim}
\end{footnotesize}
\end{frame}



\begin{frame}[fragile] 
\frametitle{Multiplication Table By DOM}
  
\begin{itemize}
\item File \texttt{multdom.js}
  \begin{footnotesize}
\begin{semiverbatim}
function add_multcells( table, rows, cols ) \{
  for( var r = 1; r <= rows; r = r + 1 ) \{
    var row = table.\intro{insertRow}(table.rows.length);
    for( var c = 1; c <= cols; c++ ) \{
      var cell = row.\intro{insertCell}(row.cells.length);
      var content = \intro{document.createTextNode}( r * c );
      cell.\intro{appendChild}(content);
    \}
  \}
\}      
\end{semiverbatim}
  \end{footnotesize}
\item File \texttt{simplemulttable2.html}
\begin{footnotesize}
\begin{semiverbatim}
\color{gray}<html><head><title>Simple Multiplication Table</title>
<script type="text/javascript" src="multdom.js" ></script>
</head><body><h1>Multiplication Table</h1>
  <table border="1px" \textcolor{black}{id="multtab"}></table> 
  <script type="text/javascript">
    \textcolor{black}{var tab = document.getElementById("multtab");}
    add_multcells( tab, 10, 5 );
  </script></body></html>      
\end{semiverbatim}
  \end{footnotesize}
\end{itemize}
\end{frame}

\begin{frame}[fragile=singleslide] 
\frametitle{Setting The Style From JavaScript}

\begin{itemize}
\item To manipulate the CSS style on a DOM element from JavaScript, we
  use the \intro{\texttt{style}} property.
\begin{semiverbatim}
\footnotesize\color{gray}function validate_and_warn( s, warning_id ) \{
  var warning = document.getElementById(warning_id);
  if ( validate_int(s) ) \{
    \textcolor{black}{warning.style.visibility = "hidden";}
    return true;
  \} else \{
    \textcolor{black}{warning.style.visibility = "visible";}
    return false;
  \}\}
\end{semiverbatim}

\item Or, if we want to change the CSS class of a DOM element, we use
  the \intro{\texttt{className}} property
\begin{semiverbatim}
\footnotesize\color{gray}function zebra( table ) \{
  for ( var i = 0; i < table.rows.length; i++ ) \{
    if( i % 2 == 0 ) \{
      \textcolor{black}{table.rows[i].className = "oddrow";}
    \}
  \}\}
\end{semiverbatim}
\end{itemize}
\end{frame}


\begin{frame}[fragile=singleslide]
 \frametitle{Moving Things With Time-Outs}
 \begin{footnotesize}
\begin{semiverbatim}\color{gray}
<html><head><title>Magic Box</title>
<style type="text/css">
 .magic \{ position: absolute; ... \}
</style>
<script type="text/javascript" >
{\color{black} var pos = 0;
 function moveBox() \{
   var box = document.getElementById("box");
   pos += 5;
   box.style.left = pos + "px";
 \}}
</script></head><body>
  <h1>The Magic Box</h1>
  <div id="box" class="magic"><p>Lo and Behold</p></div>
  <script type="text/javascript" >
  {\color{black}\intro{setInterval}(moveBox, 100);}
  </script>
</body></html>
\end{semiverbatim}
 \end{footnotesize}
\end{frame}

\begin{frame}[fragile=singleslide]
\frametitle{Interaction With Forms}

\begin{itemize}
\item Just like we can use \texttt{setInterval} to get a function
  called with a specific interval, we ask to get a JavaScript function
  called when \intro{user interaction event} happens. 

\item That the user clicks on a button, for instance.
\begin{semiverbatim}
  <input type="submit" value="Titulate!" 
         \intro{onclick="return inputchanged();"} />
\end{semiverbatim}

\item We can read or change the content of an text-input field by
  using the \texttt{value} attribute
\begin{semiverbatim}
  var name = document.getElementById("name");
  if ( name.\intro{value} != "" ) \{ ... \}
\end{semiverbatim}
\end{itemize}

\end{frame}



\begin{frame}[fragile]  
\frametitle{Getting Input From The User}
\begin{footnotesize}
\begin{semiverbatim}\color<2->{gray}
<html><head><title>Dynamic Titulation</title>
<script type="text/javascript" >
function titulate(male, name) \{ ... \}
function addParaTo(text, div) \{ ... \}
function titulateWithDate(name, out) \{ ... \}
{\color<2->{black}function inputReady() \{
  var name = document.getElementById("name");
  if ( name.value != "" ) \{ titulateWithDate(name.value, "output"); \}
  return false;
\}}
</script></head><body>
  <h1>The Titulator 2</h1>
  <form name="titulator">
  Name: <input type="text" name="name" id="name" />
  <input type="submit" value="Titulate!" 
         \intro<2->{onclick="return inputReady();}" /> 
  </form> 
  <div id="output"></div>
</body></html>
\end{semiverbatim}
\end{footnotesize}
\end{frame}



\begin{frame}[fragile=singleslide] 
\frametitle{Multiplication Table---Now With CSS}
\begin{tiny}
\begin{verbatim}
<html><head><title>Dynamic Multiplication Table With CSS Tricks</title>
<style type="text/css">
  .warning { ... ;  visibility: hidden; }
  .oddrow { background-color: silver; }
</style>
<script type="text/javascript" src="multcss.js" ></script>
<script type="text/javascript">
...
function inputchanged() {
  var cols = document.getElementById("cols");
  var rows = document.getElementById("rows");
  if ( cols.value != "" && rows.value != "" 
    && validate_and_warn(cols.value, "colswarning") 
    && validate_and_warn(rows.value, "rowswarning")) {   
     var tab = document.getElementById("multtab");
     clear_table( tab );
     add_multcells( tab, rows.value, cols.value );
     zebra( tab );
  }
}
</script></head>
<body> <h1>Dynamic Multiplication Table</h1>
  <form >
  Rows: <input type="text" name="rows" id="rows" onchange="inputchanged()" /> 
  <span id="rowswarning" class="warning">Number of rows must be an integer. 
  <img src="angry.png" /></span>
  <br />
  Columns: <input type="text" name="cols" id="cols" onchange="inputchanged()" />
  <span id="colswarning" class="warning">Number of columns must be an integer.
  <img src="angry.png" /></span>
  </form> 
  <table border="1px" id="multtab"></table>
</body></html>
\end{verbatim}
\end{tiny}
\end{frame}



\begin{frame}[fragile=singleslide] 
\frametitle{Shopping List With DOM}

\begin{scriptsize}
\begin{verbatim}
<html><head><title>Shopping Basket</title>
 <script type="text/javascript">
 function addToBasket() {
   var item = this;
   var li = item.parentNode;
   var ul = li.parentNode;
   var basket = document.getElementById("basket");
   basket.appendChild(item);
   item.onclick = null;
   ul.removeChild(li);
 }
 function addOnClicks() {
   var list = document.getElementById("list");
   var items = list.getElementsByTagName("li");
   for( var i = 0; i < items.length; i++) {
     items[i].firstChild.onclick = addToBasket;
   }}       
 </script>
 <body onload="addOnClicks()"><h1>Shopping List</h1>
  <ul id="list">
  <li><span id="beer">Beer</span></li> <li><span id="chips">Chips</span></li>
  <li><span id="fish">Fish</span></li> <li><span id="salted">Salted Peanuts</span></li></ul> <hr>
  <h1>Basket</h1>
  <div id="basket"></div> </body></html>
\end{verbatim}
\end{scriptsize}  
\end{frame}


\begin{frame}[fragile] 
\frametitle{Radio Buttons And Check-boxes}

\begin{itemize}
\item Like text-input fields radio buttons have an \texttt{value} attribute.
\item We can read or set the state of an radio button using the
  \texttt{checked} attribute.
\item When we give a group radio buttons the same name, the browser
  collects them in an array with that name.
\begin{semiverbatim}\footnotesize{}
 function reportSex() \{
    for(var i = 0; i < document.questionnaire.sex.length; i += 1) \{
      var s = document.questionnaire.sex[i];
      if( s.checked ) \{
         alert("You are: "+s.value);
      \}
    \}
 \}    
\end{semiverbatim}
\item And similar for check boxes
\end{itemize}

  
\end{frame}



\end{document}

%%% Local Variables: 
%%% mode: latex
%%% TeX-master: t
%%% End: 
