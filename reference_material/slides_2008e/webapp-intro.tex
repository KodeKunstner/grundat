\documentclass[dvipsnames,handout]{beamer}

\usepackage{preamble}
\usepackage{coursespecific}


\renewcommand*{\Lecture}{11}

\begin{document}

\begin{frame}
  \titlepage{}
\end{frame}




\begin{frame}
\frametitle{Server-Side Web Scripting---A Three-tier Architecture}

\begin{center}
  \begin{tikzpicture}[thick,>=stealth]
    \small
    \fill[blue!25] (-1.5,.6) rectangle (5.5,1.8);
    \path 
    (-4.5,1.2) node[draw,ellipse,minimum size=10mm](Client) {Client}
    (0,1.2) node[draw,rectangle,minimum size=10mm,fill=white](WS) {Web Server}
    (4,1.2) node[draw,rectangle,,minimum size=10mm,fill=white](DB) {Database Server}
    ;
    \draw[<-] (Client.north east) -- node[fill=white]{1} (WS.north west);
    \uncover<2->{\draw[->] (Client) -- node[fill=white]{2} (WS);}
    \uncover<3->{\draw[<->] (WS) -- node[fill=blue!25]{3} (DB);}
    \uncover<4->{\draw[->] (WS.south west) -- node[fill=white]{4} (Client.south east);}

  \end{tikzpicture}
\end{center}


\begin{enumerate}
\item A client obtains an HTML page with an HTML form from a Web server.
\item<2-> The client completes the HTML form, and sends it back to the
  Web server.
\item<3-> The Web server executes a program that processes data from
  the client, for instance by updating a database.
\item<4-> The Web server sends an HTML page back to the client.
\end{enumerate}

% If the HTML code is simple and does not contain too much JavaScript,
% the page can be shown by most browsers.

% The HTML code is normally presented quickly by the browser, but the
% user interface is not like a normal desktop program.
\end{frame}



\begin{frame}
\frametitle{Why Should You Care About Learning HTML?}

\begin{itemize}
  
\item To be certain that a Web site works with all browsers
  
\item There are things that cannot be done with a WYSIWYG HTML editor
  
\item To reuse HTML code written by others
  
\item WYSIAYG = What You See Is All You Get; sometimes HTML editors
  generate unnecessary HTML
  
\item For Web programming, one writes programs that \emph{generate}
  HTML; so Web programming requires that one understands HTML

\end{itemize}
\end{frame}


% \begin{frame}
%   \frametitle{Why Use An HTML Editor?}

%   Why use a HTML editor like Microsoft Frontpage, DreamWeaver, or Nvu rather
%   than writing the HTML by hand?

%   \begin{uncoverenv}<2->
%     \begin{itemize}
%     \item It always generate correct HTML (or so it should)
%     \item One can learn HTML from it (but be aware of browser specific
%       code)
%     \item It is easier to get going
%     \end{itemize}
%   \end{uncoverenv}
% \end{frame}


\begin{frame}[fragile]
\frametitle{HTML in 21 minutes}

A legal HTML document (\examplefile{legal.html}):

\begin{small}
\begin{verbatim}
<html>
  <head>
     <title>Ken's Page</title>
  </head>
  <body>
    <h2>Ken's Marvellous Home Page</h2>
      I teach at the
      <a href="http://www.ku.dk">
       University of Copenhagen
      </a>
  </body>
</html>
\end{verbatim}
\end{small}

\begin{itemize}
\item Try to format HTML code nicely and consistent

\item Write the HTML code yourself

\item In web applications you write programs that generate HTML code
\end{itemize}

\end{frame}

\begin{frame}
\frametitle{General Page Layout}

You should know about the following tags in HTML:

\begin{itemize}
\item Headings: \texttt{<h1>}\ldots \texttt{</h1>}, \ldots, \texttt{<h4>}\ldots \texttt{</h4>}
\item Rules: \texttt{<hr />}
\item Paragraphs and line breaks: \texttt{<p>}\ldots \texttt{</p>},
  \texttt{<br />}
\item Quotes: \texttt{<blockquote>}\ldots \texttt{</blockquote>}
\item Centering: \texttt{<center>}\ldots \texttt{</center>}
\item Bold text: \texttt{<b>}\ldots \texttt{</b>}
\item Italic text: \texttt{<i>}\ldots \texttt{</i>}
\item Underlined text: \texttt{<u>}\ldots \texttt{</u>}
\item Ordered lists: \texttt{<ol>}\ldots \texttt{</ol>}
\item Unordered lists: \texttt{<ul>}\ldots \texttt{</ul>}
\item List items: \texttt{<li>}\ldots \texttt{</li>}
\end{itemize}
\end{frame}

\begin{frame}
\frametitle{Hyper Links, Tables, and Images}
\begin{itemize}
\item Hyper links: \texttt{<a href="link">name</a>}
\item Local \emph{named} hyper links: \texttt{<a
    name="thename">}\ldots\texttt{</a>}
\item References to a name: \texttt{<a href="index.html\#thename">The Name</a>}
\item Mail-to links: \texttt{<a href="mailto:kfl@itu.dk">kfl@itu.dk</a>}
\item Tables: \texttt{<table>}\ldots \texttt{</table>}, \texttt{<tr>}\ldots
  \texttt{</tr>}, \texttt{<th>}\ldots \texttt{</th>} and \texttt{<td>}\ldots
  \texttt{</td>}
\item Images: \texttt{<img src="pluto.jpg" />}
\item Colors: use Cascading Style Sheets (CSS)
\end{itemize}
\end{frame}

\begin{frame}[fragile]
\frametitle{HTML forms}
  
A form can contain text areas (\verb+<textarea>+),
input fields (\verb+<input>+), and menus (\verb+<select>+)

Example: \examplefile{formular.html}

\begin{verbatim}
   <form action="mailto:kflarsen@diku.dk" 
         method="post" enctype="text/plain">
     ... more code ...
     <input type="reset" value="Start Over!" />
     <input type="submit" value="Send Form" />
   </form>
\end{verbatim}

Attributes to the \verb+<form>+ tag \texttt{action}, \texttt{method}( ,
and sometimes \texttt{enctype}) are important.  But postpone their
treatment till later.

% \begin{itemize}
% \item \verb+action="mailto:kflarsen@diku.dk"+:\\ the completed form is sent
%   by email
  
% \item \verb+method="post"+ should be used when the completed form is
%   sent by email
  
% \item \verb+enctype="text/plain"+ sends the form as ordinary text
%   (not encoded)
% \end{itemize}

A click on \verb+<input type="submit" value="Send Form" />+ sends the
completed form.

A click on \verb+<input type="reset" value="Start Over" />+ resets all
fields, text areas, and menues.
\end{frame}

\begin{frame}[fragile]
\frametitle{Text Areas --- \texttt{<textarea>}}
  
  In a text area (\verb+<textarea>+) any text can be written:

\begin{verbatim}
   <textarea name="comments" rows="5" cols="80">
     Write your comments here!
   </textarea>
\end{verbatim}

The attributes:

\begin{itemize}
\item \verb+name+ specifies a name for the field

\item \verb+rows+ specifies the number of lines (rows) in the text area
  
\item \verb+cols+ specifies the number of characters (columns) in each
  line in the text area
\end{itemize}


\end{frame}

\begin{frame}[fragile]
\frametitle{Input Fields --- \texttt{<input>} Text and Passwords}
  
  The attribute \verb+type+ determines the type of the input field:

\begin{itemize}
\item Single-line text areas:\\
\verb+<input type="text" name="lastname" />+


\item Passwords, which are not to be displayed by the browser:\\
  \verb+<input type="password" name="studentid" />+
  
\end{itemize}
\end{frame}
 

\begin{frame}[fragile]
\frametitle{Input Fields --- \texttt{<input>} Check Boxes and Buttons }
  
  More \texttt{input} types:

\begin{itemize}
\item Check boxes, possibly more
  than one item can be checked at a time:
  \begin{small}
\begin{verbatim}
<input type="checkbox" name="paper" value="nytimes" /> 
<input type="checkbox" name="paper" value="wpost" />
<input type="checkbox" name="paper" value="guardian" />
\end{verbatim}  
  \end{small}

\item Radio buttons, only one item can be checked at a time:
  \begin{footnotesize}
\begin{verbatim}
<input type="radio" name="sex" value="male" />
<input type="radio" name="sex" value="female" checked="yes" />
\end{verbatim}
  \end{footnotesize}
\item Button for resetting a form:
  \begin{small}
\begin{verbatim}
<input type="reset" ... />
\end{verbatim}
  \end{small}
\item Button for sending a completed form:
  \begin{small}
\begin{verbatim}
<input type="submit" ... />
\end{verbatim}
  \end{small}
\end{itemize}
\end{frame}





\begin{frame}[fragile]
\frametitle{Menu Choices, \texttt{<select>}}

\begin{itemize}
\item A drop-down menu, which allows the user so choose between a
  series of items (+<option>+):
  \begin{small}
\begin{verbatim}
<select name="order">
  <option value="9">Pepperoni Pizza</option>
  <option value="70">Pizza Bambino</option>
  <option value="47">Chicken Dipper (9 pieces)</option>
</select>
\end{verbatim}
  \end{small}
\item The attribute \verb+multiple+ in the \verb+<select>+ tag allows
  the user to choose more than one item.
\end{itemize}


\textbf{Complete form}

\begin{itemize}
\item Bare, with no layout structure: \examplefile{formular.html}
\item An example course evaluation with a nicer lay-out (using
  \verb+<table>+): \examplefile{formular2.html}
\end{itemize}

\end{frame}

\begin{frame}[fragile]
\frametitle{Completed Questionnaire, When Sent By Email}

\textbf{With} \verb+enctype="text/plain"+:

\begin{small}
\begin{verbatim}
 lastname=Olsen
 studentid=da4567
 sex=male
 paper=guardian
 comments=This is not a nice questionnaire.
 order=9
\end{verbatim}
\end{small}


\textbf{Without} \verb+enctype="text/plain"+:

\begin{small}
\begin{verbatim}
 lastname=Olsen&studentid=da4567&sex=male&paper=guardian
 &comments=This+is+not+a+nice+questionnaire.&order=9
\end{verbatim}
\end{small}
\end{frame}

\begin{frame}[fragile]
\frametitle{Find five bugs}

In an HTML document without consistent indentation:
\begin{footnotesize}
\begin{verbatim}
  <html><head><title>Hello</title></head>Hello<center>great page<p>
  this is in <b>bold<i> and italics</html>
\end{verbatim}
\end{footnotesize}
\begin{uncoverenv}<2->
  In an HTML document with consistent indentation:
  \begin{footnotesize}
\begin{verbatim}
  <html>
    <head>
      <title>Hello</title>
    </head>
    Hello
    <center>
      great page
      <p>
      this is in <b>bold<i> and italics
  </html>
\end{verbatim}
   \end{footnotesize}
%   Firefox shows this document without errors! Or does it?
 \end{uncoverenv}
\end{frame}


\begin{frame}
  \frametitle{Cascading Style Sheets (CSS)}
  
  \begin{itemize}
  \item We don't want to mix the logical \intro{structure} with the
    \intro{presentation} of our webpages
    \begin{itemize}
    \item Use HTML for the \emph{logical} structure
    \item Use Cascading Style Sheets (CSS) for the presentation
    \end{itemize}
  \item Advantages of CSS
    \begin{itemize}
    \item We get a consistant look of our pages
    \item We only have to change the layout in one place
    \item We can tailor the presentation to different devices (medias)
      by using different style sheets
    \end{itemize}
\end{itemize}
  
\end{frame}

\begin{frame}[fragile=singleslide]
  \frametitle{CSS Structure}
  
  \begin{itemize}
  \item A style sheet consists of a set of \intro{rules}.  Each rule
    consists of a \intro{selector} and a set of \intro{styles}. A
    style set a \intro{property} to a given \intro{value}.

  \item Example of a CSS rule with two styles:
    \begin{small}
\begin{verbatim}
  p { background-color: red;
      border: 3px solid gray;
  }
\end{verbatim}
        This rule specifies that all \texttt{<p>} elements should have
        a red background and a boder that is three pixels think,
        solid, and gray. 
      \end{small}

    \item We can style multiple elements by using a comma in the selector:
      \begin{small}
\begin{verbatim}
  h1, h2 { border-bottom: thin solid blue; }
\end{verbatim}
        This rules specifies that both \texttt{<h1>} and \texttt{<h2>}
        elements should have a thin, solid, blue border at the bottom.
      \end{small}
  \end{itemize}

\end{frame}


\begin{frame}[fragile=singleslide]
  \frametitle{CSS Continued}

  \begin{itemize}
  \item If we want \texttt{<a>} links in \texttt{<h2>} headings to be
    orange, but \emph{only} those links, we need a \intro{decendant
    selector}:
    \begin{small}
\begin{verbatim}
  h2 a { color: orange; }
\end{verbatim}
    \end{small}

  \item We can put our CSS in a \texttt{<style>} element in the
    \texttt{<head>} element of our document:
    \begin{footnotesize}\color{gray}
\begin{semiverbatim}
  <html>
  <head>...
    {\color{black}<style type="text/css">
      ...
    </style>}
  </head>
\end{semiverbatim}
    \end{footnotesize}
  \item Or we can put it in an external file, say,
    \texttt{styling.css}.  And then include it in the \texttt{<head>}
    instead:
    \begin{footnotesize}\color{gray}
\begin{semiverbatim}
  <html>
  <head>...
    {\color{black}<link rel="stylesheet" type="text/css" href="styling.css">}
  </head>
\end{semiverbatim}
    \end{footnotesize}
    Which is better, because then we can reuse the CSS.
  \end{itemize}
\end{frame}



\begin{frame}[fragile]
  \frametitle{Giving Our Form Some Style}
  
  \begin{itemize}
  \item Say, we want our form to have a blue background, white text, a
    grey dotted border, and nice wide margins.

  \item Thus, we add the following \intro{\texttt{style} element} to the
    \texttt{head} of our HTML form:
    \begin{footnotesize}\color{gray}
\begin{semiverbatim}
<html>
<head><title>Form Example</title>
{\color{black}<style type="text/css">
  body \{ background-color: blue;
         color: white;
         margin-left: 20%;
         margin-right: 20%;
         border: 3px dotted gray;
         padding: 10px 10px 10px 10px;
  \}
</style>}
</head>
<body>
<h1>Questionnaire</h1>
...
</body>
\end{semiverbatim}
    \end{footnotesize}
  \end{itemize}
\end{frame}


\begin{frame}[fragile]
  \frametitle{Style An Element With Some Class}

  \begin{itemize}
  \item What if we want one of our elements to stand out?
  \item<2-> Add a \intro{\texttt{class}} attribute:
    \begin{small}\color{gray}
      \begin{semiverbatim}
<p \textcolor{black}{class="thesexquestion"}>
Enter sex: Male <input type="radio" ...
      \end{semiverbatim}
    \end{small}
  \item<3-> Add a rule for the class in the \texttt{style} element:
    \begin{small}\color{gray}
\begin{semiverbatim}
<style type="text/css">
  body \{ ... \}
  \textcolor{black}{.thesexquestion \{ 
     background-color: rgb(50%, 0%, 0%);
  \}}
</style>
\end{semiverbatim}
    \end{small}
    (Selectors for classes starts with a '.')
  \item<4->  Use \intro{\texttt{<div>}} and \intro{\texttt{<span>}}
    elements to give extra structure to your page if needed.
    \begin{itemize}
    \item Use \texttt{<div>} for big blocks spanning multible
      paragraphs and/or headings.
    \item Use \texttt{<span>} for highlighting some inline texts, like
      a word or a sentence.
    \end{itemize}
  \end{itemize}



\end{frame}





% \begin{frame}
% \frametitle{Some HTML Rules}

% \begin{itemize}
% \item Avoid large tables for page layout (difficult and slow for
%   browsers to render).
% \item Be careful with redefinitions of link colors.%\item så lidt formattering som muligt
% \item Make it possible for the user to construct black-and-white
%   printouts (e.g., some web browsers prints white text on a blue
%   background as white text on a white background!)
% \item Think about the user's screen as a limited resource. 
% \item Avoid the use of frames (frames do not work well with printing,
%   bookmarks, or email).
% \item Remember to give your HTML pages a title (used for bookmarks,
%   e.g.)
% \end{itemize}

% A goal could be that both Firefox and Internet Explorer
% can show your page.

% What about WebTV, Palm Pilots, mobile phones, \ldots

% \end{frame}

% \begin{frame}
% \frametitle{More Web Design Rules}

% \begin{itemize}
% \item Let information be the user interface: By a click on an
%   informative word, the user is sent to a page with more in-depth
%   information about the concept (avoid ``here-links'').
% \item Provide users with a wide, flat overview of the Web site,
%   instead of a sequential or deep tree structure.
% \item Organize your Web site according to user interests, instead of
%   after the internal organization.
% \item Why use icons for navigations when words are more intuitive and
%   take up less space?
% \item Construct long documents instead of wide documents.
% \item Remember to put a link back to the index page on every page ---
%   for instance, to support users coming from search engines.
% %\item Udstyr dit web-site med en fuldtekst søgemaskine---og sørg
% %  for, at den holder statistik med de ord, som den ikke kan finde svar
% %  på.
% \item Add contact information to your pages.
% %\item A vertical menu on a page takes less space than a horizontal
% %  menu
% \end{itemize}
% \end{frame}



% \begin{frame}
% \frametitle{Use of Applets for Updating a Database}

% \begin{center}
%   \begin{tikzpicture}[thick,>=stealth]
%     \small
%     \fill[blue!25] (-1.5,.6) rectangle (5.5,1.8);
%     \path 
%     (-4.5,1.2) node[draw,ellipse,minimum size=10mm](Client) {Client}
%     (0,1.2) node[draw,rectangle,minimum size=10mm,fill=white](WS) {Web Server}
%     (4,1.2) node[draw,rectangle,,minimum size=10mm,fill=white](DB) {Database Server}
%     ;
%     \draw[<-] (Client) -- node[fill=white]{1} (WS);
%     \uncover<2->{\draw[<->] (Client) .. controls +(-40:20mm) and +(-120:20mm)
%       .. 
%      node[fill=white]{2} (DB);}
%   \end{tikzpicture}
% \end{center}
% % \begin{center}
% % \includegraphics[width=10cm]{Images/applet}
% % \end{center}

% \begin{enumerate}
% \item A client obtains an HTML page and an associated applet from a Web server.
% \item<2-> The client starts executing the applet as a program; requests
%   and updates are send directly to the database.
% \end{enumerate}

% \begin{uncoverenv}<3->
%   Comments:
%   \begin{itemize}
%   \item Applets work well only on newer browsers and applets take a
%     long time to download.

%   \item On the other hand, they can be more interactive.
  
%   \item We do not discuss applets and other similar technologies
%     (e.g., Flash) in this course.

% \end{itemize}
%   \end{uncoverenv}

% \end{frame}

% \begin{frame}[fragile]
% \frametitle{What is a Programming Language?}

% \begin{itemize}
% \item A programming language is a notation for instructing a computer
%   what to do (i.e., a notation for programs).
  
% \item There are many different programming languages, including: PHP,
%   Perl, Tcl, Standard ML, Java, C\#, C, C++, Erlang, Haskell, Pascal,
%   Forth, Fortran, Prolog, Ada, Python, Common~Lisp, Scheme, \ldots
% \end{itemize}

% \end{frame}

% \begin{frame}[fragile]
%   \frametitle{Programmers needs to be careful to get the syntax right!}

% \begin{itemize}
% \item A computer is normally very pedantic. 

% \item This issue is particularly frustrating to novice programmers.
% \end{itemize}

% Example: A Correct PHP command:

% \begin{verbatim}
%    echo "I'm alive!";
% \end{verbatim}

% A very wrong PHP command:

% \begin{verbatim}
%    Echo "I'm alive!;
% \end{verbatim}

% Experienced programmers are good at locating errors.

% \end{frame}


% \begin{frame}[fragile]
% \frametitle{Introduction to Problem Set 1}

% \begin{itemize}
% \item HTML home page
% \item HTML course overview with tables and links
% \item Use of HTML forms
% \end{itemize}

% \textbf{Material}
% \begin{itemize}
% \item Problem Set: \url{\problemset{1}}
% \item Sestoft and Larsen's HTML overview:
%    \url{\kursushjemmeside{html.html}}
%   \end{itemize}
% \end{frame}

\begin{frame}[fragile=singleslide] 
\frametitle{Writing HTML Files From Python}

\begin{small}
\begin{verbatim}
def writehello():
    f = open('hello-from-pyton.html','w')
    content = """<html>
                 <head><title>Hello World</title></head>
                 <body>
                 Hello from Python
                 We are """+ str(1+1+1+1) +
                 """ people in the room
                 </body>
                 </html>"""
    f.write(content)
    f.close()
\end{verbatim}
\end{small}
  
\end{frame}


\begin{frame}[fragile=singleslide]
  \frametitle{Javascript: History Repeats Itself}
  
  \begin{small}
\begin{semiverbatim}\color{gray}
<html><head><title>Hello JavaScript World</title></head>
<body>
 {\color{black}<script type="text/javascript">
  document.writeln("<p>Hi There,</p>");
  document.writeln("<p>Greetings from a JavaScript script</p>");   
 </script>} 
</body>
</html>
\end{semiverbatim}
  \end{small}

  \begin{itemize}
  \item The JavaScript code is kept in a \intro{\texttt{<script>}}
    element with the \intro{\texttt{type }attribute} set to
    \texttt{text/javascript}
  \item The \texttt{<script>} element can be placed
    \begin{itemize}
    \item \intro{in the \texttt{<body>} element}, when the JavaScript
      dynamically create the page content
    \item \intro{in the \texttt{<head>} elements}, when we define functions
      to be called later.
    \item \intro{in separate files}, when we have more than a trivial amount
      of code
    \end{itemize}
  \item We can use \intro{\texttt{document.writeln}} to dynamically
    write to the page.
  \end{itemize}
\end{frame}


\begin{frame}[fragile=singleslide] 
\frametitle{Functions In JavaScript}

\begin{footnotesize}
\begin{semiverbatim}\color{gray}
<html><head><title>Oooh, Now With Function Calls</title>
{\color{black}<script type="text/javascript" >
function titulate( $male, $name ) \{
  if ( $male ) \{
    return "Mr. " + $name;
  \} else \{
    return "Mrs. " + $name;
  \}
\}
</script>}
</head>
<body>
 {\color{black}<script type="text/javascript">
  var today = Date();
  document.writeln("<h1>Hi "
                  + titulate( true, "Ken Friis Larsen")+"</h1>");
  document.writeln("<p>You loaded this webpage at " + today + "</p>");     
 </script>}
</body></html>
\end{semiverbatim}%$
\end{footnotesize}
\end{frame}


\begin{frame} 
\frametitle{Some Differences Between Python and JavaScript}

\begin{itemize}
\item In JavaScript variables must be declared with
  \intro{\texttt{var}}
\item JavaScript don't have multiline strings
\item Indentation has no semantic meaning in JavaScript
\item Lists (Python) are called Arrays (JavaScript)
\item In JavaScript we modify the document in the browser rather
  sending it
\item (Completely different OO system, but fortunately that is outside
  the scope of this course)
\end{itemize}  
\end{frame}



\begin{frame}[fragile] 
\frametitle{Loops In JavaScript}
  
\begin{itemize}[<+->]

\item The file \texttt{mult.js}:
  \begin{footnotesize}
\begin{verbatim}
function multcells( rows, cols ) {
  var result = "";
  for( var r = 1; r <= rows; r = r + 1 ) {
    result += "<tr>";
    for( var c = 1; c <= cols; c++ ) {
      result += '<td align="right">' + (r*c) + "</td>";
    }
    result += "</tr>";
  }
  return result; 
}      
\end{verbatim}
  \end{footnotesize}

\item The file \texttt{simplemulttable.html}%
\begin{footnotesize}%
\begin{semiverbatim}
\color{gray}<html><head><title>Simple Multiplication Table</title>
\textcolor{black}{<script type="text/javascript" \intro{src="mult.js"} ></script>}
</head><body>
  <h1>Multiplication Table</h1>
  <table border="1px">
  <script type="text/javascript">
    \textcolor{black}{document.writeln( multcells(5,5) );}
  </script></table></body></html>
\end{semiverbatim}
\end{footnotesize}
\end{itemize}
\end{frame}


\begin{frame} 
\frametitle{History}

\begin{itemize}
\item Early 1995, Netscape hired Brendan Eich to make LiveScript for
  version 2.0 of the browser
\item Late 1995, Netscape and Sun made a deal, and LiveScript was
  renamed JavaScript
\item August 1996, Microsoft introduced JScript in version 3.0 of Internet
  Explorer
\item November 1996, Netscape submitted the JavaScript specification
  to ECMA International for standardization.
\item June 1997, ECMAScript (ECMA-262) first edition released.
\item June 1998, ECMAScript second edition
\item December 1999, ECMAScript third edition.  Corresponds roughly
  to JavaScript version 1.5
\item Work in progress, ECMAScript fourth edition.
\end{itemize}
  
\end{frame}


\begin{frame} 
\frametitle{JavaScript---Not Just For Web Applications}

\begin{itemize}
\item Windows batch scripting (Active Scripting)
\item Dashboard Widgets (Mac OS X Tiger)
\item Vista Widgets
\item Greasemonkey scripts (Firefox)
\item Unity3D game engine scripting
\item ActionScript is an other dialect of ECMAScript
\item PDF documents can contain JavaScript
\item Scripting of Adobe Photoshop
\item Can be (and is) embedded as scripting language for applications
  that needs it
\end{itemize}
  
\end{frame}


\begin{frame}[fragile]  
\frametitle{Getting Input From The User}
\begin{footnotesize}
\begin{semiverbatim}\color<2->{gray}
<html><head><title>Dynamic Titulation</title>
<script type="text/javascript" >
function titulate( male, name ) \{ ... \}
{\color{black}function inputchanged() \{
  var name = document.getElementById("name");
  if ( name.value != "" ) \{   
    var out = document.getElementById("output");
    var par = document.createElement("p");
    var text = "Hi "+titulate(true, name.value);
    par.appendChild(document.createTextNode( text ));
    out.appendChild(par);
  \}
\}}
</script></head><body>
  <h1>The Titulator 2</h1>
  <form name="titulator" \textcolor{black}{onsubmit="return false;"}>
  Name: <input type="text" id="name" \textcolor{black}{onchange="inputchanged();"} /> 
</form> 
<div id="output"></div>
</body></html>
\end{semiverbatim}
\end{footnotesize}
\end{frame}



\end{document}

