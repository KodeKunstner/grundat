\documentclass[dvipsnames]{beamer}

\mode<presentation>
{
  \usetheme{default}
%  \useoutertheme{infolines}
}

\setbeamertemplate{navigation symbols}{}
\usepackage{preamble}
\usepackage{coursespecific}


\renewcommand*{\Lecture}{3}

\begin{document}

\begin{frame}
  \titlepage{}
\end{frame}


\begin{frame}

\frametitle{Today's Subjects}

\begin{itemize}
\item General course information
\item Weekly exercises and the exam
\item Course content and motivation
\item Web publishing and Web applications
\item Detailed course content
\item HTML in 21 minutes
\end{itemize}
\end{frame}


\begin{frame}
\frametitle{General Course Information}

\begin{itemize}
\item \textbf{Course Goal:} To learn about advanced Web techniques for
  developing realistic, useful Web sites, involving information
  interchange with a database.

\item \textbf{Teacher:} \undervisernavne{}
\end{itemize}
\end{frame}


\begin{frame}
  \frametitle{Online Information}
  \begin{itemize}
  \item \textbf{The course home page} is \url{\kursushjemmeside}
  \item The home page for the course contains a detailed lecture plan,
    exercises, latest news, and other important course information.
  \item The lecture plan contains links to slides
  \item \textbf{Keep an eye} on the course home page throughout the
    semester.
  \item \textbf{When:} lectures \kursusugedag{} 17.00--19.00; exercises
    19.00--21.00
  \item Please ask questions during the lectures
  \end{itemize}
\end{frame}



\begin{frame}
\frametitle{You Will Need}

\begin{itemize}
\item \textbf{The course book:} ``Web Database Applications with PHP
  and MySQL'', which is available in the book store
  (Samfundslitteratur, beside the Information Desk) or can be ordered
  from Amazon (\url{http://www.amazon.dk}).
\item \textbf{Notes}, which are available from the course home page
  \begin{itemize}
 \item \emph{HTML overview}, Peter Sestoft and Ken~Friis~Larsen.
 \end{itemize}
\item \textbf{Related Literature}
  \begin{itemize}
  \item \textit{Head First HTML}, Freeman \& Freeman. O'Reilly 2006.
  \item \emph{Start på PHP}, Thomas G. Kristensen. IDG Forlag 2001. In Danish.
  \item \emph{Start på SQL}, F. D. Rolland. IDG Forlag 1998. In Danish.
%  \item \emph{HTML-codes for unusual characters and symbols}, MIT/W3C.  
\end{itemize}
\item \textbf{An account} at the ITU network.
\end{itemize}
\end{frame}

\begin{frame}
\frametitle{Weekly Exercises}

\begin{itemize}
\item To learn productively, you must try out things yourself!
\item Reading guide: Read the course book and notes prior to each lecture.
\item Exercises start next week after the lecture.
\item Problem set solutions must be handed in one week after the
  exercise, at the latest.
\item Problem set solutions are commented and marked the following week.
\item Please: \textbf{Finish the exercises} in good time before the deadlines.
\end{itemize}
\end{frame}

\begin{frame}
\frametitle{Exam, Curriculum, and Exercise Content}

\begin{itemize}
\item You are properly in good shape for the exam if you pass most of the
  problem sets. \textbf{But the problem set are not mandatory.}
\item The exam is a 4-hour written exam; all non-electronic 
  aids are allowed.
\item The problem sets can be solved individually or in groups.
\item The exercises cover:
  \begin{itemize}
  \item Static HTML
  \item Web programming with PHP
  \item Regular expressions
  \item Database programming with SQL
  \item Web sites constructed with HTML, PHP, and SQL
  \end{itemize}
\item The curriculum (pensum) is the course book (the pages specified
  in the lecture plan), the note about HTML, a research article,
  problem sets, and the lecture slides. Adjustments of the curriculum
  will appear during the course.
\end{itemize}

\end{frame}

\begin{frame}
\frametitle{Course Content and Motivation}

\begin{itemize}
\item Web publishing demands generalists:
  
  \begin{small}
    %\begin{quote}
      \textit{text processing, photography, publishing, system administration,
      database knowledge, understanding of user interfaces,
      programming experience, knowledge about design methodologies,
      \ldots.}
    %\end{quote}
  \end{small}
\item State of departure: Static Web sites (HTML)


\item How can we construct more interesting sites:
\begin{itemize}
\item Sites that does computations for you
  \begin{itemize}
  \item Tax computation
  \item Games
  \end{itemize}
\item Sites that are databases (dynamic HTML with database access)
  \begin{itemize}
  \item ITU's project base
  \item extratasty.com
  \item Amazon 
  \end{itemize}
\item Combinations \ldots
\end{itemize}


\item Popular and useful sites can be constructed with little means!
\end{itemize}
\end{frame}

\begin{frame}
\frametitle{Web Publishing}

\begin{itemize}
\item Web publishing is often associated with content similar to what
  is contained in magazines. Examples are product catalogues, news
  services, and so on.
  
\item Web applications are distinguished by solving a particular problem
  for a user. For instance: a todo-list, comment service, book
  ordering, ticket ordering, and so on.

\item Start by asking what a user is interested in!

\item What can you do to enrich a user?

\item Avoid:
\begin{itemize}
\item entry tunnel, exit tunnel
\item flashy slow graphics
\end{itemize}

\item Support multiple views on content:
\begin{itemize}
\item For instance, allow user comments...
\end{itemize}
\end{itemize}
\end{frame}



\begin{frame}
\frametitle{Web Applications}

From desktop applications to web applications (sometimes called web
services).


Examples:
\begin{itemize}
\item Calendar system
\item Email (\url{http://gmail.google.com})
\item Text processing and version control
\item Project management
%\item CourseGrader
\item Course evaluation
\item Blogs (\url{http://wordpress.org})
\item \ldots
\end{itemize}

Bugs and errors in Web-based programs can be fixed immediately!
\end{frame}



% \begin{frame}
% \frametitle{Development of static Web sites}

% \begin{columns}
%   \begin{column}{.6\textwidth}
% \begin{itemize}
% \item Draw a site map
% \item Give structure to content; add names to files. 
% \begin{itemize}
% \item Think about how secondary files (such as images) should be
%   organized.
% \end{itemize}
% \item Construct an ``only-text'' site.
% \item Hire a graphics designer!?
% \item Construct a maintenance plan. How is content updated?
% \begin{itemize}
% \item Is it possible to delegate responsibility to those who write the
%   content or is it necessary with a central ``Home Page Department?''
% \end{itemize}
% \item Construct periodical user tests---functional tests and usability
%   tests
% \end{itemize}    
%   \end{column}
%   \begin{column}{.39\textwidth}
%     \begin{center}
%       \includegraphics[width=\textwidth]{Images/14_1}
%     \end{center}
%   \end{column}
% \end{columns}


%\begin{small}
%Billede: \url{http://www.photo.net/wtr/thebook/napkin/14.1.gif}
%\end{small}

%\end{tabular}

%\end{frame}



% \begin{frame}
% \frametitle{Version Control}

% Problem:

% \begin{small}
%   \begin{itemize}
%   \item \textbf{Joe} downloads \textbf{Version A} of a document at 9
%     am from the company Web site and uses a whole day editing the
%     document.
%   \item \textbf{Tina} downloads \textbf{Version A} at 1 pm and
%     corrects a spelling mistake.
%   \item \textbf{Tina} uploads the document (\textbf{Version B}) to the
%     server at 1:10 pm.
%   \item \textbf{Joe} uploads his document (\textbf{Version C}) to the
%     server at 5 pm.
%   \end{itemize}
% \end{small}
% Tina's correction to the page is gone! \textbf{Version C} does not
% contain the correction in \textbf{Version B}!

% Solutions: ``locking'' of files with emails, Emacs, CVS, Subversion,\ldots.

% CVS: \url{http://www.cvshome.org/}\\
% Subversion: \url{http://subversion.tigris.org/}
% \end{frame}


\begin{frame}
\frametitle{Detailed Course Content}

\begin{itemize}
\item Writing and understanding HTML (HyperText Markup Language)
\item Programming in PHP (PHP Hypertext Preprocessor)
\item Communicating with a database using SQL (Structured Query Language)
\end{itemize}
\end{frame}


\begin{frame}
\frametitle{Server-Side Web Scripting---A Three-tier Architecture}

\begin{center}
  \begin{tikzpicture}[thick,>=stealth]
    \small
    \fill[blue!25] (-1.5,.6) rectangle (5.5,1.8);
    \path 
    (-4.5,1.2) node[draw,ellipse,minimum size=10mm](Client) {Client}
    (0,1.2) node[draw,rectangle,minimum size=10mm,fill=white](WS) {Web Server}
    (4,1.2) node[draw,rectangle,,minimum size=10mm,fill=white](DB) {Database Server}
    ;
    \draw[<-] (Client.north east) -- node[fill=white]{1} (WS.north west);
    \uncover<2->{\draw[->] (Client) -- node[fill=white]{2} (WS);}
    \uncover<3->{\draw[<->] (WS) -- node[fill=blue!25]{3} (DB);}
    \uncover<4->{\draw[->] (WS.south west) -- node[fill=white]{4} (Client.south east);}

  \end{tikzpicture}
\end{center}


\begin{enumerate}
\item A client obtains an HTML page with an HTML form from a Web server.
\item<2-> The client completes the HTML form, and sends it back to the
  Web server.
\item<3-> The Web server executes a program that processes data from
  the client, for instance by updating a database.
\item<4-> The Web server sends an HTML page back to the client.
\end{enumerate}

% If the HTML code is simple and does not contain too much JavaScript,
% the page can be shown by most browsers.

% The HTML code is normally presented quickly by the browser, but the
% user interface is not like a normal desktop program.
\end{frame}



\begin{frame}
  \frametitle{Why Learn To Program?}
  \begin{itemize}
  \item Better understanding of possibilities and limitations of dynamic
    Web pages
  \item Necessary condition for development of advanced Web sites
  \item Better possibilities for imagining new types of Web sites
  \item<+-> Programming can be a fun outlet for creative activities
  \item<+-> Programming is POWER
  \item<+-> Programming is LOVE 
  \end{itemize}
\end{frame}


\begin{frame}
  \frametitle{Why Learn PHP?}

  \begin{itemize}
  \item PHP is well suited for developing dynamic Web sites.
    
  \item PHP is freely available as a module for the open source Web
    server Apache.
    
  \item PHP is good at manipulating strings (and therefore good for
    Web programming).
    
  \item PHP is supported by many hosting companies.  
\end{itemize}
\end{frame}


\begin{frame}[fragile]
  \frametitle{Special (Domain-Specific) Language: SQL}

\intro{SQL:} A language for sending data back and forth between a
program and a database.
\begin{footnotesize}
\begin{verbatim}
 SELECT navn, addr FROM names ORDER BY name;
 INSERT INTO names (name, addr) VALUES ('Iris', 'Gallevej 435');
\end{verbatim}
\end{footnotesize}
Most database vendors use their own SQL dialect---but in many cases, the
differences are trivial.

\end{frame}


\begin{frame}[fragile]
\frametitle{Example SQL}

Given a table \texttt{students} with student data

\begin{tabular}{|l|l|l|}\hline
lastname & firstname & studentid\\\hline
Olesen & Peter & L2143\\
Hansen & Erika & J0007\\
Funder & Ulrik & Hg0014\\\hline
\end{tabular}

As an example, we can ask about first name and last name, sorted after
last name, first name:

\begin{footnotesize}
\begin{verbatim}
    SELECT firstname, lastname 
    FROM students 
    ORDER BY lastname, firstname;
\end{verbatim}
\end{footnotesize}

It is possible to add a new row to the table:
\begin{footnotesize}
\begin{verbatim}
    INSERT INTO students (lastname, firstname, studentid) 
    VALUES ('Fong', 'Joe', 'JF0032');
\end{verbatim}
\end{footnotesize}

It is possible to delete one or more rows:
\begin{footnotesize}
\begin{verbatim}
    DELETE FROM students 
    WHERE lastname = 'Fong';
\end{verbatim}
\end{footnotesize}
\ldots and much more
\end{frame}


\begin{frame}[fragile]
  \frametitle{Special (Domain-Specific) Language: HTML}
  
  \intro{HTML:} For describing the structure of hypertext and
  (partly) the layout.

  Example (\examplefile{hello.html}):

\begin{footnotesize}
\begin{verbatim}
 <html>
  <head><title>Hello</title>
  </head>
  <body>
    Hello Students
  </body>
 </html>
\end{verbatim}
\end{footnotesize}

Browsers do not normally require HTML to be strictly correct
HTML---but you should write correct HTML anyway!

\end{frame}





\begin{frame}
\frametitle{Why Should You Care About Learning HTML?}

\begin{itemize}
  
\item To be certain that a Web site works with all browsers
  
\item There are things that cannot be done with a WYSIWYG HTML editor
  
\item To reuse HTML code written by others
  
\item WYSIAYG = What You See Is All You Get; sometimes HTML editors
  generate unnecessary HTML
  
\item For Web programming, one writes programs that \emph{generate}
  HTML; so Web programming requires that one understands HTML

\end{itemize}
\end{frame}


% \begin{frame}
%   \frametitle{Why Use An HTML Editor?}

%   Why use a HTML editor like Microsoft Frontpage, DreamWeaver, or Nvu rather
%   than writing the HTML by hand?

%   \begin{uncoverenv}<2->
%     \begin{itemize}
%     \item It always generate correct HTML (or so it should)
%     \item One can learn HTML from it (but be aware of browser specific
%       code)
%     \item It is easier to get going
%     \end{itemize}
%   \end{uncoverenv}
% \end{frame}


\begin{frame}[fragile]
\frametitle{HTML in 21 minutes}

A legal HTML document (\examplefile{legal.html}):

\begin{small}
\begin{verbatim}
<html>
  <head>
     <title>Ken's Page</title>
  </head>
  <body>
    <h2>Ken's Marvellous Home Page</h2>
      I teach the
      <a href="http://www.itu.dk/courses/DSDS/E2006/">
       World's best course
      </a> (with some really nice looking students).
  </body>
</html>
\end{verbatim}
\end{small}

\begin{itemize}
\item Try to format HTML code nicely and consistent

\item Write the HTML code yourself

\item Later we shall construct programs that generate HTML code
\end{itemize}

\end{frame}

\begin{frame}
\frametitle{General Page Layout}

You should know about the following tags in HTML:

\begin{itemize}
\item Headings: \texttt{<h1>}\ldots \texttt{</h1>}, \ldots, \texttt{<h4>}\ldots \texttt{</h4>}
\item Rules: \texttt{<hr />}
\item Paragraphs and line breaks: \texttt{<p>}\ldots \texttt{</p>},
  \texttt{<br />}
\item Quotes: \texttt{<blockquote>}\ldots \texttt{</blockquote>}
\item Centering: \texttt{<center>}\ldots \texttt{</center>}
\item Bold text: \texttt{<b>}\ldots \texttt{</b>}
\item Italic text: \texttt{<i>}\ldots \texttt{</i>}
\item Underlined text: \texttt{<u>}\ldots \texttt{</u>}
\item Ordered lists: \texttt{<ol>}\ldots \texttt{</ol>}
\item Unordered lists: \texttt{<ul>}\ldots \texttt{</ul>}
\item List items: \texttt{<li>}\ldots \texttt{</li>}
\end{itemize}
\end{frame}

\begin{frame}
\frametitle{Hyper Links, Tables, and Images}
\begin{itemize}
\item Hyper links: \texttt{<a href="link">name</a>}
\item Local \emph{named} hyper links: \texttt{<a
    name="thename">}\ldots\texttt{</a>}
\item References to a name: \texttt{<a href="index.html\#thename">The Name</a>}
\item Mail-to links: \texttt{<a href="mailto:kfl@itu.dk">kfl@itu.dk</a>}
\item Tables: \texttt{<table>}\ldots \texttt{</table>}, \texttt{<tr>}\ldots
  \texttt{</tr>}, \texttt{<th>}\ldots \texttt{</th>} and \texttt{<td>}\ldots
  \texttt{</td>}
\item Images: \texttt{<img src="pluto.jpg" />}
\item Colors: use Cascading Style Sheets (CSS)
\end{itemize}
\end{frame}

\begin{frame}[fragile]
\frametitle{HTML forms}
  
A form can contain text areas (\verb+<textarea>+),
input fields (\verb+<input>+), and menus (\verb+<select>+)

Example: \examplefile{formular.html} (and \examplefile{formular2.html})

\begin{verbatim}
   <form action="mailto:kfl@itu.dk" 
         method="post" enctype="text/plain">
     ... more code ...
     <input type="reset" value="Start Over!" />
     <input type="submit" value="Send Form" />
   </form>
\end{verbatim}

Attributes to the \verb+<form>+ tag:

\begin{itemize}
\item \verb+action="mailto:kfl@itu.dk"+:\\ the completed form is sent
  by email
  
\item \verb+method="post"+ should be used when the completed form is
  sent by email
  
\item \verb+enctype="text/plain"+ sends the form as ordinary text
  (not encoded)
\end{itemize}

A click on \verb+<input type="submit" value="Send Form" />+ sends the
completed form.

A click on \verb+<input type="reset" value="Start Over" />+ resets all
fields, text areas, and menues.
\end{frame}

\begin{frame}[fragile]
\frametitle{Text Areas --- \texttt{<textarea>}}
  
  In a text area (\verb+<textarea>+) any text can be written:

\begin{verbatim}
   <textarea name="comments" rows="5" cols="80">
     Write your comments here!
   </textarea>
\end{verbatim}

The attributes:

\begin{itemize}
\item \verb+name+ specifies a name for the field

\item \verb+rows+ specifies the number of lines (rows) in the text area
  
\item \verb+cols+ specifies the number of characters (columns) in each
  line in the text area
\end{itemize}


\end{frame}

\begin{frame}[fragile]
\frametitle{Input Fields --- \texttt{<input>} Text and Passwords}
  
  The attribute \verb+type+ determines the type of the input field:

\begin{itemize}
\item Single-line text areas:\\
\verb+<input type="text" name="lastname" />+


\item Passwords, which are not to be displayed by the browser:\\
  \verb+<input type="password" name="studentid" />+
  
\end{itemize}
\end{frame}
 

\begin{frame}[fragile]
\frametitle{Input Fields --- \texttt{<input>} Check Boxes and Buttons }
  
  More \texttt{input} types:

\begin{itemize}
\item Check boxes, possibly more
  than one item can be checked at a time:
  \begin{small}
\begin{verbatim}
<input type="checkbox" name="paper" value="nytimes" /> 
<input type="checkbox" name="paper" value="wpost" />
<input type="checkbox" name="paper" value="guardian" />
\end{verbatim}  
  \end{small}

\item Radio buttons, only one item can be checked at a time:
  \begin{footnotesize}
\begin{verbatim}
<input type="radio" name="sex" value="male" />
<input type="radio" name="sex" value="female" checked="yes" />
\end{verbatim}
  \end{footnotesize}
\item Button for resetting a form:
  \begin{small}
\begin{verbatim}
<input type="reset" ... />
\end{verbatim}
  \end{small}
\item Button for sending a completed form:
  \begin{small}
\begin{verbatim}
<input type="submit" ... />
\end{verbatim}
  \end{small}
\end{itemize}
\end{frame}





\begin{frame}[fragile]
\frametitle{Menu Choices, \texttt{<select>}}

\begin{itemize}
\item A drop-down menu, which allows the user so choose between a
  series of items (+<option>+):
  \begin{small}
\begin{verbatim}
<select name="order">
  <option value="9">Pepperoni Pizza</option>
  <option value="70">Pizza Bambino</option>
  <option value="47">Chicken Dipper (9 pieces)</option>
</select>
\end{verbatim}
  \end{small}
\item The attribute \verb+multiple+ in the \verb+<select>+ tag allows
  the user to choose more than one item.
\end{itemize}


\textbf{Complete form}

\begin{itemize}
\item Bare, with no layout structure: \examplefile{formular.html}
\item An example course evaluation with a nicer lay-out (using
  \verb+<table>+): \examplefile{formular2.html}
\end{itemize}

\end{frame}

\begin{frame}[fragile]
\frametitle{Completed Questionnaire, When Sent By Email}

\textbf{With} \verb+enctype="text/plain"+:

\begin{small}
\begin{verbatim}
 lastname=Olsen
 studentid=da4567
 sex=male
 paper=guardian
 comments=This is not a nice questionnaire.
 order=9
\end{verbatim}
\end{small}


\textbf{Without} \verb+enctype="text/plain"+:

\begin{small}
\begin{verbatim}
 lastname=Olsen&studentid=da4567&sex=male&paper=guardian
 &comments=This+is+not+a+nice+questionnaire.&order=9
\end{verbatim}
\end{small}
\end{frame}

\begin{frame}[fragile]
\frametitle{Find five bugs}

In an HTML document without consistent indentation:
\begin{footnotesize}
\begin{verbatim}
  <html><head><title>Hello</title></head>Hello<center>great page<p>
  this is in <b>bold<i> and italics</html>
\end{verbatim}
\end{footnotesize}
\begin{uncoverenv}<2->
  In an HTML document with consistent indentation:
  \begin{footnotesize}
\begin{verbatim}
  <html>
    <head>
      <title>Hello</title>
    </head>
    Hello
    <center>
      great page
      <p>
      this is in <b>bold<i> and italics
  </html>
\end{verbatim}
  \end{footnotesize}
  Firefox shows this document without errors! Or does it?
\end{uncoverenv}
\end{frame}


\begin{frame}
  \frametitle{Cascading Style Sheets (CSS)}
  
  \begin{itemize}
  \item We don't want to mix the logical \intro{structure} with the
    \intro{presentation} of our webpages
    \begin{itemize}
    \item Use HTML for the \emph{logical} structure
    \item Use Cascading Style Sheets (CSS) for the presentation
    \end{itemize}
  \item Advantages of CSS
    \begin{itemize}
    \item We get a consistant look of our pages
    \item We only have to change the layout in one place
    \item We can tailor the presentation to different devices (medias)
      by using different style sheets
    \end{itemize}
\end{itemize}
  
\end{frame}

\begin{frame}[fragile=singleslide]
  \frametitle{CSS Structure}
  
  \begin{itemize}
  \item A style sheet consists of a set of \intro{rules}.  Each rule
    consists of a \intro{selector} and a set of \intro{styles}. A
    style set a \intro{property} to a given \intro{value}.

  \item Example of a CSS rule with two styles:
    \begin{small}
\begin{verbatim}
  p { background-color: red;
      border: 3px solid gray;
  }
\end{verbatim}
        This rule specifies that all \texttt{<p>} elements should have
        a red background and a boder that is three pixels think,
        solid, and gray. 
      \end{small}

    \item We can style multiple elements by using a comma in the selector:
      \begin{small}
\begin{verbatim}
  h1, h2 { border-bottom: thin solid blue; }
\end{verbatim}
        This rules specifies that both \texttt{<h1>} and \texttt{<h2>}
        elements should have a thin, solid, blue border at the bottom.
      \end{small}
  \end{itemize}

\end{frame}


\begin{frame}[fragile=singleslide]
  \frametitle{CSS Continued}

  \begin{itemize}
  \item If we want \texttt{<a>} links in \texttt{<h2>} headings to be
    orange, but \emph{only} those links, we need a \intro{decendant
    selector}:
    \begin{small}
\begin{verbatim}
  h2 a { color: orange; }
\end{verbatim}
    \end{small}

  \item We can put our CSS in a \texttt{<style>} element in the
    \texttt{<head>} element of our document:
    \begin{footnotesize}\color{gray}
\begin{semiverbatim}
  <html>
  <head>...
    {\color{black}<style type="text/css">
      ...
    </style>}
  </head>
\end{semiverbatim}
    \end{footnotesize}
  \item Or we can put it in an external file, say,
    \texttt{styling.css}.  And then include it in the \texttt{<head>}
    instead:
    \begin{footnotesize}\color{gray}
\begin{semiverbatim}
  <html>
  <head>...
    {\color{black}<link rel="stylesheet" type="text/css" href="styling.css">}
  </head>
\end{semiverbatim}
    \end{footnotesize}
    Which is better, because then we can reuse the CSS.
  \end{itemize}
\end{frame}



\begin{frame}[fragile]
  \frametitle{Giving Our Form Some Style}
  
  \begin{itemize}
  \item Say, we want our form to have a blue background, white text, a
    grey dotted border, and nice wide margins.

  \item Thus, we add the following \intro{\texttt{style} element} to the
    \texttt{head} of our HTML form:
    \begin{footnotesize}\color{gray}
\begin{semiverbatim}
<html>
<head><title>Form Example</title>
{\color{black}<style type="text/css">
  body \{ background-color: blue;
         color: white;
         margin-left: 20%;
         margin-right: 20%;
         border: 3px dotted gray;
         padding: 10px 10px 10px 10px;
  \}
</style>}
</head>
<body>
<h1>Questionnaire</h1>
...
</body>
\end{semiverbatim}
    \end{footnotesize}
  \end{itemize}
\end{frame}


\begin{frame}[fragile]
  \frametitle{Style An Element With Some Class}

  \begin{itemize}
  \item What if we want one of our elements to stand out?
  \item<2-> Add a \intro{\texttt{class}} attribute:
    \begin{small}\color{gray}
      \begin{semiverbatim}
<p \textcolor{black}{class="thesexquestion"}>
Enter sex: Male <input type="radio" ...
      \end{semiverbatim}
    \end{small}
  \item<3-> Add a rule for the class in the \texttt{style} element:
    \begin{small}\color{gray}
\begin{semiverbatim}
<style type="text/css">
  body \{ ... \}
  \textcolor{black}{.thesexquestion \{ 
     background-color: rgb(50%, 0%, 0%);
  \}}
</style>
\end{semiverbatim}
    \end{small}
    (Selectors for classes starts with a '.')
  \item<4->  Use \intro{\texttt{<div>}} and \intro{\texttt{<span>}}
    elements to give extra structure to your page if needed.
    \begin{itemize}
    \item Use \texttt{<div>} for big blocks spanning multible
      paragraphs and/or headings.
    \item Use \texttt{<span>} for highlighting some inline texts, like
      a word or a sentence.
    \end{itemize}
  \end{itemize}



\end{frame}





% \begin{frame}
% \frametitle{Some HTML Rules}

% \begin{itemize}
% \item Avoid large tables for page layout (difficult and slow for
%   browsers to render).
% \item Be careful with redefinitions of link colors.%\item så lidt formattering som muligt
% \item Make it possible for the user to construct black-and-white
%   printouts (e.g., some web browsers prints white text on a blue
%   background as white text on a white background!)
% \item Think about the user's screen as a limited resource. 
% \item Avoid the use of frames (frames do not work well with printing,
%   bookmarks, or email).
% \item Remember to give your HTML pages a title (used for bookmarks,
%   e.g.)
% \end{itemize}

% A goal could be that both Firefox and Internet Explorer
% can show your page.

% What about WebTV, Palm Pilots, mobile phones, \ldots

% \end{frame}

% \begin{frame}
% \frametitle{More Web Design Rules}

% \begin{itemize}
% \item Let information be the user interface: By a click on an
%   informative word, the user is sent to a page with more in-depth
%   information about the concept (avoid ``here-links'').
% \item Provide users with a wide, flat overview of the Web site,
%   instead of a sequential or deep tree structure.
% \item Organize your Web site according to user interests, instead of
%   after the internal organization.
% \item Why use icons for navigations when words are more intuitive and
%   take up less space?
% \item Construct long documents instead of wide documents.
% \item Remember to put a link back to the index page on every page ---
%   for instance, to support users coming from search engines.
% %\item Udstyr dit web-site med en fuldtekst søgemaskine---og sørg
% %  for, at den holder statistik med de ord, som den ikke kan finde svar
% %  på.
% \item Add contact information to your pages.
% %\item A vertical menu on a page takes less space than a horizontal
% %  menu
% \end{itemize}
% \end{frame}



% \begin{frame}
% \frametitle{Use of Applets for Updating a Database}

% \begin{center}
%   \begin{tikzpicture}[thick,>=stealth]
%     \small
%     \fill[blue!25] (-1.5,.6) rectangle (5.5,1.8);
%     \path 
%     (-4.5,1.2) node[draw,ellipse,minimum size=10mm](Client) {Client}
%     (0,1.2) node[draw,rectangle,minimum size=10mm,fill=white](WS) {Web Server}
%     (4,1.2) node[draw,rectangle,,minimum size=10mm,fill=white](DB) {Database Server}
%     ;
%     \draw[<-] (Client) -- node[fill=white]{1} (WS);
%     \uncover<2->{\draw[<->] (Client) .. controls +(-40:20mm) and +(-120:20mm)
%       .. 
%      node[fill=white]{2} (DB);}
%   \end{tikzpicture}
% \end{center}
% % \begin{center}
% % \includegraphics[width=10cm]{Images/applet}
% % \end{center}

% \begin{enumerate}
% \item A client obtains an HTML page and an associated applet from a Web server.
% \item<2-> The client starts executing the applet as a program; requests
%   and updates are send directly to the database.
% \end{enumerate}

% \begin{uncoverenv}<3->
%   Comments:
%   \begin{itemize}
%   \item Applets work well only on newer browsers and applets take a
%     long time to download.

%   \item On the other hand, they can be more interactive.
  
%   \item We do not discuss applets and other similar technologies
%     (e.g., Flash) in this course.

% \end{itemize}
%   \end{uncoverenv}

% \end{frame}

% \begin{frame}[fragile]
% \frametitle{What is a Programming Language?}

% \begin{itemize}
% \item A programming language is a notation for instructing a computer
%   what to do (i.e., a notation for programs).
  
% \item There are many different programming languages, including: PHP,
%   Perl, Tcl, Standard ML, Java, C\#, C, C++, Erlang, Haskell, Pascal,
%   Forth, Fortran, Prolog, Ada, Python, Common~Lisp, Scheme, \ldots
% \end{itemize}

% \end{frame}

% \begin{frame}[fragile]
%   \frametitle{Programmers needs to be careful to get the syntax right!}

% \begin{itemize}
% \item A computer is normally very pedantic. 

% \item This issue is particularly frustrating to novice programmers.
% \end{itemize}

% Example: A Correct PHP command:

% \begin{verbatim}
%    echo "I'm alive!";
% \end{verbatim}

% A very wrong PHP command:

% \begin{verbatim}
%    Echo "I'm alive!;
% \end{verbatim}

% Experienced programmers are good at locating errors.

% \end{frame}


\begin{frame}[fragile]
\frametitle{Introduction to Problem Set 1}

\begin{itemize}
\item HTML home page
\item HTML course overview with tables and links
\item Use of HTML forms
\end{itemize}

\textbf{Material}
\begin{itemize}
\item Problem Set: \url{\problemset{1}}
\item Sestoft and Larsen's HTML overview:
   \url{\kursushjemmeside{html.html}}
  \end{itemize}
\end{frame}

\end{document}

