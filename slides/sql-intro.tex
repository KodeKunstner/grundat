\documentclass[dvipsnames,handout]{beamer}

\usepackage{preamble}
\usepackage{coursespecific}
%\renewcommand*{\IncludeAnswers}{1}

\renewcommand*{\Lecture}{1}
\subtitle{Introduction To \sql{}}

%\newcommand*{\sql}{SQL}

\begin{document}

\begin{frame}
  \titlepage{}
\end{frame}

\begin{frame}

\frametitle{Today's Subjects}

\begin{itemize}
\item General course information
\item Course content and motivation
\item Computers are good at:
  \begin{itemize}
  \item calculations
  \item storing and retrieving data
  \item (showing pretty pictures)
  \end{itemize}

\item Databases are great for storing and retrieving data

\end{itemize}
\end{frame}


\begin{frame}
\frametitle{General Course Information}

\begin{itemize}
\item \textbf{Course Goal:} A tour of important topics in Computer
  Science.  Enabling you to make \emph{informed} choices about data
  structures and algorithms, and to give you enough knowledge and
  understand to seek more information on you own.

\item \textbf{Teachers:} \undervisernavne{}
\end{itemize}
\end{frame}


\begin{frame}
\frametitle{You Will Need}

\begin{itemize}
\item \textbf{The course book:} ``How to think like a Computer
  Scientist - Learning with Python'', by Allen Downey, Jeffrey Elkner,
  and Chris Meyers. Green Tea Press.
\item \textbf{Notes and slides}, which will be made available from the
  course homepage.
\item \textbf{Related Literature}
  \begin{itemize}
  \item \textit{Head First HTML}, Freeman \& Freeman. O'Reilly 2006.
  \item \emph{Start på SQL}, F. D. Rolland. IDG Forlag 1998. In Danish.
%  \item \emph{HTML-codes for unusual characters and symbols}, MIT/W3C.  
\end{itemize}
\item \textbf{An account} at the KUA network.
\end{itemize}
\end{frame}

% \begin{frame}
% \frametitle{Weekly Exercises}

% \begin{itemize}
% \item To learn productively, you must try out things yourself!
% \item Reading guide: Read the course book and notes prior to each lecture.
% \item Exercises start this week after the lecture.
% \item Problem set solutions must be handed in one week after the
%   exercise, at the latest.
% \item Problem set solutions are commented and marked the following week.
% \item Please: \textbf{Finish the exercises} in good time before the deadlines.
% \end{itemize}
% \end{frame}


\begin{frame} \frametitle{Today's Problem}
  \begin{Large}
    How should we store data on a computer?
  \end{Large}

  \begin{itemize}
  \item In text files?
  \item As spreadsheets?
  \item As XML files?
  \end{itemize}
\end{frame}





\begin{frame}[fragile]
\frametitle{The File System As A Database} 

The pros and cons of using the file system as a database are:

\textbf{Pros:}

\begin{itemize}
\item No need for extra software
\item The solution is easy to understand---data is stored as text in files:
\begin{footnotesize}
\begin{verbatim}
    kflarsen@diku.dk         Ken Friis Larsen
    simonsen@diku.dk         Jakob Grue Simonsen
    gates@microsoft.com      Bill Gates
\end{verbatim}
\end{footnotesize}
\end{itemize}


\textbf<+->{Cons:}

\begin{itemize}
\item<+-> Immediate performance problems (linear seek)
\item<+-> No abstraction from the data representation
\item<+-> Problems handling concurrent users
\item<+-> No standard for integration with other systems
\item<+-> Problems handling errors
\end{itemize}

\uncover<.->{You end up constructing what companies have been good at since the
60'es: (R)DBMSs.
}
\end{frame}


\begin{frame} \frametitle{When To Use A Database}
  \begin{itemize}
  \item  When you have lots of data
  \item When you need to ask complex questions:
    \begin{itemize}
    \item ``Find all the teachers that are teaching courses with less
      than 10 students, and where more that 30\% does not pass the
      course or the percentage of female students are greater than 65\%''
    \item Relational database optimised for answering such questions
    \end{itemize}
  \item Many people try to use spreadsheets as simple databases
    \begin{itemize}
    \item Works for small data sets like course grades
    \item But they don't scale up, and search capabilities are
      primitive
    \end{itemize}
  \item Increasingly common to store images, video clips, and other
    data in the database.
  \end{itemize}
\end{frame}





\begin{frame}
\frametitle{Database Terminology}
  
\begin{itemize}
\item A \intro{relational database} consists of a collection of named
  \intro{tables} (sometimes called \intro{relations}).

\item A \intro{table} consists of two parts:

  \begin{itemize}
  \item A \intro{schema} (= table description):
  
    \begin{itemize}
    \item The schema determines the shape of the table, and is rarely
      changed.

    \item The schema indicates which \intro{columns} (= fields =
      attributes) the table consists of, and which \intro{type} each
      field has.
    \end{itemize}
  \item A collection of \intro{records} (= table rows):
  
    \begin{itemize}
    \item The collection of records constitute the \intro{contents} of
      the table; it can change over time.
  
    \item Each record contains a set of \intro{values} for the fields of
      the record.

    \item The order of the records in the table is insignificant.
    \end{itemize}
  \end{itemize}
\end{itemize}
\end{frame}

\begin{frame}

\frametitle{Example: Course Data}
%\renewcommand{\arraystretch}{1}
\begin{small}
\begin{tabular}{@{}l@{}l}
  \texttt{Student:} &
  \begin{tabular}{@{}|l|l|l|}\hline
    sname & sid & address\\\hline
    Ole Olesen & L1234 & Hjemmevej 7\\
    Rikke Richardsen & L2345 & Udvej 9\\
    Søren Sørensen & J0007 & I dybet 13\\
    Ulrikka Funder & B7117 & Skovvej 11\\\hline
  \end{tabular}\\
  \\

  \texttt{Course:}
  &
  \begin{tabular}{@{}|l|l|l|}\hline
    cid & cname & teacher\\\hline
    DAT2 & Data \& Algorithms & Ken Friis Larsen \\
    DAT1 & Introduction to programming & Jakob Grue Simonsen \\ \hline
  \end{tabular}\\
  \\

  \texttt{Signup:} &
  \begin{tabular}{@{}|l|l|}\hline
    sid & cid\\\hline
    L1234 & DAT2\\
    L2345 & DAT2\\
    J0007 & DAT2\\
    J0007 & DAT1\\
    L2345 & DAT1\\\hline
  \end{tabular}
\end{tabular}
\end{small}
\end{frame}



\begin{frame}[squeeze]
\frametitle{The Secret Weapon Of Relational Databases: Queries}

    \begin{itemize}
    \item Base data in a relational database lies in \intro{tables}.

    \item Data extraction and manipulation is done using
      \intro{queries}.

    \item You only need to consider the \emph{tables} in the
      beginning when designing.

      E.g.: Student, Course, Signup, Examination, Teacher, Room,
      \ldots

    \item You can always add new kinds of queries as they are needed.
     
      E.g.: Compute the pass rate for all courses, grouped by study
      lines and immatriculation year.

    \end{itemize}

\end{frame}



\begin{frame}
\frametitle{Example: Address Book, First Attempt}

\begin{itemize}
\item We want a small database with contact information: email and
  work phone number

\item<+-> For our first attempt we might try to keep all the information
  in one table:

  \texttt{Persons:}\\
  \begin{tabular}{|l|l|l|}
    \hline
    email & name & phone\\\hline
    kflarsen@diku.dk & Ken Friis Larsen & 35 32 14 24 \\
    simonsen@diku.dk & Jakob Grue Simonsen & 35 32 14 39 \\
    \hline
  \end{tabular}

\item<+-> But what if we also want to keep track of a direct phone number
  \begin{itemize}
  \item<+-> and mobile number,
  \item<+-> and home number,
  \item<+-> and private mobile number, and \ldots
  \end{itemize}
\item<+-> A better design is to keep persons in one table and phone
  numbers in an other table.

\end{itemize}
\end{frame}


\begin{frame}
\frametitle{Example: Address Book, Second Attempt}

%\renewcommand{\arraystretch}{0.9}

\begin{itemize}
\item An excerpt of the table \texttt{Persons}:\\
  \begin{small}
    \begin{tabular}{|l|l|}\hline
      email & name \\ \hline
      kflarsen@diku.dk & Ken Friis Larsen \\
      simonsen@diku.dk & Jakob Grue Simonsen \\ \hline
    \end{tabular}
  \end{small}


\item An excerpt of the table \texttt{Phone\_numbers}:\\
  \begin{small}
    \begin{tabular}{|l|l|}\hline
      email & phone\\ \hline
      simonsen@diku.dk & 35 32 14 39 \\
      kflarsen@diku.dk & 35 32 14 24 \\
      kflarsen@diku.dk & 51 95 65 45 \\ \hline
    \end{tabular}
  \end{small}

\item Each person, uniquely identified by an email address, can be
  assigned more than one phone number.

  \begin{tikzpicture}[thick]\footnotesize
    \path
    (0,0) node[draw] (P) {Persons}
    +(-1,0) node [draw,ellipse,anchor=east] (name) {name}
    +(0,-.8) node [draw,ellipse] (email) {email}
    (4,0) node[draw] (PN) {Phone\_numbers}
    +(2.5,0) node [draw,ellipse] (phone) {phone}
    +(0,-.8) node [draw,ellipse] (phemail) {email}
    ;
    \draw (P) -- node[above,very near end] {M} (PN);
    \draw (name) -- (P);
    \draw (email) -- (P);
    \draw (phone) -- (PN);
    \draw (phemail) -- (PN);
  \end{tikzpicture}



%   \begin{tikzpicture}\footnotesize
%     \path[inner sep=0pt] 
%      (0,0) node (PN) {%
%       \begin{tabular}{@{}|l|@{}}
%         \hline
%         {Phone\_numbers} \\\hline
%         phone\\
%         email\\\hline
%       \end{tabular}}
%      (4,0) node (P) {%
%       \begin{tabular}{@{}|l|@{}}
%         \hline
%         {Persons} \\\hline
%         email\\
%         name\\\hline
%       \end{tabular}};
%      \path (PN.east) +(.25,0) node[inner sep=0pt] (Helper){};
%      \draw (P) -- (PN);
%      \draw (Helper) -- (PN.5);
%      \draw (Helper) -- (PN.-5);
%    \end{tikzpicture}

    %\includegraphics[width=7cm]{Images/mailDB}
    % \begin{tikzpicture}[x=100mm,y=100mm]
    %   \input{Images/mailDB.tikz}
    % \end{tikzpicture}

%\item The fork indicates a \emph{one to many} relation.
\end{itemize}
\end{frame}


\begin{frame}[squeeze]
\frametitle{The \sql{} Query Language}

  \begin{itemize}
  \item Relational database queries are written in \sql{}
    (Structured Query Language).

  \item \sql{} was invented by IBM together with relational
    databases around 1970.

  \item \sql{} is used nowadays in most serious database systems
    (Oracle, DB2, Postgres, SQLite, MySQL, MS SQLServer, \ldots).
    Unfortunately, each system has its own vendor-specific deviations
    from the \sql{} standard.

  \item In MS Access one often constructs queries in a query grid, but
    this is converted into \sql{} requests.
  \end{itemize}

\end{frame}

\begin{frame}
  \frametitle{The Database System MySQL And SQLite}

  \begin{itemize}
  \item This course uses the database systems SQLite and (maybe)
    MySQL.  They are robust and efficient free systems available at
    \url{www.sqlite.org} and \url{www.mysql.com}.
    
  \item Both are extremely widespread and are used by private
    organisations as well as industry.
    
  \end{itemize}
\end{frame}



\begin{frame}
\frametitle{Overview of the most important \sql{}-commands}

\begin{block}{Data Definition Language (data structure design)}
  \begin{itemize}
  \item \intro{\texttt{CREATE TABLE}} creates a new table
  \item \intro{\texttt{CREATE [UNIQUE] INDEX}} adds an (unique) index to a table
  \item \intro{\texttt{DROP TABLE}} deletes a table
  \item \intro{\texttt{ALTER TABLE}} adds or changes table columns
  \end{itemize}
\end{block}

\begin{block}{Data Manipulation Language (data manipulation)}
  \begin{itemize}
  \item \intro{\texttt{SELECT}} extracts, merges and computes data
  \item \intro{\texttt{INSERT INTO \ldots{} VALUES (\ldots{})}} inserts an
    individual record into a table
  \item \intro{\texttt{INSERT INTO \ldots{} SELECT \ldots{}}} inserts records
    from a query into a table
  \item \intro{\texttt{DELETE FROM}} removes records from a table
  \item \intro{\texttt{UPDATE} \ldots{} SET} changes records already in a table
  \end{itemize}
\end{block}
\end{frame}


\begin{frame}
  \frametitle{\sql{} And Case Sensitivity}
  
  \begin{itemize}
  \item \sql{} usually does not distinguish between upper- and
    lowercase letters.


  \item However, MySQL is case sensitive in table names (a
    \texttt{Students} table is different from a \texttt{students}
    table). SQLite is \emph{not} case sensitive.


  \item In this course I'll mostly write \sql{} commands in UPPERCASE
    letters, Capitalise table names, and write column names in
    lowercase letters.

  \end{itemize}

\end{frame}


\begin{frame}[fragile=singleslide]
\frametitle{Data Definition Language} 

\begin{itemize}

\item Example of \texttt{CREATE TABLE} commands:
\begin{small}
\begin{verbatim}
  CREATE TABLE Persons (
     email  varchar(100) NOT NULL,
     name   varchar(100) NOT NULL
  );
  CREATE TABLE Phone_numbers (
     email  varchar(100) NOT NULL,
     phone  varchar(20) NOT NULL
  );
\end{verbatim}
\end{small}
\item \texttt{NOT NULL} means that whenever data is entered into the
  table, this field must not be empty.
  
\item \texttt{varchar(100)} means that in every record, this field can
  contain at most 100 characters.

\item The intention is that the \texttt{email} column in the
  \texttt{Phone\_numbers} table \emph{refers} to a corresponding
  column in the \texttt{Persons} table: There \emph{ought to}
  exist a row in \texttt{Persons} where the \texttt{email} field
  contains the same value as \texttt{email}.
\end{itemize}
\end{frame}


\begin{frame}
\frametitle{Primary Keys}

\begin{itemize}  
\item A \intro{primary key} is a combination of fields that \intro{uniquely
  determine a row of the table}. Typically, the primary key is also the
  main vehicle for looking up rows of the table.

\item Example: \texttt{email} is the primary key of \texttt{Persons}
\end{itemize}
\end{frame}




\begin{frame}[fragile=singleslide]
  \frametitle{Declaring Primary Keys}
  \begin{itemize}
  \item You can declare the primary key when you create the table:
\begin{footnotesize}
\begin{semiverbatim}
  CREATE TABLE Persons (
     email  varchar(100) \textbf{PRIMARY KEY},
     name   varchar(100) NOT NULL
  );
\end{semiverbatim}
\end{footnotesize}

\item The following holds for primary keys:
\begin{itemize}
\item All columns in the primary key are automatically \texttt{not null}.
\item A \texttt{UNIQUE INDEX} is automatically created on the
  combination of columns of the primary key.
\end{itemize}

  \end{itemize}
\end{frame}


% \begin{frame}[fragile=singleslide]

% \frametitle{Generating Unique ID Numbers In MySQL}


% \begin{itemize}
% \item In MySQL you can use \texttt{AUTO\_INCREMENT} to generate fresh
%   ID numbers automatically when inserting new rows into a table:
%   \begin{small}
% \begin{verbatim}
%   CREATE TABLE Users (id int AUTO_INCREMENT PRIMARY KEY,
%                       name varchar(100) NOT NULL);
%   INSERT INTO Users (name) VALUES ('Ken Friis Larsen');
%   INSERT INTO Users (name) VALUES ('Jakob Grue Simonsen');
% \end{verbatim}
%   \end{small}

%   (Other database systems provide similar functionality)

% \end{itemize}
% \end{frame}



% \begin{frame}[fragile=singleslide]
% \frametitle{Ensuring Referential Constraints} 

% \begin{itemize}
% \item The referential constraint between \texttt{Phone\_numbers.email}
%   and \texttt{Persons.email} can be ensured by MySQL, but then
%   the tables must be created using the \texttt{InnoDB} engine:
% \begin{small}
% \begin{semiverbatim}
%   CREATE TABLE Persons (
%      email  varchar(100) NOT NULL,
%      name   varchar(100) NOT NULL,
%      \textbf{INDEX(email)}
%   ) \textbf{TYPE=InnoDB};
%   CREATE TABLE Phone_numbers (
%      email  varchar(100) NOT NULL,
%      phone  varchar(20) NOT NULL,
%      \textbf{FOREIGN KEY (email) REFERENCES Persons(email)}
%   ) \textbf{TYPE=InnoDB};
% \end{semiverbatim}
% \end{small}

% \item The foreign key and the referenced key must be of the same type
%   (except that string lengths need not be identical)

% \item \texttt{NULL} record values are not checked for referential
%   integrity
% \end{itemize}

% \end{frame}

\begin{frame}
\frametitle{Overview Of The Most Important \sql{} Types}

\begin{center}
\begin{tabular}{@{}lll@{}}\hline\hline
\textbf{Type}        & \textbf{Usage} & \textbf{Example value}\\\hline
\texttt{int}         & integers & \texttt{117}\\
\texttt{double}      & reals & \texttt{3.1415}\\
\texttt{varchar(80)} & text (max 80 characters) & \texttt{'Ole
  Jensen'}\\
\texttt{text}        & text (unlimited, almost) & \texttt{'Ole Jensen'}\\
\texttt{date}        & date & \texttt{'2007-09-15'}\\
\texttt{time}        & time of day & \texttt{'22:59:17'}\\
\texttt{datetime}    & date and time of day & \texttt{2002-03-10 22:59:17}\\\hline\hline
\end{tabular}
\end{center}

Other types exist, but we will not be using them here.

The types are used in the \texttt{CREATE TABLE} and \texttt{ALTER
  TABLE} commands.
 
\end{frame}

\begin{frame}[fragile=singleslide]
\frametitle{Data Manipulation Language---\texttt{INSERT}} 

Examples of \texttt{INSERT} commands:

\begin{small}
\begin{verbatim}
  INSERT INTO Persons (name, email)
  VALUES ('Ken Friis Larsen', 'kflarsen@diku.dk');

  INSERT INTO Persons (name, email)
  VALUES ('Jakob Grue Simonsen', 'simonsen@diku.dk');

  INSERT INTO Phone_numbers (email, phone)
  VALUES ('kflarsen@diku.dk', '35 32 14 24');

  INSERT INTO Phone_numbers (email, phone)
  VALUES ('simonsen@diku.dk', '35 32 14 39');

  INSERT INTO Phone_numbers (email, phone)
  VALUES ('kflarsen@diku.dk', '51 94 65 45');
\end{verbatim}
\end{small}

\end{frame}

\begin{frame}[fragile=singleslide]
\frametitle{Data Manipulation Language---\texttt{SELECT}} 

The  \texttt{SELECT} command is used to extract data from a database

Examples:

\begin{small}
\begin{verbatim}
  SELECT * FROM Persons;
\end{verbatim}
\end{small}


\begin{small}
\begin{verbatim}
  SELECT name FROM Persons;
\end{verbatim}
\end{small}

A \texttt{WHERE} clause is used to extract only some of the rows:

\begin{small}
\begin{verbatim}
  SELECT name 
  FROM Persons 
  WHERE email = 'kflarsen@diku.dk;
\end{verbatim}
\end{small}

Sorting---\texttt{ORDER BY}:
\begin{small}
\begin{verbatim}
  SELECT * FROM Persons ORDER BY name;
\end{verbatim}
\end{small}

Reverse sorting---\texttt{ORDER BY \ldots{} DESC}:
\begin{small}
\begin{verbatim}
  SELECT * FROM Persons ORDER BY name DESC;
\end{verbatim}
\end{small}

\end{frame}

\begin{frame}[fragile=singleslide]
\frametitle{Data Manipulation Language---\texttt{JOINS}}

Join:

\begin{small}
\begin{verbatim}
  SELECT * FROM Persons, Phone_numbers;
\end{verbatim}
\end{small}

A better join:

\begin{small}
\begin{verbatim}
  SELECT * FROM Persons, Phone_numbers
  WHERE Persons.email = Phone_numbers.email;
\end{verbatim}
\end{small}

Or even better:

\begin{small}
\begin{verbatim}
  SELECT name, Persons.email, phone  
  FROM Persons, Phone_numbers
  WHERE Persons.email = Phone_numbers.email;
\end{verbatim}
\end{small}

\end{frame}

\begin{frame}[fragile=singleslide]
\frametitle{Data Manipulation Language---\texttt{DELETE} and \texttt{UPDATE}}

\begin{itemize}
\item The command \texttt{DELETE} is used to delete rows from a table:

\begin{small}
\begin{verbatim}
  DELETE FROM Persons WHERE email = 'simonsen@diku.dk'; 
\end{verbatim}
\end{small}
\textbf{Beware:} The database can enter an inconsistent state when
doing this.

\item For instance, it is better to do it correctly, by first deleting from
  \texttt{Phone\_numbers}:
\begin{small}
\begin{verbatim}
  DELETE FROM Phone_numbers WHERE email = 'simonsen@diku.dk';
  DELETE FROM Persons WHERE email = 'simonsen@diku.dk'; 
\end{verbatim}
\end{small}

\item The command \texttt{UPDATE} is used to change columns in a table:
\begin{small}
\begin{verbatim}
  DELETE FROM Phone_numbers WHERE email = 'kflarsen@diku.dk';

  UPDATE Persons
  SET email = 'ken@friislarsen.net'
  WHERE email = 'kflarsen@diku.dk';
\end{verbatim}
\end{small}

\item Be careful with the  \texttt{WHERE}-constraints.

\end{itemize}
\end{frame}

\begin{frame}
\frametitle{Constructing Database Supported Web Sites} 

\begin{enumerate}[Step 1:]
  \item Make a data model
    \begin{itemize}
    \item Which information should be stored and how should it be
      represented?
    \item Activity: write \texttt{CREATE TABLE} queries.
    \end{itemize}


  \item Develop data transactions
    \begin{itemize}
    \item How do we insert data into the database?
    \item How do we extract data from the database?
    \item Activity: Write sample \sql{} queries \textcolor{gray}{and perhaps wrap these
      in Python functions}
    \end{itemize}


  \item \textcolor{gray}{Construct site-map and web-forms for implementing data transactions}



  \item \textcolor{gray}{Code Python-files for implementing data transactions in the
    site-map}

\end{enumerate}
\end{frame}



\begin{frame}

\frametitle{Exercises}

\begin{itemize}
\item BBC country information 
%\item Course database
%\item Extra exercise: modulo-11 check of CPR numbers
\end{itemize}


\begin{block}{Next time}
  \begin{itemize}
  \item \sql{} continued\ldots
  \end{itemize}
\end{block}
\end{frame}

\end{document}


