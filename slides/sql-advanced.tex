\documentclass[dvipsnames]{beamer}


\usepackage{preamble}
\usepackage{coursespecific}
\renewcommand*{\IncludeAnswers}{1}
\renewcommand*{\Lecture}{2}
\subtitle{SQL}



\begin{document}

\begin{frame}
  \titlepage{}
\end{frame}

\begin{frame}
  \frametitle{Previously On Data \& Algorithms}

\begin{itemize}
\item We use \intro{\sql} to manipulate and query our databases
\item \intro{\texttt{CREATE TABLE}} for creating tables
\item \intro{\texttt{INSERT}}, \intro{\texttt{DELETE}}, and
  \intro{\texttt{UPDATE}} for manipulating our data.
\item \intro{\texttt{SELECT}} for query our data
  \begin{itemize}
  \item we use \intro{\texttt{*}} for selecting all \emph{columns}; or we
    select columns by explicit naming them
  \item we use \intro{\texttt{WHERE}} to select which \emph{rows} we
    want in the result
  \item we can \emph{join} tables dynamically in queries with
    \intro{\texttt{,}} (or \intro{\texttt{JOIN}})
  \end{itemize}

\end{itemize}
\end{frame}



\begin{frame}
\frametitle{\sql{} Continued}

\begin{itemize}
\item Queries with computed fields
\item Aggregation and grouping with \texttt{GROUP BY}
\item Combination of \texttt{GROUP BY} and \texttt{HAVING}
\item Left and right controlled joining: \texttt{LEFT JOIN} and
  \texttt{RIGHT JOIN}
\item Indexes and efficiency
\item Creating indexes (\texttt{CREATE INDEX}) and unique indexes
\item Naming tables in queries, self-joins
%\item Which tables does the database contain?
%\item Loading data from a file
%\item Database design
%\item Normal forms
\end{itemize}
\end{frame}


% \begin{frame}[fragile=singleslide]
% \frametitle{Real Databases---RDBMS's (the ACID test)} 

% %A good database system will pass  ``the ACID test'':

% \begin{itemize}
% \item \intro{\textbf{A}tomicity:} either \emph{all} the effects of a
%   transaction are recorded, or \emph{nothing} is recorded
% \begin{itemize}
% \item When transferring money between accounts: either \emph{both}
%   accounts are updated, or \emph{none} of them are
% \end{itemize}

% \item \intro{\textbf{C}onsistency:} transactions change the database
%   from one legal state to another legal state
% \begin{itemize}
% \item The sum of a customer's accounts must be positive
% \end{itemize}

% \item \intro{\textbf{I}solation:} a transaction's effects are
%   invisible for other transactions until it commits
% \begin{itemize}
% \item Transferring between accounts while generating total balance
%   reports concurrently
% \end{itemize}

% \item \intro{\textbf{D}urability:} committed transactions survive
%   future system crashes
% \begin{itemize}
% \item When a customer has gotten his cash deposit receipt, the deposit
%   will not disappear due to server crashes
% \end{itemize}

% \end{itemize}
% \end{frame}


% \begin{frame}
%   \frametitle{ACID Properties}
  
%   \begin{itemize}
%   \item The ACID properties are for eliminating certain classes of
%     error (or rather to make it possible to avoid them).

%   \item The ACID properties are important for database supported web
%     sites, because web sites inherently have concurrent users. 

%   \item The ACID properties are especially important for sites with
%     critical data.

%   \item MySQL tables can be created to pass the ACID test, by using
%     InnoDB table type.  But at a performance cost.

%   \item Databases that do not fully pass the ACID test may still be
%     useful for sites where data is not critical.
%   \end{itemize}
% \end{frame}



% \begin{frame}

% \frametitle{Example: Course Data}
% \begin{small}
%   \begin{tabular}{@{}l@{}l}
%     \textbf{Students:} &
%     \begin{tabular}{|l|l|l|}\hline
%       sname & sid & address\\\hline
%       Ole Olesen & L1234 & Hjemmevej 7\\
%       Rikke Richardsen & L2345 & Udvej 9\\
%       Søren Sørensen & J0007 & I dybet 13\\
%       Ulrikka Funder & B7117 & Skovvej 11\\\hline
%       % INSERT INTO Student (sname,sid,address) VALUES ('Ole Olesen',
%       % 'L1234', 'Hjemmevej 7'), ('Rikke Richardsen', 'L2345', 'Udvej
%       % 9'), ('Søren Sørensen', 'J0007', 'I dybet 13'), ('Ulrikka
%       % Funder', 'B7117', 'Skovvej 11');
%     \end{tabular}\\
%     \\

%     \textbf{Courses:} &
%     \begin{tabular}{|l|l|l|}\hline
%       cid & cname & teacher\\\hline
%       DSDS & Introduction to Scripting, \dots & Ken Friis Larsen \\
%       GP & Introduction to Programming & Thomas Hildebrandt \\
%       XML & XML Processing & Lars Birkedal \\ \hline
%       % INSERT INTO Courses (cid,cname,teacher) VALUES ('DSDS', 'Web
%       % Publishing with Databases', 'Martin Elsman'), ('GP',
%       % 'Introduction to Programming', 'Morten Rhiger'), ('XML', 'XML
%       % Processing', 'Thomas Hildebrandt'); INSERT INTO Courses
%       % (cid,cname,teacher) VALUES ('ISKS', 'Advanced Language
%       % Implementation', 'Martin Elsman');
%     \end{tabular}\\
%     \\

%     \textbf{Signup:} &
%     \begin{tabular}{|l|l|}\hline
%       sid & cid\\\hline
%       L1234 & DSDS\\
%       L2345 & DSDS\\
%       J0007 & DSDS\\
%       J0007 & GP\\
%       L2345 & GP\\\hline
%       % INSERT INTO Signup (sid,cid) VALUES ('L1234','DSDS'),
%       % ('L2345','DSDS'), ('J0007','DSDS'), ('J0007','GP'),
%       % ('L2345','GP');
%     \end{tabular}
%   \end{tabular}
% \end{small}
% \end{frame}

% \begin{frame}[fragile=singleslide]
% \frametitle{Creating Tables Using \texttt{create table} Commands}

% \begin{verbatim}
%   CREATE TABLE Students (sname VARCHAR(80), 
%                          sid VARCHAR(10) NOT NULL, 
%                          address VARCHAR(80));

%   CREATE TABLE Courses (cid VARCHAR(10) NOT NULL, 
%                         cname VARCHAR(80), 
%                         teacher VARCHAR(80));

%   CREATE TABLE Signup (sid VARCHAR(10) NOT NULL, 
%                        cid VARCHAR(10) NOT NULL);
 
%   INSERT INTO Students (sname,sid,address) 
%   VALUES ('Ole Olesen', 'L1234', 'Hjemmevej 7'), 
%          ('Rikke Richardsen', 'L2345', 'Udvej 9'), 
%          ('Søren Sørensen', 'J0007', 'I dybet 13'), 
%          ('Ulrikka Funder', 'B7117', 'Skovvej 11');
%   ...
% \end{verbatim}

% \end{frame}

\begin{frame}[fragile] 
\frametitle{SELECT In Different Contexts}
  \begin{itemize}[<+->]
  \item As a calculator:
    \begin{small}
\begin{verbatim}
  SELECT 23 + 19;
  SELECT name, gdp / population FROM BBC;
\end{verbatim}
    \end{small}
  \item In \texttt{WHERE} clauses:
    \begin{small}
\begin{verbatim}
  SELECT name, population FROM BBC
  WHERE population > (SELECT population*20 FROM BBC 
                      WHERE name = 'Denmark');
  SELECT name, region FROM BBC
  where region IN (SELECT region FROM BBC 
                   WHERE name IN ('India', 'Iran'));
\end{verbatim}
    \end{small}
  \end{itemize}
\end{frame}


\begin{frame}[fragile]
\frametitle{Aggregated Results}

  \begin{itemize}[<+->]
  \item Display the total number of countries in the database:
    \begin{small}
\begin{verbatim}
  SELECT COUNT(*) FROM BBC;
\end{verbatim}
    \end{small}

  \item Fields can be named or renamed using \intro{\texttt{AS}}:
    \begin{small}
\begin{semiverbatim}
  SELECT COUNT(*) \intro{AS number_of_countries} FROM BBC;
  SELECT name, gdp / population \intro{AS gdc} FROM BBC;
\end{semiverbatim}
    \end{small}


  \item The number of countries in each region:
    \begin{small}
\begin{verbatim}
  SELECT region, COUNT(name) FROM BBC 
  GROUP BY region;
\end{verbatim}
    \end{small}
  \end{itemize}
\end{frame}

\begin{frame}\frametitle{Standard Aggregation Functions}

\begin{center}
\begin{tabular}{ll}\hline\hline
\textbf{Name} & \textbf{Meaning}\\\hline
\texttt{COUNT}(expression) & Number of values \\
\texttt{MIN}(expression) & Smallest value (or \texttt{NULL}) \\
\texttt{MAX}(expression) & Largest value (or \texttt{NULL}) \\
\texttt{SUM}(expression) & The sum of the values (or \texttt{NULL}) \\
\texttt{AVG}(expression) & The average of the values \\\hline\hline
\end{tabular}
\end{center}


The function \texttt{COUNT}, for instance, only counts the number of
non-\texttt{NULL} values.
  
\end{frame}

\begin{frame}[fragile=singleslide]
\frametitle{Another Example: Expenses At A Hospital}

\begin{columns}[t]
  \begin{column}{.5\textwidth}
    \begin{footnotesize}
\begin{verbatim}
CREATE TABLE Expenses (
  kind   TEXT NOT NULL,
  pgroup TEXT NOT NULL,
  year   INT NOT NULL,
  value  INT NOT NULL
);

INSERT INTO Expenses
VALUES ('Salary', 'Doctors', 2004, 490000);
INSERT INTO Expenses
VALUES ('Salary', 'Nurses', 2005, 1500000);
...
\end{verbatim}
  \end{footnotesize}
\end{column}

  \begin{column}{.5\textwidth}
    \texttt{Expenses} \\
    \begin{footnotesize}
      \begin{tabular}{|l|l|l|r|}\hline
        kind     & pgroup        & year & value \\\hline
        Salary   & Doctors & 2004 & 490,000.- \\ 
        Salary   & Nurses       & 2005 & 1,500,000.- \\
        Salary   & Doctors & 2005 & 500,000.- \\
        Coffee   & Doctors & 2006 & 800.- \\
        Coffee   & Nurses       & 2006 & 300.- \\
        Salary   & Nurses       & 2006 & 1,600,000.- \\
        Salary   & Doctors & 2006 & 510,000.- \\
        \hline
      \end{tabular}
    \end{footnotesize}
  \end{column}
\end{columns}
\end{frame}


\begin{frame}[fragile]
\frametitle{More Aggregation And \texttt{GROUP BY}}
  
\begin{itemize}

\item Aggregation and \texttt{GROUP BY} can be used for generating sub-totals.


\item Example: Total expenses grouped by personnel group, kind, and year for
  the years 2005--2005:
\begin{small}
\begin{verbatim}
  SELECT pgroup, kind, year, SUM(value) 
  FROM Expenses
  WHERE 2005 <= year AND year <= 2006 
  GROUP BY pgroup, kind, year; 
\end{verbatim}
\end{small}
    
\item \Question{How do we find, for each year, the total expenses of each kind?}{\texttt{SELECT year, kind, SUM(value) FROM Expenses\\ GROUP BY year, kind;}}
  
\item \Question{How do we find, for each year, the total expenses for each
  personnel group?}{\texttt{SELECT year, pgroup, SUM(value) FROM Expenses \\
    GROUP BY year, pgroup;}} 

  \end{itemize}
\end{frame}


\begin{frame}[fragile]
\frametitle{Conditions In Groupings}

\begin{itemize}

\item<1-> What if we don't want (kind, pgroup, year)-combinations
  where the sum of expenses is less than 1,000,000? 

\item<1-> We cannot use \texttt{WHERE SUM(value) >= 1000000} (because
  the sum is not yet calculated).

\item<2-> Instead, we can add \texttt{HAVING SUM(value) >= 1000000} after
  \texttt{GROUP BY}:
\begin{footnotesize}
\begin{verbatim}
  SELECT kind, pgroup, year, SUM(value) FROM Expenses
  WHERE 2005 <= year AND year <= 2006 
  GROUP BY kind, pgroup, year
  HAVING SUM(value) >= 1000000;
\end{verbatim}
\end{footnotesize}

\item<3-> \Question[4]{Consider the country example: How can you use
    \mbox{\texttt{GROUP BY \ldots{} HAVING}} construct to find the
    regions with at least 20 countries?}{\texttt{SELECT regions
      FROM BBC \\GROUP BY region
      \\HAVING COUNT(name) >= 20;}}

% SELECT cname FROM Courses, Signup WHERE Courses.cid = Signup.cid GROUP BY Courses.cid HAVING COUNT(sid) >= 3;
\end{itemize}
\end{frame}


\begin{frame}
\frametitle{A Database With Food And Breakfast Habits}
  
\begin{columns}[t]
\begin{column}{.5\textwidth}
\textbf{Dairy} \\
\begin{footnotesize}
\begin{tabular}{|l|r|r|r|}\hline 
   item    &     weight &  energy &  fat \\ \hline
   Semiskimmed &    200 &    400  &   3 \\
   Gaio     &       200 &    500  &   3 \\
   Skimmed     &    200 &    300  &   1 \\
   Buttermilk &     200 &    300  &   1 \\
   Cheasy &         200 &    325  &   1 \\
   A38 &            200 &    550  &   7 \\
   Cultura &        200 &    550  &   7 \\
   Fruit yoghurt &  200 &    750  &   6 \\ 
   Ymer &           200 &    600  &   7 \\
   Ylette &         200 &    500  &   4 \\ \hline
 \end{tabular}
\end{footnotesize}
%CREATE TABLE Dairy (item VARCHAR(100) NOT NULL, weight INT NOT NULL, energy INT NOT NULL, fat INT NOT NULL);
%INSERT INTO Dairy (item,weight,energy,fat) VALUES ('Semiskimmed', 200, 400 , 3), ('Gaio' , 200, 500 , 3), ('Skimmed' , 200, 300 , 1), ('Buttermilk', 200, 300 , 1), ('Cheasy', 200, 325 , 1), ('A38', 200, 550 , 7), ('Cultura', 200, 550 , 7), ('Fruit yoghurt', 200, 750 , 6), ('Ymer', 200, 600 , 7), ('Ylette', 200, 500 , 4);
\end{column}
\begin{column}{.4\textwidth}
\textbf{Morning} \\
\begin{footnotesize}
  \begin{tabular}{|l|r|r|}\hline
    name  &  item           & weight \\ \hline
    Ole   &  Semiskimmed    & 400 \\
    Ole   &  Fruit yoghurt  & 175 \\
    Ole   &  Brie           &  50 \\
    Ole   &  Butter         &  16 \\
    Hanne &  Cheasy         & 200 \\
    Hanne &  Skimmed        &  45 \\ \hline
    % CREATE TABLE Morning (name VARCHAR(100) NOT NULL, item
    % VARCHAR(100) NOT NULL, weight INT NOT NULL); INSERT INTO Morning
    % VALUES ('Ole', 'Semiskimmed', 400), ('Ole', 'Fruit yoghurt',
    % 175), ('Ole', 'Brie', 50), ('Ole', 'Butter', 16), ('Hanne',
    % 'Cheasy', 200), ('Hanne', 'Skimmed', 400);
  \end{tabular}
\end{footnotesize}
\end{column}  
\end{columns}

\begin{itemize}
\item The Dairy table contains item names, portion sizes (gms),
  energy content (kJ) and fat (gms) per portion.

\item The Morning table contains persons, items and quantities (gms).
\end{itemize}
\end{frame}


\begin{frame}[fragile]
  \frametitle{Computations In \sql{}-Queries}

\begin{itemize}    

\item<+-> You can write simple mathematical expressions in
  \sql{}-queries.  Find the energy and fat content for each person and
    item:
\begin{footnotesize}
\begin{verbatim}
 SELECT name, Morning.item,
        Morning.weight/Dairy.weight * Dairy.energy AS energy,
        Morning.weight/Dairy.weight * Dairy.fat AS fat
 FROM Morning, Dairy
 WHERE Dairy.item = Morning.item;
\end{verbatim}
\end{footnotesize}


\item<+-> Calculate the total energy- and fat intake of each person
\begin{footnotesize}
\begin{verbatim}
 SELECT name,
        SUM(Morning.weight/Dairy.weight * Dairy.energy) AS energy,
        SUM(Morning.weight/Dairy.weight * Dairy.fat) AS fat
 FROM Morning, Dairy
 WHERE Dairy.item = Morning.item
 GROUP BY name;
\end{verbatim}
\end{footnotesize}


\item<3-> The properties of the items and the personal data can be stored
  separately.
  
\item<3-> With \sql{} you can merge and perform calculations of the data.

\item<3-> It is quite cumbersome to do the same with spreadsheets.

\item<4-> Are there any problems with these queries?
% What about the Brie!
\end{itemize}
\end{frame}

\begin{frame}

\frametitle{Example: Course Data}
\begin{small}
  \begin{tabular}{@{}l@{}l}
    \texttt{Students:} &
    \begin{tabular}{|l|l|l|}\hline
      sname & sid & address\\\hline
      Ole Olesen & L1234 & Hjemmevej 7\\
      Rikke Richardsen & L2345 & Udvej 9\\
      Søren Sørensen & J0007 & I dybet 13\\
      Ulrikka Funder & B7117 & Skovvej 11\\\hline
    \end{tabular}\\
    \\

    \texttt{Courses:} &
    \begin{tabular}{|l|l|l|}\hline
      cid & cname & teacher\\\hline
      DAT2 & Data \& Algorithms & Ken Friis Larsen \\
      DAT1 & Introduction to programming & Jakob Grue Simonsen \\ \hline
      FP   & Functional programming & Jakob Grue Simonsen \\ \hline
    \end{tabular}\\
    \\

    \texttt{Signup:} &
    \begin{tabular}{|l|l|}\hline
      sid & cid\\\hline
      L1234 & DAT2\\
      L2345 & DAT2\\
      J0007 & DAT2\\
      J0007 & FP\\
      L2345 & FP\\\hline
    \end{tabular}
  \end{tabular}
\end{small}
\end{frame}




\begin{frame}[fragile]
\frametitle{Joining Only Includes Rows With At Least One match}

\begin{itemize}
\item<+-> Display the student IDs of students that have signed up for each
  course:
  \begin{small}
\begin{verbatim}
   SELECT cname, sid
   FROM Courses, Signup
   WHERE Courses.cid = Signup.cid;
\end{verbatim}
  \end{small}


\item<+-> Or: for each course, display the \emph{number} of students:

  \begin{small}
\begin{verbatim}
   SELECT cname, COUNT(sid)
   FROM Courses, Signup
   WHERE Courses.cid = Signup.cid
   GROUP BY cname;
\end{verbatim}
  \end{small}


\item<+-> Problem: only courses with students are shown.
\end{itemize}
\end{frame}


\begin{frame}[fragile]
\frametitle{Include All Rows From The Left Table: \texttt{LEFT JOIN}}
  
\begin{itemize}
\item<+-> Using \texttt{LEFT JOIN} all rows of the left table are
  included, even those that do not match any rows in the right table.
  \begin{footnotesize}\color<1>{gray}
\begin{semiverbatim}
   SELECT Courses.cname, Signup.sid
   FROM Courses \textcolor{black}{LEFT JOIN} Signup \textcolor{black}{ON} Courses.cid = Signup.cid;
\end{semiverbatim}
  \end{footnotesize}


\item<+-> Display the number of students for each course:
  \begin{footnotesize}
\begin{verbatim}
   SELECT cname, COUNT(sid)
   FROM Courses LEFT JOIN Signup ON Courses.cid = Signup.cid
   GROUP BY cname;
\end{verbatim}
  \end{footnotesize}


\item<+-> Display courses with less than two students:
  \begin{footnotesize}
\begin{verbatim}
   SELECT cname, COUNT(sid)
   FROM Courses LEFT JOIN Signup ON Courses.cid = Signup.cid
   GROUP BY cname HAVING COUNT(sid) < 2;
\end{verbatim}
  \end{footnotesize}

  % SELECT cname, COUNT(sid) as c FROM Courses LEFT JOIN Signup ON
  % Courses.cid = Signup.cid GROUP BY cname HAVING c < 2;


\item<4-> A normal join would forget courses with no students, so \texttt{LEFT
    JOIN} is important!

  % SELECT cname, COUNT(sid) FROM Courses, Signup WHERE Courses.cid =
  % Signup.cid GROUP BY cname HAVING COUNT(sid) < 2;

\item<4-> Correspondingly, \texttt{RIGHT JOIN} includes all rows from the
  right table, even those without a matching left table row.

\item<4-> \texttt{LEFT JOIN} and \texttt{RIGHT JOIN} are collectively known as
  \texttt{OUTER JOIN}.
\end{itemize}
\end{frame}


\begin{frame}[fragile=singleslide]
\frametitle{How Long Time Does A Query Take?}

\begin{itemize}
\item A join of two tables can take a long time to process.

\item Basically, each row in the one table must be matched with each
 row of the other.

\item So 10,000 rows in the one table and 10,000 in the
  other will require 10,000 * 10,000 = 100,000,000 comparisons.

\item This takes a long time.  When \texttt{Large1} and
  \texttt{Large2} each contain 10,000 lines, the following takes about
  30 seconds:
\begin{small}
\begin{verbatim}
   SELECT COUNT(id) 
   FROM Large1, Large2 
   WHERE Large1.id = Large2.id.
\end{verbatim}
\end{small}
  
\item Using \emph{indexes} on \texttt{Large1.id} and
  \texttt{Large2.id} it takes about 0.05 seconds; that is 400 times faster.

\item So the more rows, the more important is it to use indexes.
  
\item With 1,000,000 rows in each table, indexes will increase
  query speed by a factor of 15,000.
\end{itemize}
\end{frame}


\begin{frame}
\frametitle{What Is An Index?}

\begin{itemize}
\item An index is a ``telephone book'' for each column of a table.


\item In ordinary telephone books the following is true:

  \begin{itemize}
  \item It is hard to find the owner of a given number, because the
    numbers are not listed in order.

  \item Conversely, it is easy to find the number of a given owner,
    because the names are sorted alphabetically. 
  \end{itemize}


\item Using an index, you can find a specific value using just $\log_2(N)$
  comparisons when there are $N$ possible values.


\item Example: The National Register contains roughly 5,000,000
  persons (rows) in one table.


\item Locating a CPR number without using the index: 2,500,000 comparisons
  on average.


\item Locating a CPR number using the index: $\log_2(\mbox{5,000,000}) = 22.3$
  comparisons on average.
\end{itemize}
\end{frame}


\begin{frame}[fragile=singleslide]
\frametitle{Creating Indexes On Table Columns}

\begin{itemize}  
\item A table can have indexes on all fields
\item Creating an index on the \texttt{id} field in the
  \texttt{Large1} table:

\begin{small}
\begin{verbatim}
   CREATE INDEX Large1_id ON Large1 (id);
\end{verbatim}
\end{small}
    
\item You can require that each value in a column may occur at most once:

\begin{small}
\begin{verbatim}
   CREATE UNIQUE INDEX Students_sid ON Students (sid);
\end{verbatim}
\end{small}

This index will make it impossible to assign the same ID to two
students.  Try it!

\end{itemize}
\end{frame}


\begin{frame}[fragile=singleslide]
  \frametitle{Indexes On Combination Of Columns}
  \begin{itemize}
  \item A table can also have indexes on combinations of fields
 
\item You can make indexes combining several fields:
\begin{small}
\begin{verbatim}
   CREATE UNIQUE INDEX Signup_sid_cid ON Signup (sid, cid);
\end{verbatim}
\end{small}

This makes lookups of column combinations much faster than just having
indexes on the individual columns.

\item What is the consequence of the \texttt{unique} in the
\verb+Signup_sid_cid+ index?
  \end{itemize}
\end{frame}



\begin{frame}
\frametitle{Primary Keys}

\begin{itemize}  
\item A \intro{primary key} is a combination of fields that \intro{uniquely
  determine a row of the table}. Typically, the primary key is also the
  main vehicle for looking up rows of the table.

\item Examples: \texttt{sid} is the primary key of \texttt{Students},
\texttt{cid} is the primary key of \texttt{Courses} and \texttt{(sid, cid)}
is the primary key of \texttt{Signup}.

\end{itemize}
\end{frame}




\begin{frame}[fragile=singleslide]
  \frametitle{Declaring Primary Keys}
  \begin{itemize}
  \item You can declare the primary key when you create the table:
\begin{footnotesize}
\begin{semiverbatim}
  CREATE TABLE Students (sname TEXT NOT NULL,
                         sid VARCHAR(10) \textbf{PRIMARY KEY}, 
                         address TEXT NOT NULL);
\end{semiverbatim}
\end{footnotesize}

\item For multicolumn primary keys, each column
  must be \texttt{not null} and \texttt{primary key} is written just once:
\begin{footnotesize}
\begin{semiverbatim}
  CREATE TABLE Signup (sid VARCHAR(10) NOT NULL, 
                       cid VARCHAR(10) NOT NULL,
                       \textbf{PRIMARY KEY (sid, cid)});
\end{semiverbatim}
\end{footnotesize}

\item The following holds for primary keys:
\begin{itemize}
\item All columns in the primary key are automatically \texttt{NOT
    NULL} (well, not in SQLite).
\item A \texttt{UNIQUE INDEX} is automatically created on the
  combination of columns of the primary key.
\end{itemize}

  \end{itemize}
\end{frame}


% \begin{frame}[fragile=singleslide]

% \frametitle{Generating Unique ID Numbers In MySQL}


% \begin{itemize}
% \item In MySQL you can use \intro{\texttt{AUTO\_INCREMENT}} to generate fresh
%   ID numbers automatically when inserting new rows into a table:
%   \begin{footnotesize}
% \begin{semiverbatim}
% CREATE TABLE Persons (
%    person_id INT \textbf{AUTO_INCREMENT} PRIMARY KEY,
%    name      TEXT NOT NULL
%  );
% CREATE TABLE Phone_numbers (
%    person_id INT NOT NULL,
%    phone     VARCHAR(20) NOT NULL
% );
% CREATE TABLE Emails (
%    person_id INT NOT NULL,
%    email     TEXT NOT NULL
% );

% \textbf{INSERT INTO Persons (name) VALUES ('Ken Friis Larsen');}
% \textbf{INSERT INTO Persons (name) VALUES ('Mads Tofte');}
% \end{semiverbatim}
%   \end{footnotesize}


% \item (Other database systems provide similar functionality)

% \end{itemize}
% \end{frame}



\begin{frame}[fragile=singleslide]
\frametitle{Naming Tables In Queries, Self-Joins}

\begin{itemize}
\item We would like to find all pairs of courses that are taught by
  the same teacher.

\item i.e., find all pairs (\texttt{c1}, \texttt{c2}) of courses where
  the teacher of \texttt{c1} is identical to that of \texttt{c2}.
  
\item This can be done by joining the \texttt{Courses}-table with
  itself (self-join).

\item You can name the tables in a join by writing a new name (e.g.,
  \texttt{c1}) after the table name:

\begin{small}
\begin{verbatim}
  SELECT c1.cid, c2.cid 
  FROM Courses c1, Courses c2 
  WHERE c1.teacher = c2.teacher;
\end{verbatim}
\end{small}
    
%\item The query yields too many results:\\
%Both (15341, 15311) and (15311, 15341), and also rows like
%(10188,10188).
    
  \item Add \texttt{AND c1.cid < c2.cid} to exclude extraneous rows.
\end{itemize}
\end{frame}


% \begin{frame}[fragile=singleslide]
% \frametitle{Which Tables Does The Database Contain?}

% \begin{itemize}
% \item Display my tables:
%   \begin{small}
% \begin{verbatim}
%    SHOW TABLES;
% \end{verbatim}
%   \end{small}


% \item Display the names and types of table columns:
%   \begin{small}
% \begin{verbatim}
%    SHOW FIELDS FROM Table;
% \end{verbatim}
%   \end{small}


% \item Display the contents of a table:
%   \begin{small}
% \begin{verbatim}
%    SELECT * FROM Table;
% \end{verbatim}
%   \end{small}


% \item Loading data into a table from a text file
%   \begin{small}
% \begin{verbatim}
%   LOAD DATA LOCAL INFILE "wwwstats.txt" INTO TABLE Weblog;
% \end{verbatim}
%   \end{small}
% \end{itemize}
% \end{frame}

\begin{frame}
\frametitle{Constructing Database Supported Web Sites} 

\begin{enumerate}[Step 1:]
  \item Make a data model
    \begin{itemize}
    \item Which information should be stored and how should it be
      represented?
    \item Activity: write \texttt{CREATE TABLE} queries.
    \end{itemize}


  \item Develop data transactions
    \begin{itemize}
    \item How do we insert data into the database?
    \item How do we extract data from the database?
    \item Activity: Write sample \sql{} queries \textcolor{gray}{and perhaps wrap these
      in Python functions}
    \end{itemize}


  \item \textcolor{gray}{Construct site-map and web-forms for implementing data transactions}



  \item \textcolor{gray}{Code Python-files for implementing data transactions in the
    site-map}

\end{enumerate}
\end{frame}



\begin{frame}[fragile=singleslide]
\frametitle{Example: A Mailing List} 

\begin{itemize}
\item We want to create a mailing list system: we want to be able to
  create a list of names and emails for distributing emails.

\item \textbf{Step 1: The data model}

\begin{small}
\begin{verbatim}
CREATE TABLE Maillist (
  id         INT AUTO_INCREMENT PRIMARY KEY,
  email      varchar(100) NOT NULL,
  name       text NOT NULL
);
\end{verbatim}
\end{small}

\item The same common list is maintained by all people using the system.

\item The only information stored are the names and email addresses.

\item We insist that the name as well as the email address are present.
\end{itemize}
\end{frame}

\begin{frame}[fragile=singleslide]
\frametitle{Example: A Mailing List---Continued} 

\begin{itemize}
\item \textbf{Step 2: Data transactions}


\item The system is required to support the following kinds of data
  transactions:
  \begin{itemize}
  \item List all persons on the mailling list:
    \begin{small}
\begin{verbatim}
SELECT email, name FROM Maillist;
\end{verbatim}
    \end{small}
  \item Inserting new emails and names:
    \begin{small}
\begin{semiverbatim}
INSERT INTO Maillist (email,name) VALUES
  ('kflarsen@diku.dk', 'Ken Friis Larsen');
INSERT INTO Maillist (email,name) VALUES
  ('simonsen@diku.dk', 'Jakob Grue Simonsen');
\end{semiverbatim}
    \end{small}
  \end{itemize}
\item We will extend these system requirements later
\end{itemize}
\end{frame}



\begin{frame}
\frametitle{Next time}

\begin{itemize}
\item Hooking up SQL and Python
\item HTML and CSS
\item Web applications with Python
  % \item Websites that are databases
\end{itemize}

\end{frame}

\end{document}
