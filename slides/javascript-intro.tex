\documentclass[dvipsnames]{beamer}

\usepackage{preamble}
\usepackage{coursespecific}
%\renewcommand*{\IncludeAnswers}{1}

\renewcommand*{\Lecture}{13}
\subtitle{Introduction To JavaScript}


\begin{document}

\begin{frame}
  \titlepage{}
\end{frame}

\begin{frame} 
\frametitle{Today's Script}

\begin{itemize}
\item How does JavaScript fit in to out architecture, and what does it
  look like.
\item What can JavaScript be used for
\item Manipulating HTML in JavaScript
\item JIT Validation with JavaScript
\end{itemize}
  
\end{frame}


\begin{frame}
  \frametitle{Three-tier Architecture}
  \begin{center}
    \begin{tikzpicture}[thick,>=stealth]
    \footnotesize
    \fill[blue!25] (-1.5,-.8) rectangle (5.5,2.3);
    \path 
    (-5,2) node[draw,ellipse,minimum size=10mm](Client1) {Browser}
    (-5,0) node[draw,ellipse,minimum size=10mm](Client2) {Browser}
    (0,1.5) node[draw,rectangle,minimum size=10mm,fill=white,
                 text width=8em, text centered](WS) {Web Server\\(Apache)}
    (-.8,0) node[draw,rectangle,minimum size=10mm,fill=white,
                 text width=3em, text centered](PHP) {PHP file}
    (1,0) node[draw,rectangle,minimum size=10mm,fill=white,
                 text width=3em, text centered](HTML) {HTML file}
    (4,1.5) node[draw,rectangle,,minimum size=10mm,fill=white,
                 text width=8em, text centered](DB) {Database Server\\(MySQL)}
    ;
    \only<1|handout:0>{
      \draw[<->] (Client1) -- node[above,sloped]{HTML} (WS);
      \draw[<->] (Client2) -- node[above,sloped]{HTML} (WS);    
    }
    \only<2->{
      \draw[<->] (Client1) -- node[above,sloped]{HTML/JavaScript} (WS);
      \draw[<->] (Client2) -- node[above,sloped]{HTML/JavaScript} (WS);
    }
    \draw[<->] (Client1) -- node[below,sloped]{HTTP} (WS);
    \draw[<->] (Client2) -- node[below,sloped]{HTTP} (WS);    
    \draw[<->] (WS) -- node[above]{SQL} (DB);
    \draw[->] (PHP) -- (WS);
    \draw[->] (HTML) -- (WS);
    \end{tikzpicture}
  \end{center}


\begin{itemize}[<+->]
\item We still have a three-tier architecture
\item JavaScript/AJAX changes nothing
  \begin{itemize}
  \item It is just GUI bling
  \end{itemize}
\item JavaScript/AJAX changes everything
  \begin{itemize}
  \item We now have ``intelligent'' clients
  \end{itemize}

\item We should still \emph{never} trust data from the users.
\end{itemize}
\end{frame}


\begin{frame}[fragile=singleslide]
  \frametitle{History Repeats Itself}
  
  \begin{small}
\begin{semiverbatim}\color{gray}
<html><head><title>Hello JavaScript World</title></head>
<body>
 {\color{black}<script type="text/javascript">
  document.writeln("<p>Hi There,</p>");
  document.writeln("<p>Greetings from a JavaScript script</p>");   
 </script>} 
</body>
</html>
\end{semiverbatim}
  \end{small}

  \begin{itemize}
  \item The JavaScript code is kept in a \intro{\texttt{<script>}}
    element with the \intro{\texttt{type }attribute} set to
    \texttt{text/javascript}
  \item The \texttt{<script>} element can be placed
    \begin{itemize}
    \item \intro{in the \texttt{<body>} element}, when the JavaScript
      dynamically create the page content
    \item \intro{in the \texttt{<head>} elements}, when we define functions
      to be called later.
    \item \intro{in separate files}, when we have more than a trivial amount
      of code
    \end{itemize}
  \item We can use \intro{\texttt{document.writeln}} to dynamically
    write to the page.
  \end{itemize}
\end{frame}


\begin{frame}[fragile=singleslide] 
\frametitle{History Really Repeats Itself}

\begin{footnotesize}
\begin{semiverbatim}\color{gray}
<html><head><title>Oooh, Now With Function Calls</title>
{\color{black}<script type="text/javascript" >
function titulate( $male, $name ) \{
  if ( $male ) \{
    return "Mr. " + $name;
  \} else \{
    return "Mrs. " + $name;
  \}
\}
</script>}
</head>
<body>
 {\color{black}<script type="text/javascript">
  var today = Date();
  document.writeln("<h1>Hi "
                  + titulate( true, "Ken Friis Larsen")+"</h1>");
  document.writeln("<p>You loaded this webpage at " + today + "</p>");     
 </script>}
</body></html>
\end{semiverbatim}%$
\end{footnotesize}
\end{frame}

\begin{frame} 
\frametitle{Some Differences Between PHP and JavaScript}

\begin{itemize}
\item In JavaScript variables must be declared with
  \intro{\texttt{var}}
\item In JavaScript variables \emph{can} begin with a dollar sign, but
  they \emph{don't have to}.
\item In JavaScript you join two strings with \intro{\texttt{+}}
\item JavaScript don't have multiline strings
\item JavaScript don't expand variables inside strings
\item In JavaScript we modify the document in the browser rather
  sending it
\item (Completely different OO system, but fortunately that is outside
  the scope of this course)
\end{itemize}  
\end{frame}



\begin{frame}[fragile] 
\frametitle{Loops In JavaScript}
  
\begin{itemize}[<+->]

\item The file \texttt{mult.js}:
  \begin{footnotesize}
\begin{verbatim}
function multcells( rows, cols ) {
  var result = "";
  for( var r = 1; r <= rows; r = r + 1 ) {
    result += "<tr>";
    for( var c = 1; c <= cols; c++ ) {
      result += '<td align="right">' + (r*c) + "</td>";
    }
    result += "</tr>";
  }
  return result; 
}      
\end{verbatim}
  \end{footnotesize}

\item The file \texttt{simplemulttable.html}%
\begin{footnotesize}%
\begin{semiverbatim}
\color{gray}<html><head><title>Simple Multiplication Table</title>
\textcolor{black}{<script type="text/javascript" \intro{src="mult.js"} ></script>}
</head><body>
  <h1>Multiplication Table</h1>
  <table border="1px">
  <script type="text/javascript">
    \textcolor{black}{document.writeln( multcells(5,5) );}
  </script></table></body></html>
\end{semiverbatim}
\end{footnotesize}
\end{itemize}
\end{frame}


\begin{frame} 
\frametitle{History}

\begin{itemize}
\item Early 1995, Netscape hired Brendan Eich to make LiveScript for
  version 2.0 of the browser
\item Late 1995, Netscape and Sun made a deal, and LiveScript was
  renamed JavaScript
\item August 1996, Microsoft introduced JScript in version 3.0 of Internet
  Explorer
\item November 1996, Netscape submitted the JavaScript specification
  to ECMA International for standardization.
\item June 1997, ECMAScript (ECMA-262) first edition released.
\item June 1998, ECMAScript second edition
\item December 1999, ECMAScript third edition.  Corresponds roughly
  to JavaScript version 1.5
\item Work in progress, ECMAScript fourth edition.
\end{itemize}
  
\end{frame}


\begin{frame} 
\frametitle{JavaScript---Not Just For Web Applications}

\begin{itemize}
\item Windows batch scripting (Active Scripting)
\item Dashboard Widgets (Mac OS X Tiger)
\item Vista Widgets
\item Greasemonkey scripts (Firefox)
\item Unity3D game engine scripting
\item ActionScript is an other dialect of ECMAScript
\item PDF documents can contain JavaScript
\item Scripting of Adobe Photoshop
\item Can be (and is) embedded as scripting language for applications
  that needs it
\end{itemize}
  
\end{frame}


\begin{frame}[fragile] 
\frametitle{Manipulating HTML In JavaScript}

\begin{itemize}
\item We modify an HTML page in JavaScript by modifying the
  \intro{Document Object Model} (DOM)
\item The DOM is a hierarchical tree representation of an HTML document
\item Each node corresponds to either an \intro{element node}
  (corresponding to an HTML element) or a \intro{text node}
  (corresponding to the text in an HTML element).
\item We manipulate the DOM tree in two ways:
  \begin{itemize}
  \item By adding, removing, and modifying nodes
  \item By manipulating the CSS style of nodes
  \end{itemize}
\end{itemize}
\end{frame}

\begin{frame}[fragile] 
\frametitle{Multiplication Table By DOM}
  
\begin{itemize}
\item File \texttt{multdom.js}
  \begin{footnotesize}
\begin{semiverbatim}
function add_multcells( table, rows, cols ) \{
  for( var r = 1; r <= rows; r = r + 1 ) \{
    var row = table.\intro{insertRow}(table.rows.length);
    for( var c = 1; c <= cols; c++ ) \{
      var cell = row.\intro{insertCell}(row.cells.length);
      var content = \intro{document.createTextNode}( r * c );
      cell.\intro{appendChild}(content);
    \}
  \}
\}      
\end{semiverbatim}
  \end{footnotesize}
\item File \texttt{simplemulttable2.html}
\begin{footnotesize}
\begin{semiverbatim}
\color{gray}<html><head><title>Simple Multiplication Table</title>
<script type="text/javascript" src="multdom.js" ></script>
</head><body><h1>Multiplication Table</h1>
  <table border="1px" \textcolor{black}{id="multtab"}></table> 
  <script type="text/javascript">
    \textcolor{black}{var tab = document.getElementById("multtab");}
    add_multcells( tab, 10, 5 );
  </script></body></html>      
\end{semiverbatim}
  \end{footnotesize}
\end{itemize}
\end{frame}



\begin{frame}[fragile] 
\frametitle{Attaching JavaScript To Events}

\begin{footnotesize}
\begin{semiverbatim}
<html><head><title>Dynamic Multiplication Table</title>
<script type="text/javascript" src="multdom.js" ></script>
<script type="text/javascript">
function inputchanged() \{
  var cols = document.getElementById("cols");
  var rows = document.getElementById("rows");
  if ( cols.value != "" && rows.value != "" 
    && validate_int(cols.value) && validate_int(rows.value)) \{   
     var tab = document.getElementById("multtab");
     clear_table( tab );
     add_multcells( tab, rows.value, cols.value );
  \}
\}
</script></head><body>
  <h1>Dynamic Multiplication Table</h1>
  <form action="doesnotexists.php" >
  Rows: <input type="text" name="rows" id="rows" 
         \intro{onchange="inputchanged()}" /><br />
  Collumns: <input type="text" name="cols" id="cols" 
             \intro{onchange="inputchanged()}" />
</form><table border="1px" id="multtab"></table></body></html>    
\end{semiverbatim}
\end{footnotesize}
  
\end{frame}


\begin{frame} \frametitle{Next Time}

  \begin{itemize}
  \item Changing the CSS style of an element
  \item AJAX
    \begin{itemize}
    \item That is, some computations at the client some at the server.
    \end{itemize}
  \item Exam from Spring 2006
  \end{itemize}
  
\end{frame}



\end{document}

%%% Local Variables: 
%%% mode: latex
%%% TeX-master: t
%%% End: 

% LocalWords:  JavaScript ActionScript ECMAScript PDF CSS
