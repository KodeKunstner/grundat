\documentclass[a4paper,landscape]{slides} 
%\usepackage{a4} 
\usepackage[danish]{babel} 
\usepackage[utf8]{inputenc}
\usepackage{textcomp}
\usepackage{amsmath}
\usepackage{amssymb}
\usepackage{amsthm}
\usepackage{graphicx} 
\usepackage{verbatim} 
\usepackage{fancyhdr}
\usepackage{listings} 
\usepackage{url}
\frenchspacing
\pagestyle{plain}

\addtolength{\voffset}{-060pt}
\addtolength{\textheight}{060pt}

\title{Grundlæggende datalogi}
\author{Rasmus Jensen\footnote{\url{grundat@kommunikationogit.dk}}}
\date{2009-09-21}
\lstset{language=python}
\begin{document}
\maketitle

\begin{comment}
- Opsamling
  - Lister, dictionaries, sekvenser
  - Skildpaddegrafik
  - NB: tupler
  - Øvelse: n-kant - rekursivt og iterativt
  - Øvelse: Sierpinski trekant
  - Øvelse: polygonspiral

- Objekter
  - Skildpadder og skildpaddegrafik
  - Filer
  - Objekter
  - Hvad er et objekt
  - metoder og properties
  - funktionalitet koblet til data
  - Øvelse: Program der læser en fil med tegneinstruktioner og tegner den

- Deklaration af objekter
  - Modellering
  - Øvelse: quiz-objekt
  - Øvelse: familietræ - NB: overvej hvordan I ville implementerer ancestor-funktion

- Hvordan objekter virker
  - \verb|__dict__|, \verb|__class__|

- Eksempel: LightCycle
  - Beskrivelse af opgaven

- Mere om objekter:
  - Nedarvning, \verb|__bases__|
  - Interfaces og duck-typing
  - Lookup-regler
 
\end{comment}

\begin{slide}
	\begin{center} {\large 
            Plan for i dag
	} \end{center}
	\begin{itemize} \addtolength{\itemsep}{-\baselineskip}
		\item Opsamling: dictionaries, beregningskompleksitet, skildpaddegrafik, rekursion
		\item Supplement: tuples
                \item Anvendelse af objekter
		\item Klasser og egne objekter
		\item Detaljer
                \item Eksempel: lighecycle
	\end{itemize}
\end{slide}

\begin{slide}
	\begin{center} {\large 
            Opsamling
	} \end{center}
	\begin{itemize} \addtolength{\itemsep}{-\baselineskip}
                \item Lister, dictionaries
                \item Mutability
                \item Sekvens
                \item Beregningskompleksitet
                \item Idle, python som tekst, NB:.py
                \item Skildpaddegrafik
                \item Rekursion - at kalde sig selv
	\end{itemize}
\end{slide}

\begin{slide}
	\begin{center} {\large 
            Supplement
	} \end{center}
	\begin{itemize} \addtolength{\itemsep}{-\baselineskip}
                \item Tupler, - immutable lister
                \item pygame
                \item Øvelse: Sierpinskis trekant og koch-kurve
                \item Øvelse: Polygonspiraler
	\end{itemize}
\end{slide}

\begin{slide}
	\begin{center} {\large 
            Plan for i dag
	} \end{center}
	\begin{itemize} \addtolength{\itemsep}{-\baselineskip}
		\item Opsamling: dictionaries, beregningskompleksitet, skildpaddegrafik, rekursion
		\item Supplement: tuples
                \item Anvendelse af objekter
		\item Klasser og egne objekter
		\item Detaljer
                \item Eksempel: lighecycle
	\end{itemize}
\end{slide}


\begin{slide}
	\begin{center} {\large 
            Anvendelse af objekter
	} \end{center}
	\begin{itemize} \addtolength{\itemsep}{-\baselineskip}
                \item Hvad er et objekt, - funktioner tilknyttet data-klasse
                \item Eksempler: turtle, filer, strenge
                \item Python: alt er objekter
                \item .-notation
                \item Øvelse: indlæsning af tegneinstruktioner
	\end{itemize}
\end{slide}

\begin{slide}
	\begin{center} {\large 
            Plan for i dag
	} \end{center}
	\begin{itemize} \addtolength{\itemsep}{-\baselineskip}
		\item Opsamling: dictionaries, beregningskompleksitet, skildpaddegrafik, rekursion
		\item Supplement: tuples
                \item Anvendelse af objekter
		\item Klasser og egne objekter
		\item Detaljer
                \item Eksempel: lighecycle
	\end{itemize}
\end{slide}

\begin{slide}
	\begin{center} {\large 
            Klasser
	} \end{center}
	\begin{itemize} \addtolength{\itemsep}{-\baselineskip}
		\item class...
		\item funktioner og metodekald
                \item properties vs. klasse-variable vs. lokale variable 
		\item Eksempel/øvelse: Quiz med objekter
		\item Eksempel/øvelse: Familietræ - overvej ancester-funktion
	\end{itemize}
\end{slide}

\begin{slide}
	\begin{center} {\large 
            Plan for i dag
	} \end{center}
	\begin{itemize} \addtolength{\itemsep}{-\baselineskip}
		\item Opsamling: dictionaries, beregningskompleksitet, skildpaddegrafik, rekursion
		\item Supplement: tuples
                \item Anvendelse af objekter
		\item Klasser og egne objekter
		\item Detaljer
                \item Eksempel: lighecycle
	\end{itemize}
\end{slide}

\begin{slide}
	\begin{center} {\large 
            Mere om Objekter
	} \end{center}
	\begin{itemize} \addtolength{\itemsep}{-\baselineskip}
		\item Hvordan objekter virker: \verb|__class__|, \verb|__dict__| 
                \item Duck-typing, interfaces, klasser
                \item Nedarvning, \verb|__bases__|, resolve-mekanisme
                \item Abstraktion
                \item Modellering
                \item \verb|__getitem__|..., \verb|__add__|, \verb|__str__|, ...
                \item Eksempel: lightcycle
	\end{itemize}
\end{slide}

\begin{slide}
	\begin{center} {\large 
            Plan for i dag
	} \end{center}
	\begin{itemize} \addtolength{\itemsep}{-\baselineskip}
		\item Opsamling: dictionaries, beregningskompleksitet, skildpaddegrafik, rekursion
		\item Supplement: tuples
                \item Anvendelse af objekter
		\item Klasser og egne objekter
		\item Detaljer
                \item Eksempel: lighecycle
	\end{itemize}
\end{slide}

\end{document}
