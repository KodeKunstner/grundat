\documentclass[a4paper,landscape]{slides} 
%\usepackage{a4} 
\usepackage[danish]{babel} 
\usepackage[utf8]{inputenc}
\usepackage{textcomp}
\usepackage{amsmath}
\usepackage{amssymb}
\usepackage{amsthm}
\usepackage{graphicx} 
\usepackage{verbatim} 
\usepackage{fancyhdr}
\usepackage{listings} 
\usepackage{url}
\frenchspacing
\pagestyle{plain}

\addtolength{\voffset}{-060pt}
\addtolength{\textheight}{060pt}

\title{Grundlæggende datalogi}
\author{Rasmus Jensen\footnote{\url{grundat@kommunikationogit.dk}}}
\date{2009-09-03}
\lstset{language=python}
\begin{document}
\maketitle


\begin{slide}
	\begin{center} {\large 
            Plan for i dag
	} \end{center}
	\begin{itemize} \addtolength{\itemsep}{-\baselineskip}
    		\item Sidste uge, feedback på opgaver. Indbygget dokumentation, og eksperimentering
    		\item Problemløsning. Strenge, docstrings, lister, mere om for-løkker
    		\item Læsevejledning/studieteknik ifbm. noter. Filer, filsystem. Dokumentation og brug af indbyggede funktioner
    		\item Betingelser, boolske værdier.
    		\item Revy
    		\item Sammenfatning, afrunding
	\end{itemize}
\end{slide}

\begin{slide}
	\begin{center} {\large 
            Sidste uge
	} \end{center}
	\begin{itemize} \addtolength{\itemsep}{-\baselineskip}
		\item Funktionsdefinitioner
		\item For-løkker
		\item Kommetarer
		\item Kald af indbyggede funktioner
                \item Billeder
                \item Afleveringer (NB.: film, spil)
	\end{itemize} 
\end{slide}

\begin{slide}
	\begin{center} {\large 
            Genopfriskning, samt indbyggede funktioner
	} \end{center}
	\begin{itemize} \addtolength{\itemsep}{-\baselineskip}
		\item Husk: overblik over JES-funktioner, samt indbygget dokumentation
		\item Lav funktion der inverterer et billede. Sort (0,0,0) til hvid (255,255,255), rød(255,0,0) til turkis(0,255,255), mørkeblå(0,0,70) til lysegul(255,255,185), etc.
		\item Find ud af hvorledes man gemmer et billede fra JES
		\item Find ud af hvordan man sammensætter billeder til en kort film
	\end{itemize}
\end{slide}

\begin{slide}
	\begin{center} {\large 
            Plan for i dag
	} \end{center}
	\begin{itemize} \addtolength{\itemsep}{-\baselineskip}
    		\item Sidste uge, feedback på opgaver. Indbygget dokumentation, og eksperimentering
    		\item Problemløsning. Strenge, docstrings, lister, mere om for-løkker
    		\item Læsevejledning/studieteknik ifbm. noter. Filer, filsystem. Dokumentation og brug af indbyggede funktioner
    		\item Betingelser, boolske værdier.
    		\item Revy
    		\item Sammenfatning, afrunding
	\end{itemize}
\end{slide}


\begin{slide}
	\begin{center} {\large 
            Problemløsning (Pólya)
	} \end{center}
	\begin{itemize} \addtolength{\itemsep}{-\baselineskip}
		\item Forstå opgaven/udfordringen
		\item Find tilgang / læg en plan for løsningen
		\item Gennemfør det
		\item Review det der er gjort
		\item \emph{Find/start med en enklere opgave}
                \item Inkrementel udvikling
	\end{itemize}
\end{slide}

\begin{slide}
	\begin{center} {\large 
            Addendum til sidste uge
	} \end{center}
	\begin{itemize} \addtolength{\itemsep}{-\baselineskip}
            \item Strenge, """..."""
            \item for-løkker på andet end billeder
            \item Lister
	\end{itemize}
\end{slide}


\begin{slide}
	\begin{center} {\large 
            Øvelse
	} \end{center}
	\begin{itemize} \addtolength{\itemsep}{-\baselineskip}
            \item Kombinerede ord, - find sammensatte ord, og generer kombinationerne af prefiks og suffiks.
            \item Lav program der spiller en melodi på baggrund af et navn - bogstav til tone
            \item Hav en liste af koordinatpar, og tegn linje mellem disse
	\end{itemize}
\end{slide}

\begin{slide}
	\begin{center} {\large 
            Plan for i dag
	} \end{center}
	\begin{itemize} \addtolength{\itemsep}{-\baselineskip}
    		\item Sidste uge, feedback på opgaver. Indbygget dokumentation, og eksperimentering
    		\item Problemløsning. Strenge, docstrings, lister, mere om for-løkker
    		\item Læsevejledning/studieteknik ifbm. noter. Filer, filsystem. Dokumentation og brug af indbyggede funktioner
    		\item Betingelser, boolske værdier.
    		\item Revy
    		\item Sammenfatning, afrunding
	\end{itemize}
\end{slide}

\begin{slide}
	\begin{center} {\large 
            Læsevejledning, øvelser, aflevering og filer
	} \end{center}
	\begin{itemize} \addtolength{\itemsep}{-\baselineskip}
            \item Skimning vs. læsning
            \item In-progress
            \item Praktisk ifbm. aflevering
            \item om filer...
            \item skema, rss
	\end{itemize}
\end{slide}


\begin{slide}
	\begin{center} {\large 
            Indbyggede funktioner og dokumentation
	} \end{center}
	\begin{itemize} \addtolength{\itemsep}{-\baselineskip}
		\item JES, python, Jython, other.
		\item Python funktioner via import
                \item At læse funktionsdokumentation
		\item Objektnotation
                \item Dokumentation
                \item ord, chr, random.choice, range.
	\end{itemize}
\end{slide}

\begin{slide}
	\begin{center} {\large 
            Øvelser
	} \end{center}
	\begin{itemize} \addtolength{\itemsep}{-\baselineskip}
    		\item 100 bottles of beer
    		\item Tegn manuelt et fyldt rektangel
    		\item Lav et billede ægte sort/hvidt
    		\item Blur og sharpen
	\end{itemize}
\end{slide}

\begin{slide}
	\begin{center} {\large 
            Plan for i dag
	} \end{center}
	\begin{itemize} \addtolength{\itemsep}{-\baselineskip}
    		\item Sidste uge, feedback på opgaver. Indbygget dokumentation, og eksperimentering
    		\item Problemløsning. Strenge, docstrings, lister, mere om for-løkker
    		\item Læsevejledning/studieteknik ifbm. noter. Filer, filsystem. Dokumentation og brug af indbyggede funktioner
    		\item Betingelser, boolske værdier.
    		\item Revy
    		\item Sammenfatning, afrunding
	\end{itemize}
\end{slide}

\begin{slide}
	\begin{center} {\large 
            Betingelser
	} \end{center}
	\begin{itemize} \addtolength{\itemsep}{-\baselineskip}
		\item Sandhedsværdier
		\item if, elseif, else
                \item and, or, not
                \item Lidt mere om billeder
	\end{itemize}
\end{slide}

\begin{slide}
	\begin{center} {\large 
            Øvelser
	} \end{center}
	\begin{itemize} \addtolength{\itemsep}{-\baselineskip}
    		\item Lav billede ægte sort/hvidt
	\begin{itemize} \addtolength{\itemsep}{-\baselineskip}
                    \item Automatisk omkring gennemsnitsfargven
	\end{itemize}
                \item Posterisering
    		\item Simpel regnequiz
    		\item Markér over/under-belysning
    		\item Lav funktion der finder den lyseste farve i et billede, og en der finder den mørkeste.
    		\item Forøg kontrasten såden mørkeste bliver sort og den lyseste hvid.
    		\item Lav blur, skarphed
	\end{itemize}
\end{slide}

\begin{slide}
	\begin{center} {\large 
            Plan for i dag
	} \end{center}
	\begin{itemize} \addtolength{\itemsep}{-\baselineskip}
    		\item Sidste uge, feedback på opgaver. Indbygget dokumentation, og eksperimentering
    		\item Problemløsning. Strenge, docstrings, lister, mere om for-løkker
    		\item Læsevejledning/studieteknik ifbm. noter. Filer, filsystem. Dokumentation og brug af indbyggede funktioner
    		\item Betingelser, boolske værdier.
    		\item Revy
    		\item Sammenfatning, afrunding
	\end{itemize}
\end{slide}

\begin{slide}
	\begin{center} {\large 
            Sammenfatning, næste gang
	} \end{center}
	\begin{itemize} \addtolength{\itemsep}{-\baselineskip}
		\item Fem minutter, noter hvad I kan huske som det vigtigste - noter til jer selv.
		\item Fem minutter, diskuter det med sidekammerat, lav fælles noter, send kopi per mail.
		\item Næste gang: whileløkker
	\end{itemize}
\end{slide}


\end{document}
