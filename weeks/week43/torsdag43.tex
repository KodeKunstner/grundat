\documentclass[a4paper,landscape]{slides} 
%\usepackage{a4} 
\usepackage[danish]{babel} 
\usepackage[utf8]{inputenc}
\usepackage{textcomp}
\usepackage{amsmath}
\usepackage{amssymb}
\usepackage{amsthm}
\usepackage{graphicx} 
\usepackage{verbatim} 
\usepackage{fancyhdr}
\usepackage{listings} 
\usepackage{url}
\frenchspacing
\pagestyle{plain}

\addtolength{\voffset}{-060pt}
\addtolength{\textheight}{060pt}

\title{Grundlæggende datalogi}
\author{Rasmus Jensen\footnote{\url{grundat@kommunikationogit.dk}}}
\date{2009-10-21}
\lstset{language=python}
\begin{document}
\maketitle

\begin{comment}

- Opsamling, debugging og wiki
  - html-forms-navne, variabel-navne og kolonne-navne
  - wiki
  - dagens opgave: afleveringsopgaven til næste uge.
  - spørg også gerne hinanden om hjælp
  - debugging

- Tilstand i program
    - Flere sider fra samme program, - dispatch
    - Stateful vs. stateless connection, - hidden fields
- Evt. opsamling om programdesign
  - Inkrementel udvikling

- Programmeringssprog
    - Hvor bruges python...
    - Generelle programmeringsteknikker vs. konkrete sprog
    - Syntaks og semantik
    - Scriptingsprog, systemprogrammeringssprog
    - Programmeringsparadigmer: procedural programmering, OOP, funktionsorienteret programmering, logikprogrammering.
    - Dynamisk og statiske typer.

- JavaScript og python
- Spørgsmål?



\end{comment}




\begin{slide}
	\begin{center} {\large 
            Plan for i dag
	} \end{center}
	\begin{itemize} \addtolength{\itemsep}{-\baselineskip}
            \item Indledning, wiki, debugging
            \item Tilstand i programmer
            \item Programmeringssprog 1/2
            \item Programmeringssprog 2/2
            \item Opsamling
	\end{itemize}
\end{slide}

\begin{slide}
	\begin{center} {\large 
            Indledning, wiki
	} \end{center}
	\begin{itemize} \addtolength{\itemsep}{-\baselineskip}
            \item Html-form-navne vs. python-navne vs. kolonne-navne
            \item Hjælp hinanden
            \item Programmering vs. teknisk funktionalitet
            \item Wiki wiki wiki
            \item Dagens opgave - afleveringen til næste uge. 
	\end{itemize}
\end{slide}


\begin{slide}
	\begin{center} {\large 
           Debugging
	} \end{center}
	\begin{itemize} \addtolength{\itemsep}{-\baselineskip}
            \item Debugning - syntaksfejl, køretidsfejl, semantiske fejl
            \item Confessional debugging, ekstra par øjne
            \item Læs programmet, vær sikker på i ved hvorfor det gør som det gør.
            \item Læs fejlbeskederne, forstå fejlen
            \item Eksperimentér i fortolkeren
            \item Test
            \item Fejl i Python?
	\end{itemize}
\end{slide}
\begin{slide}
	\begin{center} {\large 
            Plan for i dag
	} \end{center}
	\begin{itemize} \addtolength{\itemsep}{-\baselineskip}
            \item Indledning, wiki, debugging
            \item Tilstand i programmer
            \item Programmeringssprog 1/2
            \item Programmeringssprog 2/2
            \item Opsamling
	\end{itemize}
\end{slide}

\begin{slide}
	\begin{center} {\large 
            Tilstand i program
	} \end{center}
	\begin{itemize} \addtolength{\itemsep}{-\baselineskip}
            \item Flere sider fra samme program
            \item Stateful vs. stateless kommunikation / forbindelse
            \item State: hidden fields, cookies, javascript, ...
	\end{itemize}
\end{slide}

\begin{slide}
	\begin{center} {\large 
            Plan for i dag
	} \end{center}
	\begin{itemize} \addtolength{\itemsep}{-\baselineskip}
            \item Indledning, wiki, debugging
            \item Tilstand i programmer
            \item Programmeringssprog 1/2
            \item Programmeringssprog 2/2
            \item Opsamling
	\end{itemize}
\end{slide}
\begin{slide}
	\begin{center} {\large 
            Programmeringssprog 1/2
	} \end{center}
	\begin{itemize} \addtolength{\itemsep}{-\baselineskip}
            \item Syntaks og semantik
            \item Repetition af Python
            \item Introduktion til \emph{programmeringssproget} JavaScript 
	\end{itemize}
\end{slide}

\begin{slide}
	\begin{center} {\large 
            Plan for i dag
	} \end{center}
	\begin{itemize} \addtolength{\itemsep}{-\baselineskip}
            \item Indledning, wiki, debugging
            \item Tilstand i programmer
            \item Programmeringssprog 1/2
            \item Programmeringssprog 2/2
            \item Opsamling
	\end{itemize}
\end{slide}
\begin{slide}
	\begin{center} {\large 
            Programmeringssprog 2/2
	} \end{center}
	\begin{itemize} \addtolength{\itemsep}{-\baselineskip}
            \item Dynamisk og statiske typer.
            \item Scriptingsprog, systemprogrammeringssprog
            \item Generelle programmeringsteknikker vs. konkrete sprog
            \item Programmeringsparadigmer: procedural programmering, OOP, funktionsorienteret programmering, logikprogrammering.
	\end{itemize}
\end{slide}

\begin{slide}
	\begin{center} {\large 
            Plan for i dag
	} \end{center}
	\begin{itemize} \addtolength{\itemsep}{-\baselineskip}
            \item Indledning, wiki, debugging
            \item Tilstand i programmer
            \item Programmeringssprog 1/2
            \item Programmeringssprog 2/2
            \item Opsamling
	\end{itemize}
\end{slide}

\begin{slide}
	\begin{center} {\large 
            Afrunding
	} \end{center}
	\begin{itemize} \addtolength{\itemsep}{-\baselineskip}
            \item Videre med opgave
            \item Fælles spørgsmål?
	\end{itemize}
\end{slide}

\begin{slide}
	\begin{center} {\large 
            Plan for i dag
	} \end{center}
	\begin{itemize} \addtolength{\itemsep}{-\baselineskip}
            \item Indledning, wiki, debugging
            \item Tilstand i programmer
            \item Programmeringssprog 1/2
            \item Programmeringssprog 2/2
            \item Opsamling
	\end{itemize}
\end{slide}

\end{document}
