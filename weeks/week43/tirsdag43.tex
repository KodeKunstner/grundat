\documentclass[a4paper,landscape]{slides} 
%\usepackage{a4} 
\usepackage[danish]{babel} 
\usepackage[utf8]{inputenc}
\usepackage{textcomp}
\usepackage{amsmath}
\usepackage{amssymb}
\usepackage{amsthm}
\usepackage{graphicx} 
\usepackage{verbatim} 
\usepackage{fancyhdr}
\usepackage{listings} 
\usepackage{url}
\frenchspacing
\pagestyle{plain}

\addtolength{\voffset}{-060pt}
\addtolength{\textheight}{060pt}

\title{Grundlæggende datalogi}
\author{Rasmus Jensen\footnote{\url{grundat@kommunikationogit.dk}}}
\date{2009-10-19}
\lstset{language=python}
\begin{document}
\maketitle

\begin{comment}

- Opsamling samt praktisk
    - Spørgsmål
    - Oprettelse af database - både via python og sqlite3-kommando
    - NB: SQL kommandoer kan lægges i fil
    - Midtvejsevaluering næste uge.
    - Evt. SQL-eksempel adressebog+tilknytning
- Øvelse: Bruger/login-system
    - Lav et python cgi-script, der sender en hjemmeside med loginformular til webbrowseren. Action på scriptet skal være selve programmet. (En del af øvelsen er at de forskellige sider ligger i en og samme fil.)
    - Lav en database med brugernavne, kodeord og login-id
    - Udvid cgi-scriptet så det tjekker om brugernavn+kodeord er i databasen og skriver en fejlmeddelse hvis det er forkert.
    - Lav en formular der kun kan udfyldes hvis brugernavn+koderord er korrekt.
    - Gem resultatet af formularen i databasen sammen med login-id. Hint: login-id kan overføres via et hidden-inputfelt.

- Database design (normalisering)
    - Eksempel adressebog+tilknytning
    - Separate tabeller for hvert sæt af egenskaber
    - Primær nøgle
    - Flyt data gentaget i rækker til separate tabeller
    - Eliminér kolonner der ikke afhænger af nøgle.

- Mere SQL
    - Agregate functions
    - Nested selects 
- Tilstand i program
    - Flere sider fra samme program, - dispatch
    - Stateful vs. stateless connection, - hidden fields

- Programmeringssprog
    - Generelle programmeringsteknikker vs. konkrete sprog
    - Syntaks og semantik
    - Scriptingsprog, systemprogrammeringssprog
    - Programmeringsparadigmer: procedural programmering, OOP, funktionsorienteret programmering, logikprogrammering.
    - Dynamisk og statiske typer.
    - Torsdag: JavaScript

- Spørgsmål?


\end{comment}

\begin{slide}
	\begin{center} {\large 
            Plan for i dag
	} \end{center}
	\begin{itemize} \addtolength{\itemsep}{-\baselineskip}
		\item Opsamling. Dagens øvelse
                \item Database design
                \item SQL. Tilstand i applikationer
                \item Programmeringssprog. 
	\end{itemize}
\end{slide}

\begin{slide}
	\begin{center} {\large 
            Opsamling samt praktisk
	} \end{center}
	\begin{itemize} \addtolength{\itemsep}{-\baselineskip}
            \item Udførsel af SQL - python vs. sqlite3-kommando
            \item NB: SQL kommandoer kan lægges i fil
            \item Midtvejsevaluering næste uge.
            \item SQL-eksempel adressebog+tilknytning
	\end{itemize}
\end{slide}

\begin{slide}
	\begin{center} {\large 
            Øvelse: Bruger/login-system
	} \end{center}
	\begin{itemize} \addtolength{\itemsep}{-\baselineskip}
            \item Lav et python cgi-script, der sender en hjemmeside med loginformular til webbrowseren. Action på scriptet skal være selve programmet. (En del af øvelsen er at de forskellige sider ligger i en og samme fil.)
            \item Lav en database med brugernavne, kodeord og login-id
            \item Udvid cgi-scriptet så det tjekker om brugernavn+kodeord er i databasen og skriver en fejlmeddelse hvis det er forkert.
            \item Lav en formular der kun kan udfyldes hvis brugernavn+koderord er korrekt.
            \item Gem resultatet af formularen i databasen sammen med login-id. Hint: login-id kan overføres via et hidden-inputfelt.
	\end{itemize}
\end{slide}

\begin{slide}
	\begin{center} {\large 
            Plan for i dag
	} \end{center}
	\begin{itemize} \addtolength{\itemsep}{-\baselineskip}
		\item Opsamling. Dagens øvelse
                \item Database design
                \item SQL. Tilstand i applikationer
                \item Programmeringssprog. 
	\end{itemize}
\end{slide}

\begin{slide}
	\begin{center} {\large 
            Database design (normalisering)
	} \end{center}
	\begin{itemize} \addtolength{\itemsep}{-\baselineskip}
            \item Eksempel adressebog+tilknytning
            \item Separate tabeller for hvert sæt af egenskaber
            \item Primær nøgle
            \item Flyt data gentaget i rækker til separate tabeller
            \item Eliminér kolonner der ikke afhænger af nøgle.
	\end{itemize}
\end{slide}

\begin{slide}
	\begin{center} {\large 
            Plan for i dag
	} \end{center}
	\begin{itemize} \addtolength{\itemsep}{-\baselineskip}
		\item Opsamling. Dagens øvelse
                \item Database design
                \item SQL. Tilstand i applikationer
                \item Programmeringssprog. 
	\end{itemize}
\end{slide}

\begin{slide}
	\begin{center} {\large 
            Mere SQL
	} \end{center}
	\begin{itemize} \addtolength{\itemsep}{-\baselineskip}
            \item Agregate functions
            \item SUM COUNT AVG
            \item Nested SELECTs
            \item SELECT a, b FROM table WHERE b IN (SELECT x FROM table)
	\end{itemize}
\end{slide}

\begin{slide}
	\begin{center} {\large 
            Plan for i dag
	} \end{center}
	\begin{itemize} \addtolength{\itemsep}{-\baselineskip}
		\item Opsamling. Dagens øvelse
                \item Database design
                \item SQL. Tilstand i applikationer
                \item Programmeringssprog. 
	\end{itemize}
\end{slide}
\begin{slide}
	\begin{center} {\large 
            Tilstand i program
	} \end{center}
	\begin{itemize} \addtolength{\itemsep}{-\baselineskip}
            \item Flere sider fra samme program
            \item Stateful vs. stateless kommunikation / forbindelse
            \item State: hidden fields, cookies, javascript, ...
	\end{itemize}
\end{slide}

\begin{slide}
	\begin{center} {\large 
            Plan for i dag
	} \end{center}
	\begin{itemize} \addtolength{\itemsep}{-\baselineskip}
		\item Opsamling. Dagens øvelse
                \item Database design
                \item SQL. Tilstand i applikationer
                \item Programmeringssprog. 
	\end{itemize}
\end{slide}
\begin{slide}
	\begin{center} {\large 
            Programmeringssprog
	} \end{center}
	\begin{itemize} \addtolength{\itemsep}{-\baselineskip}
            \item Generelle programmeringsteknikker vs. konkrete sprog
            \item Syntaks og semantik
            \item Scriptingsprog, systemprogrammeringssprog
            \item Programmeringsparadigmer: procedural programmering, OOP, funktionsorienteret programmering, logikprogrammering.
            \item Dynamisk og statiske typer.
            \item Torsdag: JavaScript
	\end{itemize}
\end{slide}

\begin{slide}
	\begin{center} {\large 
            Afrunding
	} \end{center}
	\begin{itemize} \addtolength{\itemsep}{-\baselineskip}
            \item Videre med opgave
            \item Fælles spørgsmål?
	\end{itemize}
\end{slide}


\end{document}
