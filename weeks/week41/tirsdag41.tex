\documentclass[a4paper,landscape]{slides} 
%\usepackage{a4} 
\usepackage[danish]{babel} 
\usepackage[utf8]{inputenc}
\usepackage{textcomp}
\usepackage{amsmath}
\usepackage{amssymb}
\usepackage{amsthm}
\usepackage{graphicx} 
\usepackage{verbatim} 
\usepackage{fancyhdr}
\usepackage{listings} 
\usepackage{url}
\frenchspacing
\pagestyle{plain}

\addtolength{\voffset}{-060pt}
\addtolength{\textheight}{060pt}

\title{Grundlæggende datalogi}
\author{Rasmus Jensen\footnote{\url{grundat@kommunikationogit.dk}}}
\date{2009-10-5}
\lstset{language=python}
\begin{document}
\maketitle

\begin{comment}
 - opsamling/repetition
 - afleveringsopgaven
 - kommandoprompt, ssh, tilgang til server etc.
 - repetition http, introduktion til cgi
 - opsamling om html, forms, strukturerede data
\end{comment}

\begin{slide}
	\begin{center} {\large 
            Plan for i dag
	} \end{center}
	\begin{itemize} \addtolength{\itemsep}{-\baselineskip}
		\item 
	\end{itemize}
\end{slide}

\begin{slide}
	\begin{center} {\large 
            Opsamling
	} \end{center}
	\begin{itemize} \addtolength{\itemsep}{-\baselineskip}
                \item 
	\end{itemize}
\end{slide}

\end{document}
