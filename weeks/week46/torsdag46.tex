\documentclass[a4paper,landscape]{slides} 
%\usepackage{a4} 
\usepackage[danish]{babel} 
\usepackage[utf8]{inputenc}
\usepackage{textcomp}
\usepackage{amsmath}
\usepackage{amssymb}
\usepackage{amsthm}
\usepackage{graphicx} 
\usepackage{verbatim} 
\usepackage{fancyhdr}
\usepackage{listings} 
\usepackage{url}
\frenchspacing
\pagestyle{plain}

\addtolength{\voffset}{-060pt}
\addtolength{\textheight}{060pt}

\title{Grundlæggende datalogi}
\author{Rasmus Jensen\footnote{\url{grundat@kommunikationogit.dk}}}
\date{2009-11-9}
\lstset{language=python}
\begin{document}
\maketitle
\begin{comment}
* Praktisk
** Forum, - gerne m. spørgsmål/svar.
** Dagens gennemgående eksempel: søgetræer
** Præsentationer af IDE'er etc. til næste torsdag.
** Sig til når det går for hurtigt, og hvis der er emner I trænger til at får uddybet.
** IT-projekter, planlægning, inkrementel udvikling, design vs. kodning, dokumentation 
** Hvad kan det bruges til: sammenhæng med den `rigtige' verden. Underliggende forståelse/dannelse, samt konkrete færdigheder.
* Databaser og træstrukturer. Klasser og objekter
** Tabeller vs. databaser
** Motivation: Databaser, hvordan fungerer de... Query optimiser, Indeks, 
** Start på eksempel med binære søgetræer.
** Design og overvejelser, træ-strukturer
* Søgetræer (fortsat)
** Praktisk implementation og test
** Unit-test, intern afprøvning
** API og afgrænsning
* Opsamling og perspektivering
** Databaser og søgetræer in real life, hukommelseshierakier
** Syntaks og semantik, standardiserede sprog, HTML vs. Python vs. SQL
** Modularisering af programmer, selvstændige dele.
\end{comment}
\begin{slide}
	\begin{center} {\large 
            Plan for i dag
	} \end{center}
	\begin{itemize} \addtolength{\itemsep}{-\baselineskip}
            \item Praktisk/indledning - gennemgående eksempel med træstruktur
            \item Databaser og træstrukturer. Klasser og objekter
            \item Søgetræer - fortsat
            \item Opsamling og perspektivering
	\end{itemize}
\end{slide}

\begin{slide}
	\begin{center} {\large 
            Praktisk
	} \end{center}
	\begin{itemize} \addtolength{\itemsep}{-\baselineskip}
                \item Forum, - gerne m. spørgsmål/svar. 
                \item Dagens gennemgående eksempel: søgetræer 
                \item Præsentationer af IDE'er etc. på næste torsdag. 
                \item Sig til når det går for hurtigt, og hvis der er emner I trænger til at får uddybet. 
                \item IT-projekter, planlægning, inkrementel udvikling, design vs. kodning, dokumentation 
                \item Hvad kan det bruges til: sammenhæng med den `rigtige' verden. Underliggende forståelse/dannelse, samt konkrete færdigheder. 
	\end{itemize}
\end{slide}

\begin{slide}
	\begin{center} {\large 
            Plan for i dag
	} \end{center}
	\begin{itemize} \addtolength{\itemsep}{-\baselineskip}
            \item Praktisk/indledning - gennemgående eksempel med træstruktur
            \item Databaser og træstrukturer. Klasser og objekter
            \item Søgetræer - fortsat
            \item Opsamling og perspektivering
	\end{itemize}
\end{slide}

\begin{slide}
	\begin{center} {\large 
            Databaser og træstrukturer. Klasser og objekter 
	} \end{center}
	\begin{itemize} \addtolength{\itemsep}{-\baselineskip}
            \item Tabeller vs. databaser 
            \item Motivation: Databaser, hvordan fungerer de... Query optimiser, Indeks, 
            \item Start på eksempel med binære søgetræer. 
            \item Design og overvejelser, træ-strukturer 
            \item Unit-test, intern afprøvning
	\end{itemize}
\end{slide}

\begin{slide}
	\begin{center} {\large 
            Plan for i dag
	} \end{center}
	\begin{itemize} \addtolength{\itemsep}{-\baselineskip}
            \item Praktisk/indledning - gennemgående eksempel med træstruktur
            \item Databaser og træstrukturer. Klasser og objekter
            \item Søgetræer - fortsat
            \item Opsamling og perspektivering
	\end{itemize}
\end{slide}

\begin{slide}
	\begin{center} {\large 
            Søgetræer (fortsat) 
	} \end{center}
	\begin{itemize} \addtolength{\itemsep}{-\baselineskip}
            \item Praktisk implementation og test 
            \item NB: Kapitel 20
	\end{itemize}
\end{slide}

\begin{slide}
	\begin{center} {\large 
            Plan for i dag
	} \end{center}
	\begin{itemize} \addtolength{\itemsep}{-\baselineskip}
            \item Praktisk/indledning - gennemgående eksempel med træstruktur
            \item Databaser og træstrukturer. Klasser og objekter
            \item Søgetræer - fortsat
            \item Opsamling og perspektivering
	\end{itemize}
\end{slide}

\begin{slide}
	\begin{center} {\large 
            Opsamling og perspektivering 
	} \end{center}
	\begin{itemize} \addtolength{\itemsep}{-\baselineskip}
            \item Databaser og søgetræer in real life, hukommelseshierakier 
            \item Syntaks og semantik, standardiserede sprog, HTML vs. Python vs. SQL 
            \item Modularisering af programmer, selvstændige dele. API og grænseflader
	\end{itemize}
\end{slide}

\begin{slide}
	\begin{center} {\large 
            Plan for i dag
	} \end{center}
	\begin{itemize} \addtolength{\itemsep}{-\baselineskip}
            \item Praktisk/indledning - gennemgående eksempel med træstruktur
            \item Databaser og træstrukturer. Klasser og objekter
            \item Søgetræer - fortsat
            \item Opsamling og perspektivering
	\end{itemize}
\end{slide}


\end{document}
