\documentclass[a4paper,landscape]{slides} 
%\usepackage{a4} 
\usepackage[danish]{babel} 
\usepackage[utf8]{inputenc}
\usepackage{textcomp}
\usepackage{amsmath}
\usepackage{amssymb}
\usepackage{amsthm}
\usepackage{graphicx} 
\usepackage{verbatim} 
\usepackage{fancyhdr}
\usepackage{listings} 
\usepackage{url}
\frenchspacing
\pagestyle{plain}

\addtolength{\voffset}{-060pt}
\addtolength{\textheight}{060pt}

\title{Grundlæggende datalogi}
\author{Rasmus Jensen\footnote{\url{grundat@kommunikationogit.dk}}}
\date{2009-09-16}
\lstset{language=python}
\begin{document}
\maketitle


\begin{slide}
	\begin{center} {\large 
            Plan for i dag
	} \end{center}
	\begin{itemize} \addtolength{\itemsep}{-\baselineskip}
		\item Opsamling fra sidst: afleveringer, feedback, samt dictionaries
                \item Code review
		\item Diskussion af createCollage...
                \item Binær søgning og beregningskompleksitet
		\item Kommandoprompt, samt IPython
		\item Skildpaddegrafik 
                \item Rekursion
	\end{itemize}
\end{slide}

\begin{slide}
	\begin{center} {\large 
            Opsamling siden sidst: Afleveringsopgaven, feedback, samt dictionaries
	} \end{center}
	\begin{itemize} \addtolength{\itemsep}{-\baselineskip}
                \item Spejling i diagonalen - designvalg
                \item Tak for feedback, stil spørgsmål, sig til når det går for hurtigt, hjælp hinanden, noter/hjemmeside/feedback, lokale/programmeringscafe.
                \item Idle/JES: indbyggede funktioner i JES
                \item Øvelse fra sidst: direkte oversættelse
                \item Øvelse: Hyppighed af bogstaver i en tekst
                \item Øvelse: Funktion der gentager sig selv uden brug af for eller while-løkker...
	\end{itemize}
\end{slide}

\begin{slide}
	\begin{center} {\large 
            Plan for i dag
	} \end{center}
	\begin{itemize} \addtolength{\itemsep}{-\baselineskip}
		\item Opsamling fra sidst: afleveringer, feedback, samt dictionaries
                \item Code review
		\item Diskussion af createCollage...
                \item Binær søgning og beregningskompleksitet
		\item Kommandoprompt, samt IPython
		\item Skildpaddegrafik 
                \item Rekursion
	\end{itemize}
\end{slide}


\begin{slide}
	\begin{center} {\large 
            Code review
	} \end{center}
	\begin{itemize} \addtolength{\itemsep}{-\baselineskip}
                \item Kodekonventioner, opsætning, navngivning, kommentarsprog, læsbarhed
                \item Dokumentation, docstrings
                \item Dont repeat yourself - DRY
                \item Not Invented Here - NIH
                \item At læse ved at følge flowet.
                \item Længde af funktioner
                \item Test/afprøvning
                \item Code review som redskab ifbm. udvikling, øvelse: læs hinandens kode
	\end{itemize}
\end{slide}


\begin{slide}
	\begin{center} {\large 
            Createcollage - gennemgang af koden
	} \end{center}
	\begin{itemize} \addtolength{\itemsep}{-\baselineskip}
                \item Læsbarhed ifbm. opsætning, linjeskift/luft
                \item 0/1-indeksstart
                \item Navngivningskonventioner Java vs Python
                \item Refaktorering
                \item \verb|copyInto(...)|
                \item Rækkefølge af hvad koden gør, i.e. først redigering så skrivning af billeder. Opdeling i logisk afsnit
                \item Docstring/kommentar der fortæller hvad funktionen gør
                \item Kommentarer undervejs
                \item Længde af funktion
	\end{itemize}
\end{slide}

\begin{slide}
	\begin{center} {\large 
            Plan for i dag
	} \end{center}
	\begin{itemize} \addtolength{\itemsep}{-\baselineskip}
		\item Opsamling fra sidst: afleveringer, feedback, samt dictionaries
                \item Code review
		\item Diskussion af createCollage...
                \item Binær søgning og beregningskompleksitet
		\item Kommandoprompt, samt IPython
		\item Skildpaddegrafik 
                \item Rekursion
	\end{itemize}
\end{slide}

\begin{slide}
	\begin{center} {\large 
             Binær søgning og beregningskompleksitet
	} \end{center}
	\begin{itemize} \addtolength{\itemsep}{-\baselineskip}
                \item Gæt-et-tal-eksemplet 10: 4, 100: 7, 1000: 10, 10000: 14. 
                \item Binær søgning
                \item Intervalstørrelse for gæt. $x_0 = 1$,  $x_n = 2x_{n-1} + 1$
                \item Beregning via python
                \item Worst case og $\mathcal{O}$-notation. Afslappet i forhold til konstanter.
                \item Binær søgning: beregningskompleksit $\mathcal{O}(\log(n))$
	\end{itemize}
\end{slide}

\begin{slide}
	\begin{center} {\large 
            Plan for i dag
	} \end{center}
	\begin{itemize} \addtolength{\itemsep}{-\baselineskip}
		\item Opsamling fra sidst: afleveringer, feedback, samt dictionaries
                \item Code review
		\item Diskussion af createCollage...
                \item Binær søgning og beregningskompleksitet
		\item Kommandoprompt, samt IPython
		\item Skildpaddegrafik 
                \item Rekursion
	\end{itemize}
\end{slide}

\begin{slide}
	\begin{center} {\large 
            Kommandoprompt samt ipython
	} \end{center}
	\begin{itemize} \addtolength{\itemsep}{-\baselineskip}
                \item A la fortolkervinduet i Python
                \item \verb|cd| \verb|ls| \verb|cp| \verb|mv| \verb|rm| \verb|pwd| \verb|python| \verb|ipython|
                \item Udførsel af programmer, argumenter
                \item Work directory, path, scripts, \verb|#!/usr/bin/env python|
                \item Pipes, redirection, variable
                \item Shells, sh-varianter, Windows PowerShell
                \item Hvorfor: sekvens af tekst i stedet for klik, analogi-tifingersystemet.
	\end{itemize}
\end{slide}

\begin{slide}
	\begin{center} {\large 
            Plan for i dag
	} \end{center}
	\begin{itemize} \addtolength{\itemsep}{-\baselineskip}
		\item Opsamling fra sidst: afleveringer, feedback, samt dictionaries
                \item Code review
		\item Diskussion af createCollage...
                \item Binær søgning og beregningskompleksitet
		\item Kommandoprompt, samt IPython
		\item Skildpaddegrafik 
                \item Rekursion
	\end{itemize}
\end{slide}

\begin{slide}
	\begin{center} {\large 
            Skildpaddegrafik
	} \end{center}
	\begin{itemize} \addtolength{\itemsep}{-\baselineskip}
                \item Vektorgrafik vs. rastergrafik
		\item Øvelse: lav funktion der tegner en trekant med skildpaddegrafik, en der tegner en firkant
                \item Skildpadde som et objekt
		\item Øvelse: lav funktion der tegner en n-kant med skildpaddegrafik
	\end{itemize}
\end{slide}

\begin{slide}
	\begin{center} {\large 
            Plan for i dag
	} \end{center}
	\begin{itemize} \addtolength{\itemsep}{-\baselineskip}
		\item Opsamling fra sidst: afleveringer, feedback, samt dictionaries
                \item Code review
		\item Diskussion af createCollage...
                \item Binær søgning og beregningskompleksitet
		\item Kommandoprompt, samt IPython
		\item Skildpaddegrafik 
                \item Rekursion
	\end{itemize}
\end{slide}

\begin{slide}
	\begin{center} {\large 
            Rekursion
	} \end{center}
	\begin{itemize} \addtolength{\itemsep}{-\baselineskip}
                \item Funktion der kalder sig selv
                \item Stop-betingelse via if
                \item Praktisk hvis man itererer over struktur
		\item Eksempel små til store bogstaer i liste af lister/strenge
		\item Eksempel med fraktal - koch-kurve
		\item Øvelse: lav program der tegner sierpinskis trekant
	\end{itemize}
\end{slide}

\end{document}
