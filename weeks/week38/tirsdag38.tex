\documentclass[a4paper,landscape]{slides} 
%\usepackage{a4} 
\usepackage[danish]{babel} 
\usepackage[utf8]{inputenc}
\usepackage{textcomp}
\usepackage{amsmath}
\usepackage{amssymb}
\usepackage{amsthm}
\usepackage{graphicx} 
\usepackage{verbatim} 
\usepackage{fancyhdr}
\usepackage{listings} 
\usepackage{url}
\frenchspacing
\pagestyle{plain}

\addtolength{\voffset}{-060pt}
\addtolength{\textheight}{060pt}

\title{Grundlæggende datalogi}
\author{Rasmus Jensen\footnote{\url{grundat@kommunikationogit.dk}}}
\date{2009-09-14}
\lstset{language=python}
\begin{document}
\maketitle


\begin{slide}
	\begin{center} {\large 
            Plan for i dag
	} \end{center}
	\begin{itemize} \addtolength{\itemsep}{-\baselineskip}
		\item Opsamling
		\item Biblioteksfunktioner
                \item Lister og strenge
		\item Problemløsning - hvordan man kommer i gang
                \item Dictionaries
		\item Idle
		\item Rekursion
	\end{itemize}
\end{slide}

\begin{slide}
	\begin{center} {\large 
            Opsamling
	} \end{center}
	\begin{itemize} \addtolength{\itemsep}{-\baselineskip}
                \item Kald af funktioner
                \item Billeder, farvemodeller
                \item Variable
                \item Definition af funktioner: def
                \item Gentagelse: while, for
                \item Problemløsning, inkrementel udvikling
                \item Lister, strenge, heltal
                \item Brug af biblioteksfunktioner, import ...
	\end{itemize}
\end{slide}

\begin{slide}
	\begin{center} {\large 
            Biblioteksfunktioner
	} \end{center}
	\begin{itemize} \addtolength{\itemsep}{-\baselineskip}
                \item Builtin, standard biblioteket, eksterne api'er, python integreret som scriptingsprog.
                \item Moduler, samling af funktioner etc.  \verb|import ...| Kald af funktioner med .-notation.  \verb|from ... import ...|
                \item docs.python.org, google, ...
                \item Øvelse: lav program der henter en hjemmeside, solsort.com, og udskriver den på skærmen (hint: urllib2)
                \item Udfordring: lav en funktion der gentager sig selv, uden brug af while/for-loops...
	\end{itemize}
        \begin{verbatim}
        Den er stor som véd, men større,
        den som ved hvor man kan spørre,
                               - Kumbel
        \end{verbatim}
\end{slide}

\begin{slide}
	\begin{center} {\large 
            Plan for i dag
	} \end{center}
	\begin{itemize} \addtolength{\itemsep}{-\baselineskip}
		\item Opsamling
		\item Biblioteksfunktioner
                \item Lister og strenge
		\item Problemløsning - hvordan man kommer i gang
                \item Dictionaries
		\item Idle
		\item Rekursion
	\end{itemize}
\end{slide}


\begin{slide}
	\begin{center} {\large 
            Lister og strenge
	} \end{center}
	\begin{itemize} \addtolength{\itemsep}{-\baselineskip}
                \item Repetition: definition af strenge og lister, for, random.choice
                \item Udvælgelse af elementer, subscript, slice
                \item Mutable - kan ændres
		\item Sekvens - itererbar
                \item Funktioner: split, strip, 
                \item Øvelse/eksempel - form-brev
                \item Øvelse/eksempel - quiz med spørgsmål
	\end{itemize}
\end{slide}

\begin{slide}
	\begin{center} {\large 
            Plan for i dag
	} \end{center}
	\begin{itemize} \addtolength{\itemsep}{-\baselineskip}
		\item Opsamling
		\item Biblioteksfunktioner
                \item Lister og strenge
		\item Problemløsning - hvordan man kommer i gang
                \item Dictionaries
		\item Idle
		\item Rekursion
	\end{itemize}
\end{slide}


\begin{slide}
	\begin{center} {\large 
            Problemløsning, - at komme i gang med programmeringen
	} \end{center}
	\begin{itemize} \addtolength{\itemsep}{-\baselineskip}
		\item Beskriv hvad programmet skal kunne gøre
		\item Start eventuelt med kommentarer
                \item Inkrementel udvikling
                \item Øvelse/eksempel - quiz med spørgsmål
                \item Øvelse/eksempel - fortune-program
	\end{itemize}
\end{slide}

\begin{slide}
	\begin{center} {\large 
            Plan for i dag
	} \end{center}
	\begin{itemize} \addtolength{\itemsep}{-\baselineskip}
		\item Opsamling
		\item Biblioteksfunktioner
                \item Lister og strenge
		\item Problemløsning - hvordan man kommer i gang
                \item Dictionaries
		\item Idle
		\item Rekursion
	\end{itemize}
\end{slide}


\begin{slide}
	\begin{center} {\large 
            Dictionaries/hashtabeller
	} \end{center}
	\begin{itemize} \addtolength{\itemsep}{-\baselineskip}
		\item Konstanter i stedet for tal som indices. Større fleksibilitet
		\item NB: \verb|has_key|
                \item Øvelse/eksempel: Simpelt direkte oversættelsesprogram.
                \item Eksempel på hvad det svarer til med lister (langsom implementation). Hashtabeller.
                \item Øvelse/eksempel: Ordfrekvens
	\end{itemize}
\end{slide}

\begin{slide}
	\begin{center} {\large 
            Plan for i dag
	} \end{center}
	\begin{itemize} \addtolength{\itemsep}{-\baselineskip}
		\item Opsamling
		\item Biblioteksfunktioner
                \item Lister og strenge
		\item Problemløsning - hvordan man kommer i gang
                \item Dictionaries
		\item Idle
		\item Rekursion
	\end{itemize}
\end{slide}


\begin{slide}
	\begin{center} {\large 
            Idle
	} \end{center}
	\begin{itemize} \addtolength{\itemsep}{-\baselineskip}
                \item Om python miljøer, fortolker vs. editor
                \item JES, IPython, Idle
                \item Jython, CPython, IronPython, ...
		\item Sammenligning mellem JES og Idle
                \item NB: 2.6 vs. 3.1
	\end{itemize}
\end{slide}

\begin{slide}
	\begin{center} {\large 
            Rekursion
	} \end{center}
	\begin{itemize} \addtolength{\itemsep}{-\baselineskip}
                \item Funktion der kalder sig selv
                \item Stop-betingelse via if
                \item Praktisk hvis man itererer over struktur
		\item Eksempel med fraktal - koch-kurve
	\end{itemize}
\end{slide}


\begin{slide}
	\begin{center} {\large 
            Sammenfatning, næste gang
	} \end{center}
	\begin{itemize} \addtolength{\itemsep}{-\baselineskip}
		\item Fem minutter, noter hvad I kan huske som det vigtigste - noter til jer selv.
		\item Fem minutter, diskuter det med sidekammerat
	\end{itemize}
\end{slide}


\end{document}
