\documentclass[a4paper,landscape]{slides} 
%\usepackage{a4} 
\usepackage[danish]{babel} 
\usepackage[utf8]{inputenc}
\usepackage{textcomp}
\usepackage{amsmath}
\usepackage{amssymb}
\usepackage{amsthm}
\usepackage{graphicx} 
\usepackage{verbatim} 
\usepackage{fancyhdr}
\usepackage{listings} 
\usepackage{url}
\frenchspacing
\pagestyle{plain}

\addtolength{\voffset}{-060pt}
\addtolength{\textheight}{060pt}

\title{Grundlæggende datalogi}
\author{Rasmus Jensen\footnote{\url{grundat@kommunikationogit.dk}}}
\date{2009-10-28}
\lstset{language=python}
\begin{document}
\maketitle
\begin{slide}\begin{center} {\large Plan for i dag } \end{center} \begin{itemize} \addtolength{\itemsep}{-\baselineskip}
    \item  Opsamling fra sidst
    \item  Bits, bytes og talsystemer
    \item  Objekter og databaser
    \item  Informationteori
\end{itemize} \end{slide} 
\begin{slide}\begin{center} {\large Opsamling fra sidst } \end{center} \begin{itemize} \addtolength{\itemsep}{-\baselineskip}
    \item  Praktisk: wiki-egne-noter, samt starting :15-sharp
    \item  Beregningskompleksitet
    \item  Problemløsning - divide and conquer
    \item  Øvelse: lav funktion der sorterer en liste (og når denne virker, kig på afleveringsopgaven)
\end{itemize} \end{slide} \begin{slide}\begin{center} {\large Løsning af sorteringsopgaven } \end{center} \begin{itemize} \addtolength{\itemsep}{-\baselineskip}
\item ... kode ...
\end{itemize} \end{slide} \begin{slide}\begin{center} {\large Plan for i dag } \end{center} \begin{itemize} \addtolength{\itemsep}{-\baselineskip} \item  Opsamling fra sidst \item  Bits, bytes og talsystemer \item  Objekter og databaser \item  Informationteori
\end{itemize} \end{slide} \begin{slide}\begin{center} {\large Bits, bytes og talsystemer:  } \end{center} \begin{itemize} \addtolength{\itemsep}{-\baselineskip}
    \item  Binære tal, - hvordan ``tænker'' en computer. (repetition)
    \item  Hexadecimale tal, oktale tal...
    \item  bits, bytes, words
    \item  Størrelse på hukommelse
    \item  HTML farver..
\end{itemize} \end{slide} \begin{slide}\begin{center} {\large Øvelser med tal: } \end{center} \begin{itemize} \addtolength{\itemsep}{-\baselineskip}
    \item  Binære tal: 11, 101, 110010, 10000, 101011, 110, 100, 1010101, 1101, 1110
    \item  Hexadecimale tal: 123, A4, FF, 100, EF, 1B, 1000, 10000, FF00, F, CD, CC   
    \item  Decimale tal: 321, 256, 65535, 65536, 17, 42, 57, 216, 512, 600, 25, 100
    \item  Octale tal: 31, 17, 123, 100
\end{itemize} \end{slide} \begin{slide}\begin{center} {\large Plan for i dag } \end{center} \begin{itemize} \addtolength{\itemsep}{-\baselineskip} \item  Opsamling fra sidst \item  Bits, bytes og talsystemer \item  Objekter og databaser \item  Informationteori
\end{itemize} \end{slide} \begin{slide}\begin{center} {\large Objekter og databaser } \end{center} \begin{itemize} \addtolength{\itemsep}{-\baselineskip}
    \item  ORM - Object relational mapping
    \item  Genopfriskning - klasser, simpel personklassen
    \item  Genopfriskning SQL - tabel for personklassen
    \item  Primary key
    \item  Funktioner til indlæsning og serialisering af person
\end{itemize} \end{slide} \begin{slide}\begin{center} {\large Plan for i dag } \end{center} \begin{itemize} \addtolength{\itemsep}{-\baselineskip} \item  Opsamling fra sidst \item  Bits, bytes og talsystemer \item  Objekter og databaser \item  Informationteori
\end{itemize} \end{slide} \begin{slide}\begin{center} {\large Informationsteori } \end{center} \begin{itemize} \addtolength{\itemsep}{-\baselineskip}
    \item  Hvad er kodning, entropi
    \item  Kommunikationskanaler, information og bits
    \item  Tyve spørgsmål
    \item  Grænse for beregningskompleksitet for sortering ved sammenligning via entropi af ordning
    \item  Indkodning, kodetræer, grådige algoritmer, Huffman-kodning
	\end{itemize}
\end{slide}


\end{document}
