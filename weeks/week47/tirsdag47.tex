\documentclass[a4paper,landscape]{slides} 
%\usepackage{a4} 
\usepackage[danish]{babel} 
\usepackage[utf8]{inputenc}
\usepackage{textcomp}
\usepackage{amsmath}
\usepackage{amssymb}
\usepackage{amsthm}
\usepackage{graphicx} 
\usepackage{verbatim} 
\usepackage{fancyhdr}
\usepackage{listings} 
\usepackage{url}
\frenchspacing
\pagestyle{plain}

\addtolength{\voffset}{-060pt}
\addtolength{\textheight}{060pt}

\title{Grundlæggende datalogi}
\author{Rasmus Jensen\footnote{\url{grundat@kommunikationogit.dk}}}
\date{2009-11-16}
\lstset{language=python}
\begin{document}
\maketitle
\begin{comment}
* Praktisk info, og opsamling
** Værktøjsdag
** Torsdag - præsentationer
** Strengformattering i python
** Binære søgetræer, træer generelt, hukommelseshierakier

* Tekstuel opmarkering
** Layout vs. indhold. Opdeling efter funktionalitet
** Opmarkeringssprog: HTML, docbook, LaTeX, wiki, pandoc, ...
** Træstruktur af tekstdata
** Tekst versus binære formater
** Typografi, læsbarhed
** Introduktion til LaTeX for rapportskrivning

* Versionskontrol
** VCS - version control systems.
** Unlimited undo samt backup.
** Samarbejdsværktøj
** Tekst vs. binære data
** Cvs, svn, git, mercurial, ...
** Eksempel: svn hosted at code.google.com (http for adgang fra kua's net, diverse GUI, kræver open source)

* Perspektivering og repetition
** Python som programmeringssprog, og programmering generelt
** Web-programmering: simple CGI-scripts for forståelse, - versus CMS
** Algoritmer og model af maskinen som grundforståelse
** Designprincipper og metoder, - inkrementel udvikling, NIH, DRY, modularisering, 3-tier model, præsentation vs. indhold.
** Værktøj, - scripting, kommandoprompt, markupsprog, versionskontrol, ... 
\end{comment}

\begin{slide}
	\begin{center} {\large 
            Plan for i dag
	} \end{center}
	\begin{itemize} \addtolength{\itemsep}{-\baselineskip}
            \item Praktisk/indledning - værktøjsdag
            \item Opmarkeringssprog
            \item Versionsstyring
            \item Opsamling og perspektivering
	\end{itemize}
\end{slide}

\begin{slide}
	\begin{center} {\large 
            Praktisk info, og opsamling
	} \end{center}
	\begin{itemize} \addtolength{\itemsep}{-\baselineskip}
              \item Værktøjsdag
              \item Torsdag - præsentationer
              \item Strengformattering i python
              \item Binære søgetræer, træer generelt, hukommelseshierakier
	\end{itemize}
\end{slide}

\begin{slide}
	\begin{center} {\large 
            Plan for i dag
	} \end{center}
	\begin{itemize} \addtolength{\itemsep}{-\baselineskip}
            \item Praktisk/indledning - værktøjsdag
            \item Opmarkeringssprog
            \item Versionsstyring
            \item Opsamling og perspektivering
	\end{itemize}
\end{slide}

\begin{slide}
	\begin{center} {\large 
        Tekstuel opmarkering
	} \end{center}
	\begin{itemize} \addtolength{\itemsep}{-\baselineskip}
              \item Layout vs. indhold. Opdeling efter funktionalitet
              \item Opmarkeringssprog: HTML, docbook, LaTeX, wiki, pandoc, ...
              \item Træstruktur af tekstdata
              \item Tekst versus binære formater
              \item Typografi, læsbarhed
              \item Introduktion til LaTeX for rapportskrivning
	\end{itemize}
\end{slide}

\begin{slide}
	\begin{center} {\large 
            Plan for i dag
	} \end{center}
	\begin{itemize} \addtolength{\itemsep}{-\baselineskip}
            \item Praktisk/indledning - værktøjsdag
            \item Opmarkeringssprog
            \item Versionsstyring
            \item Opsamling og perspektivering
	\end{itemize}
\end{slide}

\begin{slide}
	\begin{center} {\large 
        Versionskontrol
	} \end{center}
	\begin{itemize} \addtolength{\itemsep}{-\baselineskip}
              \item VCS - version control systems.
              \item Unlimited undo samt backup.
              \item Samarbejdsværktøj
              \item Tekst vs. binære data
              \item Cvs, svn, git, mercurial, ...
              \item Eksempel: svn hosted at code.google.com (http for adgang fra kua's net, diverse GUI, kræver open source)
	\end{itemize}
\end{slide}

\begin{slide}
	\begin{center} {\large 
            Plan for i dag
	} \end{center}
	\begin{itemize} \addtolength{\itemsep}{-\baselineskip}
            \item Praktisk/indledning - værktøjsdag
            \item Opmarkeringssprog
            \item Versionsstyring
            \item Opsamling og perspektivering
	\end{itemize}
\end{slide}

\begin{slide}
	\begin{center} {\large 
        Perspektivering og repetition
	} \end{center}
	\begin{itemize} \addtolength{\itemsep}{-\baselineskip}
              \item Python som programmeringssprog, og programmering generelt
              \item Web-programmering: simple CGI-scripts for forståelse, - versus CMS
              \item Algoritmer og model af maskinen som grundforståelse
              \item Designprincipper og metoder, - inkrementel udvikling, NIH, DRY, modularisering, 3-tier model, præsentation vs. indhold.
              \item Værktøj, - scripting, kommandoprompt, markupsprog, versionskontrol, ... 
	\end{itemize}
\end{slide}


\end{document}
