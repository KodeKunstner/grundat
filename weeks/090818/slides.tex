\documentclass[a4paper,landscape]{slides} 
%\usepackage{a4} 
\usepackage[danish]{babel} 
\usepackage[utf8]{inputenc}
\usepackage{textcomp}
\usepackage{amsmath}
\usepackage{amssymb}
\usepackage{amsthm}
\usepackage{graphicx} 
\usepackage{verbatim} 
\usepackage{fancyhdr}
\usepackage{listings} 
\usepackage{url}
\frenchspacing
\pagestyle{plain}

\addtolength{\voffset}{-060pt}
\addtolength{\textheight}{060pt}

\title{Grundlæggende datalogi}
\author{Rasmus Jensen\footnote{\url{rasmusjensen@solsort.dk}}}
\date{20090818}
\lstset{language=python}
\begin{document}
%\maketitle

\begin{slide}
	\begin{center} {\large 
            Kursusoversigt
	} \end{center}
	\begin{itemize} \addtolength{\itemsep}{-\baselineskip}
    		\item Introduktion til programmering via Python og billedbehandling. Iteration og betingelser, funktioner, programdesign, dokumentation, debugging, kodeskik, billeder.
    		\item Mere om programmering i Python -- lister, tekst, hashtabeller, objekter, funktionsorienteret programmering. Sprogabstraktion og repetetion ved introduktion af JavaScript.
    		\item Webprogrammering og databaser. CGI, 3-tier model, SQL. YouText.
    		\item Algoritmer og datastrukturer. Lister, stakke, grafer, beregningskompleksitet, prioritetskøer. Evt. statistisk stavekontrol.
	\end{itemize}
\end{slide}




\end{document}
