\documentclass[a4paper,landscape]{slides} 
%\usepackage{a4} 
\usepackage[danish]{babel} 
\usepackage[utf8]{inputenc}
\usepackage{textcomp}
\usepackage{amsmath}
\usepackage{amssymb}
\usepackage{amsthm}
\usepackage{graphicx} 
\usepackage{verbatim} 
\usepackage{fancyhdr}
\usepackage{listings} 
\usepackage{url}
\frenchspacing
\pagestyle{plain}

\addtolength{\voffset}{-060pt}
\addtolength{\textheight}{060pt}

\title{Grundlæggende datalogi}
\author{Rasmus Jensen\footnote{\url{grundat@kommunikationogit.dk}}}
\date{2009-09-03}
\lstset{language=python}
\begin{document}
\maketitle


\begin{slide}
	\begin{center} {\large 
            Plan for i dag
	} \end{center}
	\begin{itemize} \addtolength{\itemsep}{-\baselineskip}
    		\item[08:15] Velkommen, opsamling fra sidst, introduktion til programmering, billed-koordinatsystem
    		\item[09:00] Praktisk om kurset, pixels, gennemløb af billede
    		\item[09:45] Repræsentation af data i computeren, farvemodeller, mere om billeder
    		\item[10:45] Dokumentation, opsummering
    		\item[11:45] Sammenfatning, næste gang...
	\end{itemize}
\end{slide}

\begin{slide}
	\begin{center} {\large 
            Velkommen og god morgen
	} \end{center}
	\begin{itemize} \addtolength{\itemsep}{-\baselineskip}
		\item Plan for i dag
		\item Introduktion til programmering
		\item JES
                \item Billeder, koordinater og øvelse
	\end{itemize}
\end{slide}

\begin{slide}
	\begin{center} {\large 
            Introduktion til programmering
	} \end{center}
	\begin{itemize} \addtolength{\itemsep}{-\baselineskip}
		\item Kald af funktioner ()
		\item Datatyper og beregninger
		\item Variable
		\item Definition af funktioner
	\end{itemize}
\end{slide}

\begin{slide}
	\begin{center} {\large 
            Definition af funktioner
	} \end{center}
	\begin{itemize} \addtolength{\itemsep}{-\baselineskip}
		\item Indrykning har betydning
		\item Parametre
		\item Lokale variable
	\end{itemize}
\end{slide}

\begin{slide}
	\begin{center} {\large 
            JES
	} \end{center}
	\begin{itemize} \addtolength{\itemsep}{-\baselineskip}
		\item Fortolker
		\item Editor
		\item Hjælp
		\item Python-filer
	\end{itemize}
\end{slide}

\begin{slide}
	\begin{center} {\large 
            Billeder og koordinator
	} \end{center}
	\begin{itemize} \addtolength{\itemsep}{-\baselineskip}
		\item (0,0) i øverste venstre hjørne
		\item Omvendt y-akse 
		\item NB: (0,0) vs. (1,1) 1.ed vs 2.ed
	\end{itemize}
\end{slide}

\begin{slide}
	\begin{center} {\large 
            Øvelse: lav funktion der tegner en trekant 
	} \end{center}
	\begin{itemize} \addtolength{\itemsep}{-\baselineskip}
    		\item[08:15] Velkommen, opsamling fra sidst, introduktion til programmering, billed-koordinatsystem
    		\item[09:00] Praktisk om kurset, pixels, gennemløb af billede
    		\item[09:45] Repræsentation af data i computeren, farvemodeller, mere om billeder
    		\item[10:45] Dokumentation, opsummering
    		\item[11:45] Sammenfatning, næste gang...
	\end{itemize}
\end{slide}

\begin{slide}
	\begin{center} {\large 
            Praktisk om kurset, pixels, gennemløb af billede
	} \end{center}
	\begin{itemize} \addtolength{\itemsep}{-\baselineskip}
		\item Opsamling på øvelse med trekant
		\item Overblik over kursusforløb
                \item Praktisk om kurset: litteratur, omfang
	                afleveringsopgaver, eksamen, undervisningsform
                \item Lidt mere om billeder
	\end{itemize}
\end{slide}

\begin{slide}
	\begin{center} {\large 
            Overblik over kursusforløb
	} \end{center}
	\begin{itemize} \addtolength{\itemsep}{-\baselineskip}
		\item Del 1: Grundlæggende programmering, billeder
		\item Del 2: Programmeringsteknikker, (ca. 24/9)
		\item Del 3: Web-programmering (ca. 3/11)
		\item Del 4: Algoritmer og datastrukturer (ca. 17/11)
	\end{itemize}
\end{slide}

\begin{slide}
	\begin{center} {\large 
            Praktisk om kurset
	} \end{center}
	\begin{itemize} \addtolength{\itemsep}{-\baselineskip}
		\item Status for lærebog - forsinket, parallelt, nb.: dikt-noter
		\item Afleveringsopgaver x 7, - afleveringsform
		\item Eksamen
		\item Omfang, undervisningsform
                \item Stil spørgsmål, feedback
	\end{itemize}
\end{slide}

\begin{slide}
	\begin{center} {\large 
            Om Billeder
	} \end{center}
	\begin{itemize} \addtolength{\itemsep}{-\baselineskip}
		\item Pixels
		\item Eksempel med JES-explore, samt GIMP
		\item for-løkker, gennemløb af billede, lav lysere
	\end{itemize}
\end{slide}

\begin{slide}
	\begin{center} {\large 
            Øvelse
	} \end{center}
	\begin{itemize} \addtolength{\itemsep}{-\baselineskip}
		\item Lav billede mørkere, eller endnu lysere
		\item Eksperimenter med at tegne oven på et billede
                    -- enkelt pixels, samt figurer via JES-funktioner
	\end{itemize}
\end{slide}

\begin{slide}
	\begin{center} {\large 
            Videre plan
	} \end{center}
	\begin{itemize} \addtolength{\itemsep}{-\baselineskip}
    		\item[08:15] Velkommen, opsamling fra sidst, introduktion til programmering, billed-koordinatsystem
    		\item[09:00] Praktisk om kurset, pixels, gennemløb af billede
    		\item[09:45] Repræsentation af data i computeren, farvemodeller, mere om billeder
    		\item[10:45] Dokumentation, opsummering
    		\item[11:45] Sammenfatning, næste gang...
	\end{itemize}
\end{slide}

\begin{slide}
	\begin{center} {\large 
            Repræsentation af data, farvemodeller, mere om billeder
	} \end{center}
	\begin{itemize} \addtolength{\itemsep}{-\baselineskip}
		\item Computer, binære tal
		\item Tekst som tal
		\item Farver
	\end{itemize}
\end{slide}

\begin{slide}
	\begin{center} {\large 
            Computere og binære tal
	} \end{center}
	\begin{itemize} \addtolength{\itemsep}{-\baselineskip}
		\item Computer - forvokset regnemaskine
		\item Binære tal, - detaljer
                \item 1 10 100 1000 10000 ....
                \item 1 2 4 8 16 32 64 ...
                \item Øvelse: skriv som binære tal: 100, 42
                \item Øvelse: skriv som decimaltal: 1101 101010
	\end{itemize}
\end{slide}

\begin{slide}
	\begin{center} {\large 
            Tekst som tal
	} \end{center}
	\begin{itemize} \addtolength{\itemsep}{-\baselineskip}
            \item kodning
            \item ord, chr, unichr
            \item Øvelse dit eget navn med tal.
	\end{itemize}
\end{slide}

\begin{slide}
	\begin{center} {\large 
            Farver som tal
	} \end{center}
	\begin{itemize} \addtolength{\itemsep}{-\baselineskip}
		\item Lysstyrke og farvekomponenter
		\item Rød, grøn, blå
		\item Øjets tapceller
		\item Lysfarver - additiv vs. subtraktiv farveblanding
		\item Typisk 8 bit intensiteter per farvekanal
		\item Lav hhv rødt, grønt, blåt, turkist, gult, gråt billede med makeEmptyPicture og makeColor
	\end{itemize}
\end{slide}

\begin{slide}
	\begin{center} {\large 
            Øvelse: lav funktion der ændrer farverne i et billede. Eksempler: lav billedet sort/hvidt eller sepia. Gør billedet varmere eller koldere ved at ændre den røde farveintensitet. Udtræk én af farvekanalerne. Ændre farven baseret på x eller y koordinat eller blanding af disse.
	} \end{center}
	\begin{itemize} \addtolength{\itemsep}{-\baselineskip}
    		\item[08:15] Velkommen, opsamling fra sidst, introduktion til programmering, billed-koordinatsystem
    		\item[09:00] Praktisk om kurset, pixels, gennemløb af billede
    		\item[09:45] Repræsentation af data i computeren, farvemodeller, mere om billeder
    		\item[10:45] Dokumentation, opsummering
    		\item[11:45] Sammenfatning, næste gang...
	\end{itemize}
\end{slide}

\begin{slide}
	\begin{center} {\large 
            Dokumentation
	} \end{center}
	\begin{itemize} \addtolength{\itemsep}{-\baselineskip}
		\item Kommentarer
		\item At læse kildekode
		\item Docstrings
	\end{itemize}
\end{slide}

\begin{slide}
	\begin{center} {\large 
            Opsummering
	} \end{center}
	\begin{itemize} \addtolength{\itemsep}{-\baselineskip}
		\item Funktionsdefinitioner
		\item for-løkker
                \item Binære tal
		\item Billedkoordinater 
		\item Farver
		\item Praktiske informationer
		\item Spørgsmål?
	\end{itemize}
\end{slide}


\begin{slide}
	\begin{center} {\large 
            Øvelse: lav funktion der ændrer farverne i et billede. Eksempler: lav billedet sort/hvidt eller sepia. Gør billedet varmere eller koldere ved at ændre den røde farveintensitet. Udtræk én af farvekanalerne. Ændre farven baseret på x eller y koordinat eller blanding af disse. Husk dokumentation og læsbarhed
	} \end{center}
	\begin{itemize} \addtolength{\itemsep}{-\baselineskip}
    		\item[08:15] Velkommen, opsamling fra sidst, introduktion til programmering, billed-koordinatsystem
    		\item[09:00] Praktisk om kurset, pixels, gennemløb af billede
    		\item[09:45] Repræsentation af data i computeren, farvemodeller, mere om billeder
    		\item[10:45] Dokumentation, opsummering
    		\item[11:45] Sammenfatning, næste gang...
	\end{itemize}
\end{slide}

\begin{slide}
	\begin{center} {\large 
            Sammenfatning, næste gang
	} \end{center}
	\begin{itemize} \addtolength{\itemsep}{-\baselineskip}
		\item Fem minutter, noter hvad I kan huske som det vigtigste - noter til jer selv.
		\item Fem minutter, diskuter det med sidekammerat, lav fælles noter, send kopi per mail.
		\item Næste gang: betingelser, programdesign, debugning. Notesæt med uddrag fra ``How to Think Like a Computer Scientist'' kommer online i eftermiddag/aften.
	\end{itemize}
\end{slide}


\end{document}
