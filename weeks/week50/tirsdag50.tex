\documentclass[a4paper,landscape]{slides} 
%\usepackage{a4} 
\usepackage[danish]{babel} 
\usepackage[utf8]{inputenc}
\usepackage{textcomp}
\usepackage{amsmath}
\usepackage{amssymb}
\usepackage{amsthm}
\usepackage{graphicx} 
\usepackage{verbatim} 
\usepackage{fancyhdr}
\usepackage{listings} 
\usepackage{url}
\frenchspacing
\pagestyle{plain}

\addtolength{\voffset}{-060pt}
\addtolength{\textheight}{060pt}

\title{Grundlæggende datalogi}
\author{Rasmus Jensen\footnote{\url{grundat@kommunikationogit.dk}}}
\date{2009-12-7}
\lstset{language=python}
\begin{document}
\maketitle

\begin{slide} \begin{center} {\large 
Planen for i dag
} \end{center} \begin{itemize} \addtolength{\itemsep}{-\baselineskip}
   \item Praktisk om eksamen
   \item Kurset kogt ned til 15 minutter
   \item Værktøj
   \item Perspektivering
   \item Afrunding
\end{itemize} \end{slide} \begin{slide} \begin{center} {\large 
   Praktisk om eksamen
} \end{center} \begin{itemize} \addtolength{\itemsep}{-\baselineskip}
       \item  Udleveres via absalon torsdag kl. 9. Hviss fejl paa absalon: Jakob ved MEFI's studieadministration
       \item  Normalsidebegrebet: programkode ses som teknisk tekst 1200,  almindelig tekst som almindelig tekst 2400.
        \item Ikke individuel pensumopgivelse.
       \item  Husk gruppeangivelse paa wiki.
\end{itemize} \end{slide} \begin{slide} \begin{center} {\large 
Kurset kogt ned til 15 minutter
} \end{center} \begin{itemize} \addtolength{\itemsep}{-\baselineskip}
   \item Introduktion til Python via multimedier
   \item Webprogrammering og SQL
   \item Data og algoritmer
\end{itemize} \end{slide} \begin{slide} \begin{center} {\large 
    Øvelse
} \end{center} \begin{itemize} \addtolength{\itemsep}{-\baselineskip}
       \item  Review og brainstorm: Lav jeres egen kronologiske gennemgang af kurset, hvilke begreber I er stødt paa, hvad I har arbejdet med i opgaverne etc. Øvelser, forelæsninger, litteratur.
\end{itemize} \end{slide} \begin{slide} \begin{center} {\large 
Planen for i dag
} \end{center} \begin{itemize} \addtolength{\itemsep}{-\baselineskip}
   \item Praktisk om eksamen
   \item Kurset kogt ned til 15 minutter
   \item Værktøj
   \item Perspektivering
   \item Afrunding
\end{itemize} \end{slide} \begin{slide} \begin{center} {\large 
Øvelse i grupper paa tværs af holdet
} \end{center} \begin{itemize} \addtolength{\itemsep}{-\baselineskip}
   \item Diskuter begreber fundet ved forrige øvelse
   \item Lav gruppevis side paa wiki med diskussion
   \item Skiftes i ring
   \item Saml begreber/emner som i ønsker repeteret.
\end{itemize} \end{slide} \begin{slide} \begin{center} {\large 
    Værktøj
} \end{center} \begin{itemize} \addtolength{\itemsep}{-\baselineskip}
       \item  Versionsstyring
       \item  Scripting
       \item  \LaTeX
\end{itemize} \end{slide} \begin{slide} \begin{center} {\large 
Planen for i dag
} \end{center} \begin{itemize} \addtolength{\itemsep}{-\baselineskip}
   \item Praktisk om eksamen
   \item Kurset kogt ned til 15 minutter
   \item Værktøj
   \item Perspektivering
   \item Afrunding
\end{itemize} \end{slide} \begin{slide} \begin{center} {\large 
Øvelse:
} \end{center} \begin{itemize} \addtolength{\itemsep}{-\baselineskip}
   \item Læs og diskuter hinandens aflevering nr. 4/5, hvilke forskelle og hvorfor
\end{itemize} \end{slide} \begin{slide} \begin{center} {\large 
Perspektivering
} \end{center} \begin{itemize} \addtolength{\itemsep}{-\baselineskip}
       \item  Hvad er programmering
       \item  Hvad har vi lært, smagsprøver, hold det ved lige og byg videre.
       \item  Open Source mv. 
\end{itemize} \end{slide} \begin{slide} \begin{center} {\large 
Afrunding
} \end{center} \begin{itemize} \addtolength{\itemsep}{-\baselineskip}
       \item  Evt. fælles kronologisk gennemgang
       \item  Tak for denne gang
\end{itemize} \end{slide}


\begin{comment}
torsdag:
        centralitet
        netværk, shortest path,
        repetition - webapplikationer, sql
\end{comment}


\end{document}
