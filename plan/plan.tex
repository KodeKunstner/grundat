\documentclass[12pt]{article} 
\usepackage{a4} 
\usepackage{makeidx}
\usepackage[danish]{babel} 
\usepackage[utf8]{inputenc}
\usepackage{textcomp}
\usepackage{amsmath}
\usepackage{amssymb}
\usepackage{amsthm}
\usepackage{graphicx} 
\usepackage{verbatim} 
\usepackage{fancyhdr}
\usepackage{listings} 
%\usepackage[colorlinks,pagebackref]{hyperref}
\usepackage{backref}
\usepackage{url}
%\usepackage[paperheight=220mm,paperwidth=170mm]{geometry}
\frenchspacing
\makeindex
\pagestyle{plain}
\newcommand{\Oh}[0]{ \mathcal{O} }
%\addtolength{\voffset}{-78pt}
%\addtolength{\textheight}{73pt}


\title{Udkast til undervisningsplan for \\Grundlæggende Datalogi}

\date{2009}

\begin{document}

\maketitle

\paragraph{Status:}


\subsection{Uge 1 -- Introduktion}
Hvad er datalogi, hvor er det sjovt/praktisk. 
I gang med programmering, introduktion til python og JES.
Start på digitale billeder, hvad er det for noget, oplæg til eksperimentering.
\setcounter{tocdepth}{1}
\tableofcontents
\section{Overblik}
\begin{tabular}{cp{7em}p{15em}p{5em}}
Uge&Emne&Indhold&Litteratur\\
1&Introduktion& variable, simple funktioner & Guz1-3\\
2&Billeder& for, if, dokumentation & Guz3-4, evt. noter\\
3&Billeder& & Guz4-5,13, evt. noter\\
4&Tekst og datastrukturer& & Guz10-11,14 \\
5&Hashtabeller og objekter& & 14, evt. noter\\
6&JavaScript (syntaks vs. semantik), repetition& & Guz10-11,13, evt. noter\\
\end{tabular}
\section{Udgangspunkt og læringsmål}

\section{Overordnet design}
\section{Plan}
\subsection{Introduktion til programmering }
\subsection{Introduktion (uge 1)}

\paragraph{Formål/indhold:}
Komme i gang med programmeringen. Variable og funktioner. Introduktion til billeder.

\paragraph{Forelæsninger:} 
Introduktion til kurset og programmering generelt.
Eksempler på hvor datalogi er sjovt/praktisk.
Introduktion til JES. 
Udtryk, variable, funktionsdefinitioner.

\paragraph{Øvelser:}
Installer JES, at eksperimentere og komme i gang med programmeringen.
Øvelser fra bogen: 2.2, 2.3, 2.4. 1.5, 1.2, 1.3


\subsection{Billeder (uge 1-3) }

Billeder vil være objektet i forbindelse med den videre formidling.

I slutningen af uge 1, introduceres billeder og farver kort, med instruktion i at eksperimentere med disse. 

\subsection{Lister og tekst (uge 4)}


\subsection{Hashtabeller og objekter (uge 5)}
\subsection{JavaScript samt repetition (uge 6)}
\subsection{Introduktion til web-applikationer og SQL (uge 7-8)}
TBD.
\subsection{Algoritmer og datastrukturer(uge 9-12)}
TBD.

\end{document}

