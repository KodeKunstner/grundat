\documentclass[12pt]{article} 
\usepackage{a4} 
\usepackage{makeidx}
\usepackage[danish]{babel} 
\usepackage[utf8]{inputenc}
\usepackage{textcomp}
\usepackage{amsmath}
\usepackage{amssymb}
\usepackage{amsthm}
\usepackage{graphicx} 
\usepackage{verbatim} 
\usepackage{fancyhdr}
\usepackage{listings} 
%\usepackage[colorlinks,pagebackref]{hyperref}
\usepackage{backref}
\usepackage{url}
%\usepackage[paperheight=220mm,paperwidth=170mm]{geometry}
\frenchspacing
\makeindex
\pagestyle{plain}
\newcommand{\Oh}[0]{ \mathcal{O} }
%\addtolength{\voffset}{-78pt}
%\addtolength{\textheight}{73pt}


\title{Udkast til undervisningsplan for \\Grundlæggende Datalogi}

\date{2009}

\begin{document}

\maketitle

\paragraph{Status:} 
Anden halvdel er preliminary.

4.2 skal måske flyttes om efter 5.2, og tilsvarende 5.1 og 5.2 lidt frem, i hvilket fald 4.2 også kan være objekt orienteret.

Mere om afprøvning/test i første halvdel.

Sikr at ting der bliver brugt i anden halvdel også kommer med i første halvdel, eller eksplicit behandles i anden halvdel, dette inkluderer \%, tupler, list comprehension, o.m.a., der ikke dækkes, eller kun nævnes meget overfladisk i a-multimedia-approach.
Også mere fokus på standard libraries, og 0-indeks i stedet for 1-indeks som brugt i billedeksemplerne.

Angående litteratur, kan dele af How to Think Like a Computer Scientist måske tilrettes til noget af algoritme delen.


\setcounter{tocdepth}{2}
\tableofcontents
%\section{Overblik}
%\begin{tabular}{cp{7em}p{15em}p{5em}}
%Uge&Emne&Indhold&Litteratur\\
%1&Introduktion& variable, simple funktioner & Guz1-3\\
%2&Billeder& for, if, dokumentation & Guz3-4, evt. noter\\
%3&Billeder& & Guz4-5,13, evt. noter\\
%4&Tekst og datastrukturer& & Guz10-11,14 \\
%5&Hashtabeller og objekter& & 14, evt. noter\\
%6&JavaScript (syntaks vs. semantik), repetition& & Guz10-11,13, evt. noter\\
%\end{tabular}

\section{Semesterplan}
\subsection{Introduktion til programmering (uge 1.1)}

\subsubsection{Uge 1.1}

\paragraph{Forelæsning:} 
Introduktion til kurset og programmering generelt.
Eksempler på hvor datalogi er sjovt/praktisk.
Introduktion til JES. 
Udtryk, variable, funktionsdefinitioner.

\paragraph{Øvelser:}
Installer JES, at eksperimenter og kom i gang med programmeringen.
Øvelser fra bogen: 2.2, 2.3, 2.4, 2.1, 1.5, 1.2, 1.3 (Så mange som I kan nå)

\paragraph{Litteratur:} Guz 1-2


\subsection{Billeder (uge 1.2-3) }
%\paragraph{Formål/indhold:}
%Grundlæggende programmering: løkker, betingelser, redskaber til programmering.
%Representation af billeder.


\subsubsection{Uge 1.2}
\paragraph{Forelæsning:} 
Introduktion til representation af billeder og farver i computeren. Lidt om dokumentation, docstrings, kommentarer.

\paragraph{Øvelser:}
Øvelser fra bogen: 2.1, 3.1
Lille pixelbaseret tegning, i.e. smiley (forståelse af position af pixels, samt grundlæggende farver) 
Lav et program der genererer eller ændrer et billede, tag eventuelt udgangspunkt i nogle af opgaverne 3.3-3.10

\paragraph{Litteratur:} Guz 2-3, noter om dokumentation

\subsubsection{Uge 2.1}
\paragraph{Forelæsning:} 
For-løkker, ranges, betingelser.
At læse kode/code review. 
Grafikeksempler, opfølgning på afleveringer.

\paragraph{Øvelser:} 
Review and rewrite recipe 28.
Gå sammen i grupper, og kig på afleveringerne fra sidste gang, udleverede kodeeksempler, samt opg. 3.2, 4.2, 4.3, 4.7, 4.8.


\paragraph{Litteratur:} Guz 3-4, noter om (uformelle!) code reviews

\subsubsection{Uge 2.2}
\paragraph{Forelæsning:} 

Diskussion/demo af dele af kapitel 4, filtre, blur.
Opfølgning på øvelser -  lidt om debugging og programdesign.
Kort introduktion til versionskontrolsystemer som værktøj(nb: hav også mere detaljeret webcast om dette).
Variabel størrelse blur, skala.

\paragraph{Øvelser:}
Lav et programmer der ændrer et billeder, foreslag:
Træk en uskarp, formørket udgave af billedet fra det selv. 
Tag udgangspunkt i 4.1, 4.6, og 4.10.
Byg videre på øvelsen i uge 1.2

\paragraph{Litteratur:} Guz 4
\subsubsection{Uge 3.1}
\paragraph{Forelæsning:} 
Opfølgning på øvelser, debugging med udgangspunkt i øvelsen med subtraktion af uskarpt billede.
Debugging og programdesign. Lidt om test
(evt lidt om performance eksemplificeret ved variabel størrelse blur)


\paragraph{Øvelser:}
9.1, 9.2, eksempler på programstumper
\paragraph{Litteratur:} Guz 5, 9


\subsubsection{Uge 3.2}
\paragraph{Forelæsning:} 
Afrunding af billeder, repetition.
(evt fold/map på pixels, eksempel brugerdefineret funktion til ændring af farver, mappet på billede, samt illustreret med histogram)

\paragraph{Øvelser:}
Lav et programmer der ændrer et billeder,
eksempelvis lokal ændring af farver, øget skarphed,
byg videre på øvelsen i uge 2.2
\paragraph{Litteratur:} Guz 1-5 (evt. dele af 9 og 14)

\subsection{Lister og tekst (uge 4)}
\subsubsection{Uge 4.1}
\paragraph{Forelæsning:} 
Lister, for-løkker over lister etc. (opsamling af kendte sprogkonstruktioner på lister i stedet for billeder).
Tekst. Eventuelt struktureret tekst, i.e. xml, (html, tex,) ... Simpel søgning.
Idle.
\paragraph{Øvelser:}
Installer idle.
Generering af tilfældige ord/sætninger
Templatede breve/skriverier
\paragraph{Litteratur:} 10-10.5, 11-11.2

\subsubsection{Uge 4.2}
\paragraph{Forelæsning:} 
Simple træstrukturer, og funktioner der behandler disse, simpel rekursion over struktur. (Baseret på sxml-lignende html-træer)
\paragraph{Øvelser:}
Prettyprint af liste
Udtræk af informationer fra strukturerede data.
\paragraph{Litteratur:} 10-10.5, 11-11.2, 14-14.2

\subsection{Hashtabeller og objekter (uge 5)}
\subsubsection{Uge 5.1}
\paragraph{Forelæsning:} 
Opfølgning på sidste uge
Hashtabeller, lidt om performance
\paragraph{Øvelser:}
Ordfrekvens.
Direkte oversættelse.
\paragraph{Litteratur:} dele af 13, noter, måske fra how to think like a computer scientist

\subsubsection{Uge 5.2}
\paragraph{Forelæsning:} 
Objekter, klasser.
Eksempler: moduler, dom.
\paragraph{Øvelser:}
Måske: Dataudtræk øvelse fra uge 4.2, denne gang på dom, samt  14.4, 14.7, 14.8. Eller evt. kombination af at hente tekstdata fra nettet, og ud fra disse lave et billede.
\paragraph{Litteratur:} Guz 14.3, noter

\subsection{JavaScript samt repetition (uge 6)}
\subsubsection{Uge 6.1}
\paragraph{Forelæsning:} 
Forskellige former for programmeringssprog.
Introduktion til JavaScript, og repetition af python ved at gennemgå kendte konstruktioner i Python, og demonstration af hvorledes de udføres i JavaScript.
\paragraph{Øvelser:}
Reimplementer øvelser fra uge 4.1 i JavaScript.
\paragraph{Litteratur:} Noter, javascript.crockford.com, Guz 16

\subsubsection{Uge 6.2}
\paragraph{Forelæsning:} 
Repetition og perspektivering
\paragraph{Øvelser:}

\paragraph{Litteratur:}

\subsection{Introduktion til web-applikationer og SQL (uge 7-8)}
\subsubsection{Uge 7.1}
\paragraph{Forelæsning:} 
Introduktion og i gang med databaser og webapplikationer.
3-tier model. Grænseflader/api.
\paragraph{Øvelser:}
Få SQL og CGI op at køre.
Simpelt program med at gemme/hente data via SQL.
Python CGI script med random quotes (ikke nødvendigvis SQL).
(Kig på YouText del 1 hvis der er tid.)
\paragraph{Litteratur:} Gamle slides, m.m. -> Noter

\subsubsection{Uge 7.2}
\paragraph{Forelæsning:}
Databaser, grundlæggende SQL.
\paragraph{Øvelser:}
SQL øvelser (i.e. eksempler på SQL kode - hvad gør det, spot en bug, etc.).
YouText del 1, trin 1+2.
\paragraph{Litteratur:} Gamle slides -> Noter 

\subsubsection{Uge 8.1}
\paragraph{Forelæsning:} 
Opsamling på øvelser
Mere om SQL, og den underliggende model
\paragraph{Øvelser:}
Fortsat YouText del 1
\paragraph{Litteratur:} Gamle slides -> Noter 

\subsubsection{Uge 8.2}
\paragraph{Forelæsning:} 
Webapplikationer, html, css, python-cgi
\paragraph{Øvelser:}
YouText del 1, afleveres denne uge.
\paragraph{Litteratur:}

\subsection{Algoritmer og datastrukturer}
\subsubsection{Uge 9.1}
\paragraph{Forelæsning:}
Opfølgning på YouText del 1
Introduktion til beregningskompleksitet
Evt. lidt søgning og sortering
\paragraph{Øvelser:}
\paragraph{Litteratur:}

\subsubsection{Uge 9.2}
\paragraph{Forelæsning:} 
Statistisk stavekontrol
\paragraph{Øvelser:}
Start på YouText del 2
\paragraph{Litteratur:}

\subsubsection{Uge 10.1}
\paragraph{Forelæsning:} 
Lister og stakke
\paragraph{Øvelser:}
\paragraph{Litteratur:}

\subsubsection{Uge 10.2}
\paragraph{Forelæsning:} 
Prioritetskøer
\paragraph{Øvelser:}
\paragraph{Litteratur:}

\subsubsection{Uge 11.1}
\paragraph{Forelæsning:} 
træer
\paragraph{Øvelser:}
\paragraph{Litteratur:}
\subsubsection{Uge 11.2}
\paragraph{Forelæsning:} 
Grafer 
\paragraph{Øvelser:}
\paragraph{Litteratur:}

\subsubsection{Uge 12.1}
\paragraph{Forelæsning:} 
Grafer og grafalgoritmer
\paragraph{Øvelser:}
\paragraph{Litteratur:}
\subsubsection{Uge 12.2}
\paragraph{Forelæsning:} 
Opsamling, 
\paragraph{Øvelser:}
\paragraph{Litteratur:}

\end{document}

