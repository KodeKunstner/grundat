\documentclass[12pt]{article} 
\usepackage{a4} 
\usepackage{makeidx}
\usepackage[danish]{babel} 
\usepackage[utf8]{inputenc}
\usepackage{textcomp}
\usepackage{amsmath}
\usepackage{amssymb}
\usepackage{amsthm}
\usepackage{graphicx} 
\usepackage{verbatim} 
\usepackage{fancyhdr}
\usepackage{listings} 
%\usepackage[colorlinks,pagebackref]{hyperref}
\usepackage{backref}
\usepackage{url}
%\usepackage[paperheight=220mm,paperwidth=170mm]{geometry}
\frenchspacing
\makeindex
\pagestyle{plain}
\newcommand{\Oh}[0]{ \mathcal{O} }
%\addtolength{\voffset}{-78pt}
%\addtolength{\textheight}{73pt}


\title{Udkast til undervisningsplan for \\Grundlæggende Datalogi}

\date{2009}

\begin{document}

\maketitle

\paragraph{Status:}


\setcounter{tocdepth}{2}
\tableofcontents
\section{Overblik}
%\begin{tabular}{cp{7em}p{15em}p{5em}}
%Uge&Emne&Indhold&Litteratur\\
%1&Introduktion& variable, simple funktioner & Guz1-3\\
%2&Billeder& for, if, dokumentation & Guz3-4, evt. noter\\
%3&Billeder& & Guz4-5,13, evt. noter\\
%4&Tekst og datastrukturer& & Guz10-11,14 \\
%5&Hashtabeller og objekter& & 14, evt. noter\\
%6&JavaScript (syntaks vs. semantik), repetition& & Guz10-11,13, evt. noter\\
%\end{tabular}

\section{Semesterplan}
\subsection{Introduktion til programmering (uge 1.1)}

\subsubsection{Uge 1.1}

\paragraph{Forelæsning:} 
Introduktion til kurset og programmering generelt.
Eksempler på hvor datalogi er sjovt/praktisk.
Introduktion til JES. 
Udtryk, variable, funktionsdefinitioner.

\paragraph{Øvelser:}
Installer JES, at eksperimenter og kom i gang med programmeringen.
Øvelser fra bogen: 2.2, 2.3, 2.4, 2.1, 1.5, 1.2, 1.3 (Så mange som I kan nå)

\paragraph{Litteratur:} Guz 1-2


\subsection{Billeder (uge 1.2-3.2) }
%\paragraph{Formål/indhold:}
%Grundlæggende programmering: løkker, betingelser, redskaber til programmering.
%Representation af billeder.


\subsubsection{Uge 1.2}
\paragraph{Forelæsning:} 
Introduktion til representation af billeder og farver i computeren. Lidt om dokumentation, docstrings, kommentarer.

\paragraph{Øvelser:}
Øvelser fra bogen: 2.1, 3.1
Lille pixelbaseret tegning, i.e. smiley (forståelse af position af pixels, samt grundlæggende farver) 
Åben opgave: Lav et program der genererer eller ændrer et billede, tag eventuelt udgangspunkt i nogle af opgaverne 3.3-3.10

\paragraph{Litteratur:} Guz 2-3, noter om dokumentation

\subsubsection{Uge 2.1}
\paragraph{Forelæsning:} 
For-løkker, ranges, betingelser.
At læse kode/code review. 
Grafikeksempler, opfølgning på afleveringer.

\paragraph{Øvelser:} 
Review and rewrite recipe 28.
Gå sammen i grupper, og kig på afleveringerne fra sidste gang, udleverede kodeeksempler, samt opg. 3.2, 4.2, 4.3, 4.7, 4.8.


\paragraph{Litteratur:} Guz 3-4, noter om (uformelle!) code reviews

\subsubsection{Uge 2.2}
\paragraph{Forelæsning:} 

Diskussion/demo af dele af kapitel 4, filtre, blur.
Opfølgning på øvelser -  lidt om debugging og programdesign.
Kort introduktion til versionskontrolsystemer som værktøj(nb: hav også mere detaljeret webcast om dette).
Generalisering af blur, skala.

\paragraph{Øvelser:}
Åben opgave: Lav et programmer der ændrer et billeder, foreslag:
Træk en uskarp, formørket udgave af billedet fra det selv. Tag evt. udgangspunkt i 
4.1, 4.6, og 4.10.
Byg evt. videre på øvelsen i uge 1.2

\paragraph{Litteratur:} Guz 4
\subsubsection{Uge 3.1}
\paragraph{Forelæsning:} 
Opfølgning på øvelser, debugging med udgangspunkt i "Træk en uskarp, formørket udgave af billedet fra det selv."
Smagsprøve på performance via den generaliserede blur.
Debugging og programdesign.

\paragraph{Øvelser:}
9.1, 9.2, 
\paragraph{Litteratur:} Guz 5, 9


\paragraph{Litteratur:} 
\subsubsection{Uge 3.2}
\paragraph{Forelæsning:} 
Afrunding af billeder, repetition.
Evt. start på lister i billedkontekst.
\paragraph{Øvelser:}
Åben opgave: Lav et programmer der ændrer et billeder,
byg evt. videre på øvelsen i uge 1.2
\paragraph{Litteratur:} Guz 1-5,9

\subsection{Lister og tekst (uge 4)}

\subsection{Hashtabeller og objekter (uge 5)}
\subsection{JavaScript samt repetition (uge 6)}
\subsection{Introduktion til web-applikationer og SQL (uge 7-8)}
TBD.
\subsection{Algoritmer og datastrukturer(uge 9-12)}
TBD.

\end{document}

