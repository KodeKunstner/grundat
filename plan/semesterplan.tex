\documentclass[12pt]{article} 
\usepackage{a4} 
\usepackage{makeidx}
\usepackage[danish]{babel} 
\usepackage[utf8]{inputenc}
\usepackage{textcomp}
\usepackage{amsmath}
\usepackage{amssymb}
\usepackage{amsthm}
\usepackage{graphicx} 
\usepackage{verbatim} 
\usepackage{fancyhdr}
\usepackage{listings} 
%\usepackage[colorlinks,pagebackref]{hyperref}
\usepackage{backref}
\usepackage{url}
%\usepackage[paperheight=220mm,paperwidth=170mm]{geometry}
\frenchspacing
\makeindex
\pagestyle{plain}
\newcommand{\Oh}[0]{ \mathcal{O} }
%\addtolength{\voffset}{-78pt}
%\addtolength{\textheight}{73pt}


\title{Semesterplan for \\Grundlæggende Datalogi}

\date{2009}

\begin{document}

\section*{Semesterplan for Grundlæggende Datalogi 2009}

\paragraph{Introduktion til programmering via billedbehandling}
\begin{center} \begin{tabular}{p{4em}p{30em}}
Uge 1&
I gang med programmering (i python).  Udtryk, variable, funktionsdefinitioner.  Repræsentation af billeder i computeren.  Dokumentation og kommentarer i programmer\\
Uge 2&Gennemløb af punkterne i et billede. For-løkker, betingelser. At læse kildekode, code reviews \\
Uge 3&Programdesign, afprøvning, fejlfinding, versionsstyring. Opsamling og repetition \\
\end{tabular}\end{center}

\paragraph{Mere om programmering}
\begin{center} \begin{tabular}{p{4em}p{30em}}
Uge 4&Tekst, lister og hashtabeller. Strukturerede data. Performance \\
Uge 5&Objekter. Objekt-Orienteret Programmering. Rekursion \\
Uge 6&Perspektivering, opsamling og repetition. Forskellige former for programmeringssprog. Syntaks og semantik. JavaScript\\
\end{tabular}\end{center}

\paragraph{Webprogrammering og databaser}
\begin{center}\begin{tabular}{p{4em}p{30em}}
Uge 7& Introduktion til databaser og webapplikationer. 3-tier model. Grundlæggende SQL \\
Uge 8& Mere om databaser og hvorledes de fungerer. Webapplikationer: python-cgi, html og css 
\end{tabular}\end{center}

\paragraph{Algoritmer og datastrukturer}
\begin{center}\begin{tabular}{p{4em}p{30em}}
Uge 9& Statistisk stavekontrol\\
Uge 10& Lister, stakker, introduktion til grafer\\
Uge 11& Grafer og grafalgoritmer. Beregningskompleksitet\\
Uge 12& Prioritetskøer. Opsamling og repetition\\
\end{tabular}\end{center}
\end{document}
